\documentclass{beamer}
%\usepackage[spanish]{babel}
\usepackage[ansinew]{inputenc}
\usepackage{amsmath}
\usepackage{enumerate}
\usepackage{exscale}
\usepackage{indentfirst}
\usepackage{latexsym}
\usepackage{proof}

\usetheme{warsaw}

\newcommand{\lpobjective}[2]{\textsc{Objective:} #1 \[ #2 \]}
\newcommand{\lprestriction}[3]{\textsc{Subject to:} #1 \[ #2 \qquad #3 \]}
\newcommand{\lpineq}[2]{\[ #1 \qquad #2 \]}

\begin{document}

\title{PCP}
\author{Santiago Palladino}
\date{June 2010}

\begin{frame}
\titlepage
\end{frame}

\begin{frame}
\frametitle{Index}
\tableofcontents
\end{frame} 

\setlength{\parskip}{10pt plus 1pt minus 1pt}

\section{Introduction}
\subsection{Problem definition}

\begin{frame}
\frametitle{Partitioned Colouring Problem}

\begin{definition}
Given a graph $G=(V,E,P)$, being $P$ a partitioning of the set of nodes $\{V_1,\ldots,V_p\}$, $G$ is \textit{partition coloured} if exactly one node $v_i$ per partition $V_i$ is coloured and no two adjacents nodes have the same colour.
\end{definition}

As with the standard graph colouring problem, we seek to minimize the number of colours required to partition-colour the graph.

\end{frame} 

\begin{frame} 
\frametitle{Motivation}

Introduced by Li and Sinha in 2000 as a means to solve the Routing and Wavelength Assignment (RWA) problem in Wavelength Division Multiplexed (WDM) optical networks. 

Used to handle wavelength assignment conflicts between lightpaths sharing common fiber links.

Other techniques to solve the RWA problem itself are out of the scope of this work.

\end{frame} 

\subsection{Related work}

\begin{frame} 
\frametitle{Related work}

\begin{description}[Noronha, Ribeiro; 2006]

\item[Li, Simha; 2000]{Greedy one step and two step heuristics}
\item[Noronha, Ribeiro; 2006]{Tabu search heuristic}
\item[Frota et al; 2009]{Branch and cut based on asymmetric representatives formulation}

\end{description}

\pause

We will propose an alternative integer programming formulation of the problem based on Mendez-Diaz and Zabala's model for standard coloring.

\end{frame} 

\section{Model}
\subsection{Model definition}

\begin{frame}
\frametitle{Model}

Every variable $x_ij$ is true if vertex $i$ is colored with color $j$. Variables $w_j$ are true if color $j$ is used in coloring the graph. 

\uncover<1->{
\lpobjective{Minimize sum of colors used}
{\min \sum_{j \in C} w_{j}}
}

\uncover<2->{
\lprestriction{Neighbours shall not have the same colour, and variable $w_j$ must be true if the color is to be used}
{x_{ij} + x_{kj} \leq w_j}{\forall j, \forall (i,k) \in E}
}
\uncover<3->{
\lprestriction{Every \only<3>{vertex}\alert<4>{\only<4>{partition}} has exactly one colour assigned}
{\uncover<4>{\sum _{x_i \in p}} \sum_{j \in C} x_{ij} = 1}{\forall i \in V \uncover<4>{, p \in P}}
}

\end{frame} 

\begin{frame}
\frametitle{Breaking symmetries}

We also take symmetry breaking constraints from the original model.

\lprestriction{Color $j$ cannot be used unless color $j-1$ was used}
{w_{j} \leq w_{j-1}}{\forall j \neq 0 \in C}

\lprestriction{Color $j$ is only used if there is actually a node colored with it}
{{\sum_{i \in V}} x_{ij} \geq w_j}{\forall j \in C}


\end{frame} 


\begin{frame}
\frametitle{Reworking adjacency constraints}

Adjacency constraints can be replaced by either of the following, which improve the relaxation's resolution.

\lprestriction{A node $i_0$ and all of its neighbours in a partition $p_0$ cannot share the same the color}
{\sum_{i \in p_0 \cap N(i_0)} x_{ij_0} + x_{i_0j_0} \leq w_{j_0}}{\forall i_0 \in V, j_0 \in C, p_0 \in P}


\lprestriction{A color $j_0$ may be used on a node $i_0$ or on at most $r$ of its neighbours, being $r$ the number of different partitions in $N(i_0)$}
{\sum_{i \in N(i_0)} x_{ij_0} + r \times x_{i_0j_0} \leq r \times w_{j_0}}{ \forall j_0 \in C, i_0 \in V}

\end{frame} 

\subsection{Valid inequalities}

\begin{frame}
\frametitle{Extended clique inequalities}

Let $K \subseteq V$ an \textit{extended clique} if for every pair $v,w \in K$, either $v$ and $w$ are adjacent or belong to the same partition.

We define the extended clique inequality as
\lpineq{\sum_{i \in K} x_{ij_0} \leq w_{j_0}}{\forall j_0 \in C}

We use a greedy heuristic to find maximal cliques in the graph which violate this inequality.

\end{frame} 

\begin{frame}
\frametitle{Block color inequalities}

Given the symmetry breaking constraints, if a partition is not be coloured with color $j$ then it cannot be coloured using any colour with a higher label.

This allows us to define the block color constraints as:
\lpineq{\sum_{i \in p_0}\sum_{j \geq j_0} x_{ij} \leq w_{j_0}}{\forall p_0 \in P, j_0 \in C}

All these constraints are simply handled by brute force.

\end{frame} 

\begin{frame}
\frametitle{Component independent set inequalities}

Let $I \subseteq V$ be a \textit{component independent set} if for every pair $v,w \in I$, $v$ and $w$ are not adjacent and belong to different partitions.

This allows us to reuse the independent set inequality from the standard coloring problem.

\lpineq{\sum _{i \in I} x_{ij_0} \leq \alpha(I) w_{j_0}}{\forall j_0 \in C}

Which can be strengthened considering symmetry breaking.

\lpineq{\sum _{i \in I} x_{ij_0} + \sum ^n _{j = n - \alpha(I) + 1} \sum _{i \in V} x_{ij} \leq \alpha(I) w_{j_0} + w_{n - \alpha(I) + 1}}
{\forall j_0 \leq n - \alpha(G)}

\end{frame}

\begin{frame}
\frametitle{Hole and path inequalities}

Component independent set inequalities can be specialized as component hole and component path inequalities, using $\lfloor |H|/2 \rfloor$ and $\lceil |P|/2 \rceil$ respectively as their cardinals.

We use a greedy heuristic constructing a path from each node for each colour. The criteria for choosing a node is based on its corresponding $x_ij$ value and whether it has been previously visited or not.

\end{frame}

\begin{frame}
\frametitle{G' indepentent set inequalities}

Given a partitioned graph $G=(V,E,P)$, we define the graph $G'=(V',E')$, with $V' = P$ and $E'$ such that $p_1,p_2 \in V'$ are adjacent iif every node in partition $p_1$ in $G$ is adjacent to every node in partition $p_2$ in $G$. More formally:
\[
V' = P, \qquad E' = \{(p_1,p_2) : p_1,p_2 \in V' \wedge \forall v \in p_1 \forall w \in p_2 : (v,w) \in E \}
\]

This grants another way to reuse independent set inequalities from the standard coloring problem; let $I'$ be an independent set in $G'$, then:

\lpineq{\sum{p \in I'} \sum_{i \in p} x_{ij_0} \leq \alpha(I') w_{j_0}}{\forall j_0 \in C}

\end{frame}


\begin{frame}
\frametitle{Branch and cut}

Branching strategy, primal heuristic, pruning
Using DSATUR
Initial?

\end{frame} 

\begin{frame}
\frametitle{Results}

Results in cut and branch, to determine strategies
Then results in branch and cut, show that there was improvement by applying cuts in more nodes
Show time and node count
Compare vs cplex and brazil, br cannot be compared in terms of time, only node size, be polite

\end{frame} 


\end{document}
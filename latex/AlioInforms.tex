\documentclass{beamer}
%\usepackage[spanish]{babel}
\usepackage[ansinew]{inputenc}
\usepackage{amsmath}
\usepackage{enumerate}
\usepackage{exscale}
\usepackage{indentfirst}
\usepackage{latexsym}
\usepackage{proof}

\usetheme{warsaw}

\newcommand{\lpobjective}[2]{\textsc{Objective:} #1 \[ #2 \]}
\newcommand{\lprestriction}[3]{\textsc{Subject to:} #1 \[ #2 \qquad #3 \]}
\newcommand{\lpineq}[2]{\[ #1 \qquad #2 \]}

\begin{document}

\title[A Branch \& Cut Algorithm for PCP]{A Branch \& Cut Algorithm for the Partitioned Graph Coloring Problem}
\author[Santiago Palladino, Isabel M�ndez-D�az, Paula Zabala]{Santiago Palladino, Isabel M�ndez-D�az, Paula Zabala \\ \{spalladino,imendez,pzabala\}@dc.uba.ar \\ \vskip 10pt \scriptsize{Facultad de Ciencias Exactas y Naturales \\ Universidad de Buenos Aires}}
\institute{ALIO-INFORMS\\Joint International Meeting}
\date{June 2010}

\begin{frame}
\titlepage
\end{frame}

\begin{frame}
\frametitle{Index}
\tableofcontents
\end{frame} 

\setlength{\parskip}{10pt plus 1pt minus 1pt}

\section{Introduction}
\subsection{Problem definition}

\begin{frame}
\frametitle{Partitioned Colouring Problem}

\begin{definition}
Given a graph $G=(V,E,P)$, being $P$ a partitioning of the set of nodes $\{V_1,\ldots,V_p\}$, $G$ is \textit{partition coloured} if exactly one node $v_i$ per partition $V_i$ is coloured and no two adjacents nodes have the same colour.
\end{definition}

As with the standard graph colouring problem, we seek to minimize the number of colours required to partition-colour the graph.

\end{frame} 

\begin{frame} 
\frametitle{Motivation}

Introduced as a means to solve the Routing and Wavelength Assignment (RWA) problem in Wavelength Division Multiplexed (WDM) optical networks.

Used to handle wavelength assignment conflicts between lightpaths sharing common fiber links, minimizing the number of wavelengths used, when all connection requirements are already known.

Other techniques to solve the RWA problem itself are out of the scope of this work.

\end{frame} 

\subsection{Related work}

\begin{frame} 
\frametitle{Related work}

\begin{description}[Noronha, Ribeiro; 2006]

\item[Li, Simha; 2000]{Greedy one step and two step heuristics}
\item[Noronha, Ribeiro; 2006]{Tabu search heuristic}
\item[Frota et al; 2009]{Branch and cut based on asymmetric representatives formulation}

\end{description}

\pause

We will propose an alternative integer programming formulation of the problem based on Mendez-Diaz and Zabala's model for standard coloring.

\end{frame} 

\section{Model}
\subsection{Model definition}

\begin{frame}
\frametitle{Model}

Every variable $x_{ij}$ is true if vertex $i$ is colored with color $j$. Variables $w_j$ are true if color $j$ is used in coloring the graph. 

\uncover<1->{
\lpobjective{Minimize sum of colors used}
{\min \sum_{j \in C} w_{j}}
}

\uncover<2->{
\lprestriction{Neighbours shall not have the same colour, and variable $w_j$ must be true if the color is to be used}
{x_{ij} + x_{kj} \leq w_j}{\forall j, \forall (i,k) \in E}
}
\uncover<3->{
\lprestriction{Every \only<3>{vertex}\alert<4>{\only<4>{partition}} has exactly one colour assigned}
{\uncover<4>{\sum _{x_i \in p}} \sum_{j \in C} x_{ij} = 1}{\forall i \in V \uncover<4>{, p \in P}}
}

\end{frame} 

\begin{frame}
\frametitle{Improving LP}

We also take symmetry breaking constraints from the original model and additional restrictions for eliminating fractional solutions.

\lprestriction{Color $j$ cannot be used unless color $j-1$ was used}
{w_{j} \leq w_{j-1}}{\forall j \neq 0 \in C}

\lprestriction{The label of a color used for a partition cannot exceed $\chi$}
{\sum_{j \in C} w_{j} \geq \sum_{x_i \in p} \sum_{j \in C} x_{ij}}{\forall p \in P}

\end{frame} 


\begin{frame}
\frametitle{Reworking adjacency constraints}

Adjacency constraints can be replaced by either of the following, which improve the relaxation's resolution.

\lprestriction{A node $i_0$ and all of its neighbours in a partition $p_0$ cannot share the same the color}
{\sum_{i \in p_0 \cap N(i_0)} x_{ij_0} + x_{i_0j_0} \leq w_{j_0}}{\forall i_0 \in V, j_0 \in C, p_0 \in P}


\lprestriction{A color $j_0$ may be used on a node $i_0$ or on at most $r$ of its neighbours, being $r$ the number of different partitions in $N(i_0)$}
{\sum_{i \in N(i_0)} x_{ij_0} + r x_{i_0j_0} \leq r w_{j_0}}{ \forall j_0 \in C, i_0 \in V}

\end{frame} 

\subsection{Valid inequalities}

\begin{frame}
\frametitle{Extended clique inequalities}

Let $K \subseteq V$ an \textit{extended clique} if for every pair $v,w \in K$, either $v$ and $w$ are adjacent or belong to the same partition.

We define the extended clique inequality as
\lpineq{\sum_{i \in K} x_{ij_0} \leq w_{j_0}}{\forall j_0 \in C}

We use a greedy heuristic to find maximal cliques in the graph which violate this inequality.

\end{frame} 

\begin{frame}
\frametitle{Block color inequalities}

Given the symmetry breaking constraints, if a partition is not be coloured with color $j$ then it cannot be coloured using any colour with a higher label.

This allows us to define the block color constraints as:
\lpineq{\sum_{i \in p_0}\sum_{j \geq j_0} x_{ij} \leq w_{j_0}}{\forall p_0 \in P, j_0 \in C}

All these constraints are simply handled by brute force.

\end{frame} 

\begin{frame}
\frametitle{Component independent set inequalities}

Let $I \subseteq V$ be a \textit{component independent set} if for every pair $v,w \in I$, $v$ and $w$ are not adjacent and belong to different partitions.

This allows us to reuse the independent set inequality from the standard coloring problem.

\lpineq{\sum _{i \in I} x_{ij_0} \leq \alpha(I) w_{j_0}}{\forall j_0 \in C}

Which can be strengthened considering symmetry breaking.

\lpineq{\sum _{i \in I} x_{ij_0} + \sum ^n _{j = n - \alpha(I) + 1} \sum _{i \in V} x_{ij} \leq \alpha(I) w_{j_0} + w_{n - \alpha(I) + 1}}
{\forall j_0 \leq n - \alpha(G)}

\end{frame}

\begin{frame}
\frametitle{Hole and path inequalities}

Component independent set inequalities can be specialized as component hole and component path inequalities, using $\lfloor |H|/2 \rfloor$ and $\lceil |P|/2 \rceil$ respectively as their cardinals.

We use a greedy heuristic constructing a path from each node for each colour. The criteria for choosing a node is based on its corresponding $x_{ij}$ value and whether it has been previously visited or not.

\end{frame}

\begin{frame}
\frametitle{G' independent set inequalities}

Given a partitioned graph $G=(V,E,P)$, we define the graph $G'=(V',E')$, with $V' = P$ and $E'$ such that $p_1,p_2 \in V'$ are adjacent iif every node in partition $p_1$ in $G$ is adjacent to every node in partition $p_2$ in $G$. More formally:
\[
E' = \{(p_1,p_2) : p_1,p_2 \in V' \wedge \forall v \in p_1 \forall w \in p_2 : (v,w) \in E \}
\]

This grants another way to reuse independent set inequalities from the standard coloring problem; let $I'$ be an independent set in $G'$, then:

\lpineq{\sum{p \in I'} \sum_{i \in p} x_{ij_0} \leq \alpha(I') w_{j_0}}{\forall j_0 \in C}

\end{frame}

\section{Branch \& Cut}

\begin{frame}
\frametitle{Branch and Cut}

Using the model previously defined along with the valid inequalities implemented as cutting planes, we implemented a branch and cut algorithm to tackle this problem, adding initial and primal heuristics and specific branching strategies.

\end{frame} 

\subsection{Cuts}

\begin{frame}
\frametitle{Cutting planes}

We implemented the inequalities in the previous section as cutting planes, tested initially in a Cut and Branch algorithm, and then added to a Branch and Cut structure.

Cuts are applied aggressively on the root of the branching tree and every few nodes on the rest of the tree until a certain depth, in order to keep relaxations easy to solve.

\end{frame}

\begin{frame}
\frametitle{Cutting planes}

The cuts with the best performance were Extended Clique and Block Color, and are therefore the most aggressively applied. 

Component independent set and $G'$ independent set cuts are applied only if a minimum number of the previous cuts is not found.

\end{frame}

\subsection{DSatur}
\begin{frame}
\frametitle{DSatur}

DSatur is an exact method for finding an optimal coloring for a graph by implicitly enumerating all possible colourings.

It is a sequential algorithm in which nodes are chosen based on the \textit{degree of saturation}: the number of different colours used for its neighbours in the current solution. 

Harder nodes (higher 
degree of saturation) are chosen first.

\end{frame} 

\begin{frame}
\frametitle{Extending DSatur}

Classic DSatur is used for standard colouring. We propose two extensions for partitioned colouring:

\begin{itemize}

\item{\textit{Easiest node:} We first select the easiest node (lowest degree of saturation) to be coloured in each partition, and then we apply standard criteria to pick the hardest one from that set.}

\item{\textit{Hardest partition:} We first pick the hardest partition to be coloured according to its degree of saturation, size and uncoloured nodes; and then we pick the easiest node (lowest degree) from that partition.}

\end{itemize}

\end{frame} 

\begin{frame}
\frametitle{Heuristics}

Although DSatur is an exact algorithm, it quickly generates good enough solutions, therefore being a good heuristic by bounding the number of iterations or time.

DSatur proved to be an excellent initial heuristic, sometimes actually arriving to the optimal solution by itself.

We also adapted it to serve as a primal heuristic by fixing all nodes with a high enough value in the relaxation and executing DSatur on the remaining ones.

\end{frame} 

\subsection{Branching}
\begin{frame}
\frametitle{Branching}

We established a static branching strategy for $x_{ij}$ variables prioritizing on:

\begin{itemize}
\item{The higher the number of partitions adjacent to node $i$}
\item{The smaller the size of the partition where node $i$ is}
\item{The higher the colour label $j$}
\end{itemize}

\end{frame}

\begin{frame}
\frametitle{Pruning}

When the number of $x_{ij}$ variables fixed to $1$ during the branching process is high enough, we stop the branch and cut algorithm in that node, pruning the subtree.

The exact solution corresponding to that node is found using DSatur as an exact algorithm, enumerating all possible colourings of the remaining partitions.

\end{frame}

\section{Results}
\subsection{Implementation}

\begin{frame}
\frametitle{Implementation}

The previous algorithm was implemented in Java $1.6$ on top of Cplex $12.1$ and executed on a $2.80$ GHz core with $4$ Gb RAM. 

We implemented both a Cut and Branch and a Branch and Cut algorithm to test the effectiveness of the strategies devised.

Note that work is still in progress and the implementation, much less its parametrization, is not final.

\end{frame}

\begin{frame}
\frametitle{Test Suites}

We built two test suites with random generated graphs:
\begin{itemize}
\item Fixed density of $50\%$, nodes from $60$ to $90$
\item Fixed node count of $80$, density from $20\%$ to $80\%$
\end{itemize}

Partition size for both suites was fixed to $2$. 

Each instance was run for 30 minutes and the solution gap and node count in the branch tree is reported.

\end{frame} 

\subsection{Preliminary Results}

\begin{frame}
\frametitle{Fixed node count}

\begin{tabular}{|c|cc|cc|cc|}
\hline
\multicolumn{1}{|c|}{Graph} & \multicolumn{2}{|c|}{Cplex 12.1} & \multicolumn{2}{|c|}{Branch \& Cut} & \multicolumn{2}{|c|}{Cut \& Branch}
\\
\hline
density & nodes & gap & nodes & gap & nodes & gap
\\
\hline
20\% & 9568 &    0.0 & 1478 &    0.0 & 603 & 0.0
\\
40\% & 736217 & 0.33 & 7685 & 0.43 & 22659 & 0.45
\\
60\% & 190564 & 0.50 & 646 & 0.45 & 4626 & 0.46
\\
80\% & 21736 & 0.50 & 89 & 0.44 & 303 & 0.45
\\
\hline
\end{tabular} 

\end{frame} 

\begin{frame}
\frametitle{Fixed density}

\begin{tabular}{|c|cc|cc|cc|}
\hline
\multicolumn{1}{|c|}{Graph} & \multicolumn{2}{|c|}{Cplex 12.1} & \multicolumn{2}{|c|}{Branch \& Cut} & \multicolumn{2}{|c|}{Cut \& Branch}
\\
\hline
nodes & nodes & gap & nodes & gap & nodes & gap
\\
\hline
60 & 180611 & 0.20 & 2775 & 0.0 & 11148 & 0.0
\\
70 & 495048 & 0.33 & 12866 & 0.34 & 30461 & 0.37
\\
80 & 157471 & 0.43 & 887 & 0.33 & 7941 & 0.47 
\\
90 & 145067 & 0.43 & 606 & 0.47 & 3543 & 0.48 
\\
\hline

 
\end{tabular} 

\end{frame} 

%Results in cut and branch, to determine strategies
%Then results in branch and cut, show that there was improvement by applying cuts in more nodes
%Show time and node count
%Compare vs cplex and brazil, br cannot be compared in terms of time, only node size, be polite



\end{document}

\section{Final Results}

Having fixed the best configurations of the algorithm for binomial random graphs in the previous sections, such as models, initial heuristic, branching strategies, primal heuristic and cuts, we now compare our branch and cut algorithm for \PCP{} against other solutions.

\subsection{Comparison with CPLEX}

The first evaluation to perform is to analyze the improvement introduced by the custom modifications we made on top of the \textsc{cplex} engine, by comparing our results to those returned by an unmodified execution of \textsc{cplex}\footnote{All tests were performed against version 12.1 of \textsc{cplex}.}. 

We used binomial random graphs with 90 nodes, 2 nodes per partition, and picked 2 instances for each node-density pair; with running time of 2 hours.

First, we compared our algorithm to \textsc{cplex}'s default branch and cut, both with and without fixing an initial clique and performing other simplifications to the model (described in section \ref{sec:bnc}). In all cases we provided the same initial solution $\chi_0$, which considerably reduced the number of variables in the model by eliminating those $x_{ij}$ and $w_j$ with $j > \chi_0$.

Then, we performed the same tests, but this time using \textsc{cplex} dynamic search algorithm, instead of the traditional branch and cut we were using. This algorithm, introduced in version 11 of \textsc{cplex} and improved in version 12, uses the same building blocks as traditional branch and cut, but does not allow for user customization via callbacks, therefore working as a black box solver, often yielding better results than its counterpart.

\begin{table}[h]
\centering
\begin{tabular}{|c|cc|cc|cc|}
\hline
\multicolumn{1}{|c|}{Graph} & \multicolumn{2}{|c|}{Cplex branch and cut} & \multicolumn{2}{|c|}{Cplex branch and cut} & \multicolumn{2}{|c|}{Custom \PCP{}}
\\
\multicolumn{1}{|c|}{} & \multicolumn{2}{|c|}{w/o initial clique} & \multicolumn{2}{|c|}{with initial clique} & \multicolumn{2}{|c|}{branch and cut}
\\
\hline
& gap & time & gap & time & gap & time
\\
\hline
EW 20\% N=90 & 0.0\% & 1.49 & 0.0\% & 1.48 & 0.0\% & 0.141
\\
EW 40\% N=90 & 16.7\% & 7200 & 16.7\% & 7200 & 16.7\% & 7200
\\
EW 60\% N=90 & 33.3\% & 7200 & 36.7\% & 7200 & 22.2\% & 7200
\\
EW 80\% N=90 & 26.7\% & 7200 & 23.3\% & 7200 & 11.3\% & 7200
\\
\hline 
\end{tabular} 
\caption{Average gap and running time in seconds for graphs with different densities, comparing our custom \PCP{} branch and cut algorithm with \textsc{cplex}'s default branch and cut, with and without fixing an initial clique for the resolution.}
\label{table:final:cplexbnc}
\end{table}

\begin{table}[h]
\centering
\begin{tabular}{|c|cc|cc|cc|}
\hline
\multicolumn{1}{|c|}{Graph} & \multicolumn{2}{|c|}{Cplex Dynamic Search} & \multicolumn{2}{|c|}{Cplex Dynamic Search} & \multicolumn{2}{|c|}{Custom \PCP{}}
\\
\multicolumn{1}{|c|}{} & \multicolumn{2}{|c|}{w/o initial clique} & \multicolumn{2}{|c|}{with initial clique} & \multicolumn{2}{|c|}{branch and cut}
\\
\hline
& gap & time & gap & time & gap & time
\\
\hline
EW 20\% N=90 & 0.0\% & 0.758 & 0.0\% & 0.758 & 0.0\% & 0.141
\\
EW 40\% N=90 & 16.7\% & 7200 & 16.7\% & 7200 & 16.7\% & 7200
\\
EW 60\% N=90 & 22.2\% & 7200 & 22.2\% & 7200 & 22.2\% & 7200
\\
EW 80\% N=90 & 11.8\% & 7200 & 12.0\% & 7200 & 11.3\% & 7200
\\
\hline 
\end{tabular} 
\caption{Average gap and running time in seconds for graphs with different densities, comparing our custom \PCP{} branch and cut algorithm with \textsc{cplex}'s dynamic search, with and without fixing an initial clique for the resolution.}
\label{table:final:cplexdynamicsearch}
\end{table}

The obtained results showed that the customizations oriented towards solving the \PCP{} did produce an improvement in the solution. The difference with \textsc{CPLEX}'s traditional branch and cut algorithm is remarkable, requiring $10\%$ of the time to solve to optimality in sparse instances, and achieving more than a $10\%$ improvement in graphs with a high density.

As for \textsc{CPLEX} dynamic search, it is clear that it performs much better than its branch and cut, but there are still improvements attained by out algorithm in graphs with very low and high density, in terms of time and gap respectively.

Something interesting to notice is that fixing the initial clique does not always report a benefit when running \textsc{CPLEX} algorithms, even though it did report a significative improvement on our customized algorithm.

\subsection{Comparison with Asymmetric Representatives Branch and Cut}

We also compared our algorithm to the other branch and cut algorithm designed specifically for the \PCP{} we found in the literature: the one devised by Frota, Maculan, Noronha and Ribeiro in \cite{frota2010branch}, based on the asymmetric representatives formulation for traditional coloring, (\cite{campelo2004cliques},\cite{campelo2008asymmetric}).

It is worth noting that both implementations and execution environments differ considerably. While the aforementioned algorithm was run under Linux, implemented in C++ and  using \textsc{XPRESS} to solve linear relaxations, our algorithm was executed in Windows, implemented in Java and built on top of the \textsc{CPLEX} engine using its Java API. This makes both algorithms difficult to compare using the reported results of their respective implementations; nevertheless, we will be presenting this comparison as an informative result.

Even though multiple results are presented in \cite{frota2010branch}, we focused in the number of instances reported to have been solved to optimality in random graphs with $90$ nodes and $2$ nodes per partition, with different edge densities (which is also the kind of graphs in which we focused our testing in these last sections). 

For each density, we executed our algorithm in five instances of binomial (Erdos-R\'enyi) graphs and five instances of powerlaw-cluster\footnote{As described in \ref{subsec:results:instances}, these graphs are generated by three parameters: node count $n$, number of nodes $m$ to which each new node is attached, and probability $p$ to add an extra edge generating a triangle when a new node is added. These graphs are constructed with an initial empty graph of size $m$. In order to attain densities higher than $50\%$ with this kind of graphs, we modified the algorithm to start with a binomial graph of size $m$, and iteratively expand it to the desired node count $n$ using the original procedure.} (Holme-Kim) graphs. Every graph instance was ran until the optimal solution was found, or for up to two hours.

\begin{table}[h]
\centering
\begin{tabular}{|c|rl|rl|rl|}
\hline
Graph & \multicolumn{2}{|c|}{B\&C} & \multicolumn{2}{|c|}{B\&C} & \multicolumn{2}{|c|}{Frota et al.} \\
Density & \multicolumn{2}{|c|}{Binomial} & \multicolumn{2}{|c|}{Holme-Kim} & \multicolumn{2}{|c|}{} \\
\hline
20\% & \b{100\%} & (5/5) & \b{100\%} & (5/5) & 20\% & 3/15 \\
40\% & 0\% & (0/5) & \b{100\%} & (5/5) & 7\% & 1/15 \\
60\% & 0\% & (0/5) & 60\% & (3/5) & \b{80\%} & 12/15 \\
80\% & 0\% & (0/5) & 0\% & (0/5) & \b{100\%} & 15/15 \\
\hline 
\end{tabular} 
\caption{Fraction of the tested instances that were solved to optimality using the implemented branch and cut algorithm on both binomial and powerlaw cluster graphs of different densities, and fraction solved to optimality as reported by Frota et al. in \cite{frota2010branch}.}
\label{table:final:frotaetal}
\end{table}

The obtained results (presented in table \ref{table:final:frotaetal}) are most interesting. Whereas our algorithm outperforms \cite{frota2010branch} in low density graphs, the latter wins in high density graphs. It is also worth noting that our algorithm handles clustered graphs much better than binomial ones, most probably because of the high level of symmetry usually found in binomial graphs.

Also, both algorithms have a tight bound on the difference between the lower bound and the solution found in most cases: for algorithm by Frota et al. this difference is never greater than one color, and in our algorithm it reaches its maximum difference of two colors only in a few high-density binomial graphs.

\vskip 25pt

These results show that different models and different algorithms can tackle the same problem efficiently in different cases. While our algorithm easily solved low density instances, the branch and cut based on the asymmetric representatives model for the \PCP{} performs clearly better in high density graphs. Instances with medium density are still the most difficult to solve, as also happens in standard coloring problems.

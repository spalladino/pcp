\documentclass[12pt]{report}

\usepackage[spanish,english]{babel}
\usepackage[ansinew]{inputenc}
\usepackage{amsmath}
\usepackage{amsthm}
\usepackage{enumerate}
\usepackage{exscale}
\usepackage{indentfirst}
\usepackage{latexsym}
\usepackage[noend]{algorithmic}
\usepackage{algorithm}
\usepackage{tikz} 
\usepackage{rotating}
\usepackage{hyperref}
\usepackage{color}
\usepackage{ulem}
\usepackage{subfig}
\usepackage{multirow}

\usetikzlibrary{backgrounds,fit,arrows} 

% Page size
\addtolength{\textheight}{2cm}

% LPs
\newenvironment{lpmodel}{\subsection{Model:} }{ }
\newcommand{\lpobjective}[2]{\textsc{Objective:} #1 \[ #2 \]}
\newcommand{\lprestriction}[3]{\textsc{Subject to:} #1 \[ #2 \qquad #3 \]}

% Acronyms
\newcommand{\PCP}{\textsc{pcp}}
\newcommand{\TODO}[1]{\textcolor{red}{[\textbf{\textsc{ToDo:} #1}]}}

% Items
\newcommand{\defitem}[2]{\item{\textbf{#1:} #2}}

% Style
\renewcommand{\b}[1]{\textbf{#1}}
\newcommand{\spacedhrule}{\vskip 25pt \hrule \vskip 25pt}

% Math
\newcommand{\sumheight}{\ensuremath{\phantom{\sum_{j \in C}}}}
\newcommand{\ceil}[1]{\ensuremath{\left\lceil #1 \right\rceil}}
\newcommand{\floor}[1]{\ensuremath{\left\lfloor #1 \right\rfloor}}

% Algorithmics
\newcommand{\algorithmiccall}{\textbf{call}\ }
\newcommand{\algorithmiccontinue}{\textbf{continue}\ }
\newcommand{\algorithmicprocedure}{\textbf{procedure}\ }
\newcommand{\algorithmicendprocedure}{\algorithmicend\ \algorithmicprocedure}
\newenvironment{ALC@proc}{\begin{ALC@g}}{\end{ALC@g}}
\newcommand{\PROC}[1]{\algorithmicprocedure #1\ \begin{ALC@proc}}
\newcommand{\CALL}[1]{\STATE \algorithmiccall #1}
\newcommand{\CONTINUE}[1]{\STATE \algorithmiccontinue #1}
\ifthenelse{\boolean{ALC@noend}}
	{\newcommand{\ENDPROC}{\end{ALC@proc}}}
	{\newcommand{\ENDPROC}{\end{ALC@proc}\ALC@it\algorithmicendprocedure}}

% Graphs
\newcommand{\tikzparams}{}

\newenvironment{samplegraphpicture}{
	\begin{tikzpicture}
		[n/.style= {minimum size=3mm,thick,circle,draw=black},
		 b/.style= {minimum size=3mm,thick,circle,draw=black,fill=black},
		 g/.style= {minimum size=3mm,thick,circle,draw=black!40,fill=black!40},
		 nhl/.style= {minimum size=3mm,thick,circle,draw=red,fill=red},
		 nred/.style= {minimum size=3mm,thick,circle,draw=black!40,fill=red!60},
		 nblue/.style= {minimum size=3mm,thick,circle,draw=black!40,fill=blue!60},
		 ngreen/.style= {minimum size=3mm,thick,circle,draw=black!40,fill=green!60},
		 nyellow/.style= {minimum size=3mm,thick,circle,draw=black!40,fill=yellow!60},
		 part/.style={fill=gray!10,circle},
		 llpart/.style={fill=gray!5,circle},
		 hlpart/.style={fill=gray!20,circle}] 
}{
	\end{tikzpicture}
}	

\newenvironment{scaledsamplegraphpicture}[1]{
	\begin{tikzpicture}
		[scale=#1, transform shape,
		 n/.style= {minimum size=3mm,thick,circle,draw=black},
		 b/.style= {minimum size=3mm,thick,circle,draw=black,fill=black},
		 g/.style= {minimum size=3mm,thick,circle,draw=black!40,fill=black!40},
		 nhl/.style= {minimum size=3mm,thick,circle,draw=red,fill=red},
		 nred/.style= {minimum size=3mm,thick,circle,draw=black!40,fill=red!60},
		 nblue/.style= {minimum size=3mm,thick,circle,draw=black!40,fill=blue!60},
		 ngreen/.style= {minimum size=3mm,thick,circle,draw=black!40,fill=green!60},
		 nyellow/.style= {minimum size=3mm,thick,circle,draw=black!40,fill=yellow!60},
		 part/.style={fill=gray!10,circle},
		 llpart/.style={fill=gray!5,circle},
		 hlpart/.style={fill=gray!20,circle}] 
}{
	\end{tikzpicture}
}	

\newcommand{\samplegraphcaption}{}
\newenvironment{samplegraph}[2]{
	\begin{figure}[h]
	\renewcommand{\samplegraphcaption}{\caption{\label{#1} #2}}
	\centering
	\begin{samplegraphpicture}
}{
	\end{samplegraphpicture}
	\samplegraphcaption
	\end{figure}
}

% Configs
\newtheorem{theorem}{Theorem}
\numberwithin{equation}{chapter}
\numberwithin{table}{subsection}
\numberwithin{algorithm}{chapter}
\numberwithin{figure}{chapter}
\setcounter{tocdepth}{1}

\includeonly{heur}

% Document start
\begin{document}

\begin{titlepage}

\begin{center}
\large{Universidad de Buenos Aires \\ Facultad de Ciencias Exactas y Naturales \\ Departamento de Computaci�n}
\vskip 40pt

\huge{\sffamily{Tesis de Licenciatura\\}}
\vskip 4pt
\large{Santiago Miguel Palladino}
\vskip 30pt
\hrule
\vskip 25pt
\huge{\sffamily{Algoritmo de Branch and Cut para el Problema de Coloreo Particionado}}
\vskip 25pt
\hrule
\vskip 30pt

\Large{\textbf{Directores}}
\vskip 5pt
\large{Isabel M�ndez D�az\\Paula Zabala}
\vskip 50pt
\Large{Mayo 2011}
\end{center}
\end{titlepage}
\begin{otherlanguage}{spanish}
\begin{abstract}

El problema de coloreo particionado, \PCP{}, es una generalizaci�n del cl�sico problema de coloreo de grafos. En esta variante el conjunto de nodos del grafo de entrada se encuentra particionado, y el problema consiste en colorear un solo nodo por partici�n utilizando la menor cantidad de colores posible, manteniendo la restricci�n de que dos nodos adyacentes no pueden compartir color.

Este problema fue propuesto por Li y Simha en el contexto del problema de \textit{ruteo y asignaci�n de longitudes de onda} (RWA) en redes \textit{multiplexadas por divisi�n de longitud de onda} (WDM). Dichos autores proponen una resoluci�n en dos etapas: una primera en la que se generan posibles caminos como soluciones factibles para el problema de ruteo, y una segunda en la que se determinan los caminos a usar y se les asignan longitudes de onda, buscando minimizar la cantidad de longitudes de onda usadas. Esta �ltima etapa se corresponde con una instancia del \PCP{}.

El \PCP{}, al igual que coloreo tradicional de grafos, es un problema NP completo, con lo que no se conoce un algoritmo que pueda resolverlo en tiempo polinomial. Por este motivo, la mayor�a de los enfoques para resolver este problema se basan en t�cnicas heur�sticas, dejando poco lugar a algoritmos exactos para la resoluci�n del mismo.

En este trabajo modelamos el \PCP{} como un problema de programaci�n lineal entera, generalizando el modelo propuesto por M�ndez-D�az y Zabala para coloreo de grafos, lo que nos permite resolverlo mediante la t�cnica de \textit{branch and cut}. Para ello, desarrollamos una heur�stica inicial, una heur�stica primal, estrategias de branching, y algoritmos de separaci�n para distintas familias de desigualdades v�lidas que caracterizamos para el poliedro. A partir de estos componentes implementamos el algoritmo de \textit{branch and cut} para la resoluci�n del \PCP{}.

\end{abstract}
\end{otherlanguage}
\begin{abstract}

The partitioned graph coloring problem, \PCP{}, is a generalization of the classic graph coloring problem. In this variant the set of the input graph's nodes is partitioned, and the problem relies in coloring exactly one node per partition using the lowest possible number of different colors, maintaining the constraint that two adjacent nodes may not use the same color.

This problem was first stated by Li and Simha in the context of the \textit{routing and wavelength assignment} (RWA) problem in \textit{wavelength division multiplexed} (WDM) networks. The authors propose a two-stage resolution: a first stage in which possible \textit{lightpaths} are generated, which are feasible solutions to the routing problem, and a second stage where the lightpaths to be used are selected and each of them is assigned a wavelength, looking to minimize the number of different wavelengths. This last stage can be modelled as an instance of the \PCP{}.

The \PCP{}, like traditional graph coloring, is an NP complete problem, which means that there are no polynomial algorithms known for its resolution. Therefore, most of the work on this problem in the literature is targeted towards heuristic approaches, with only a few efforts for developing exact algorithms.

In this work we modelled the \PCP{} as an integer linear programming problem, generalizing the model proposed by M�ndez-D�az and Zabala for graph coloring, which can be solved via \textit{branch and cut} algorithms. In order to develop such an algorithm, we implemented an initial heuristic, a primal heuristic, branching strategies and separation algorithms for the families of valid inequalities we found; these components were the building blocks for our \textit{branch and cut} algorithm for solving the \PCP{}.

\end{abstract}

\tableofcontents
\clearpage

%!TEX root = pcp.tex

\section{Introduction}
\label{sec:introduction}

Needless to say, graphs are widely used for modeling different scenarios in multiple areas of expertise, as well as for solving problems on those scenarios by translating them into well-known problems. 

One of those problems is the graph coloring problem, which consists in assigning a color to each node in a graph, with the constraint that two adjacent nodes may not have the same color. The objective is to generate a valid coloring using the minimum number of colors.

One of the most famous real life problems which led to the generalization of the graph coloring problem was the \textit{4 colors problem}. In 1852, the question of whether any planar map could be colored using only four colors, in such a way that no two regions sharing a border had the same color, was posed. Modeling neighbour regions as adjacent nodes in a planar graph led to the planar graph coloring problem, which was eventually generalized into coloring a generic graph.

Graph coloring is widely used in multiple applications, such as schedule assignment to solve time incompatibilities, assignment of radio frequencies to prevent interference between neighboring radios, or even assigning variables to registers during the flow of a program.

The coloring of a graph is defined formally as a function that, given an input graph $G = <V,E>$, being $V$ the set of nodes and $E$ the set of undirected edges, assigns each node $v \in V$ to a natural number which represents a color. A \textit{k-coloring} is an assignment which uses exactly $k$ different colors.

\begin{figure}[h]
	\centering
	\begin{tikzpicture} 
		[n/.style= {minimum size=3mm,thick,circle,draw=black!75}] 
		
		\node[n,fill=red!75] (n0) at ( 0,0) {}; 
		
		\node[n,fill=blue!75] (n1) at ( 1,1) {}
			edge [-]	(n0)
		;

		\node[n,fill=green!75] (n2) at ( 2,0) {}
			edge [-]	(n1)
			edge [-]	(n0)
		;

		\node[n,fill=blue!75] (n3) at ( 1,-1) {}
			edge [-]	(n2)
			edge [-]	(n0)
		;
		
	\end{tikzpicture} 
\caption{Sample 3-coloring of a diamond graph.}
	\label{fig:samplecoloring}
\end{figure}

This problem has been proved to be \textit{NP-Complete}, and has been widely studied in the literature, being approached both by heuristic and exact methods for its resolution.

In this work we study a generalization of this problem, which is the \textit{partitioned coloring problem} or \PCP{}.

\subsection{Partitioned graph coloring problem}

A \textit{partitioned graph} is defined as a tuple $G = <V,E,P>$ of $n$ vertices, $m$ edges and $q$ partitions respectively. The set $P$ contains $P_1, \ldots ,P_q$ sets of nodes which constitute a partition of $V$. Therefore, for every node $v \in V$, there is exactly one $P_k \in P$ such that $v \in P_k$, and every $P_i \in P$ is nonempty.

The partitioned coloring problem is defined as an assignment of colors to the nodes of the graph $G$, with the restriction that no two adjacent nodes may have the same color, but requiring only one node per partition to be colored. Once again, the goal is to minimize the number of colors required.

\begin{figure}[h]
	\centering
	\begin{tikzpicture} 
		[n/.style= {minimum size=3mm,thick,circle,draw=black!75}] 
		
		\node[n,fill=black!25] (n0) at ( 0,0) {}; 
		
		\node[n,fill=blue!75] (n1) at ( 1,1) {}
			edge [-]	(n0)
		;

		\node[n,fill=black!25] (n2) at ( 4,0) {}
			edge [-]	(n1)
			edge [-]	(n0)
		;

		\node[n,fill=blue!75] (n3) at ( 3,-1) {}
			edge [-]	(n2)
			edge [-]	(n0)
		;
		
			\begin{pgfonlayer}{background} 
				\node [fill=black!10,circle,fit=(n0) (n1),label=170:$P_1$] {}; 
				\node [fill=black!10,circle,fit=(n2) (n3),label=350:$P_2$] {}; 
			\end{pgfonlayer}
		
	\end{tikzpicture} 
\caption{Sample 1-coloring of a partitioned diamond graph.}
	\label{fig:samplepartitionedcoloring}
\end{figure}

\subsubsection{Complexity}

It is easy to see that when $|P_i| = 1\ \forall P_i \in P$, this is, there is a single node per partition, the partitioned coloring problem is equivalent to the standard graph coloring problem previously mentioned. In terms of complexity classes, PCP belongs to the same class as the standard coloring problem.

\begin{theorem}
The decision version of PCP is NP-Complete.
\end{theorem}

\begin{proof}
We will prove NP-Completeness by proving both belonging to NP and NP-Hard classes.

\begin{itemize}
\item{\textit{NP:} Given an input partitions graph $G = <V,E,P>$ and an assignment of colors for a subset of nodes, checking that the number of colors used is $k$ is trivial, and a simple algorithm such as \ref{alg:pcpvalidity} can easily check the validity of the coloring in polynomial time.}
\item{\textit{NP-Hard:} Any instance of standard graph $k-coloring$ can be converted to an instance of PCP by partitioning the initial graph $G$ in such a way that every partition contains a single node. The solution to the original $k-coloring$ problem is the same as the solution to the constructed \PCP{}. Since standard coloring is NP-Hard, this implies that PCP is NP-Hard as well.}
\end{itemize}

\end{proof}

\begin{algorithm}
\caption{Polynomial time algorithm for checking validity of a partition coloring}
\label{alg:pcpvalidity}
\begin{algorithmic}

\FORALL{partition $p$ in $P$}
	\FORALL{node $v$ in $p$}
		\IF {$v$ has a color $j$ assigned}
			\STATE mark $p$ as colored
			\FORALL {neighbour $u$ to $v$}
				\IF{$u$ has the same color assigned as $v$}
					\RETURN false
				\ENDIF	
			\ENDFOR
		\ENDIF
	\ENDFOR
	\IF {no node $v$ in $P$ was colored}	
		\RETURN false
	\ENDIF	
\ENDFOR

\end{algorithmic}
\end{algorithm}

\subsection{Motivation}

This problem was first proposed by Li and Sinha \cite{Li00thepartition} as part of a two-phase resolution for the offline variant of the min-RWA in WDM optical networks.

A Wavelength Division Multiplexed (WDM) optical network consists in a network in which links are optical fibers capable of transmitting a specified number of different wavelengths. The Routing and Wavelength Assignment (RWA) problem consists in, given a desired set of connections between pairs of nodes, establish routes between those nodes using the network's links.

Every route is composed by a set of consecutive lightpaths. A lightpath is defined as a point to point connection between two adjacent nodes in the network using a certain wavelength. Although there are networks in which the nodes are capable of transforming wavelengths within the same route, we will assume that every route uses the same wavelength across all of its lightpaths; this restriction is known as the \textit{wavelength continuity constraint}.

The second restriction to be satisfied is the \textit{wavelength clash constraint} which imposes that different lightpaths in the same physical link must have different wavelengths. Together with the previous constraint, it is implied that two different routes that share at least one physical link must use different wavelengths.

In the offline or static version of the RWA problem, the set of connections to be established is known beforehand. The counterpart of this version is the \textit{dynamic} RWA in which connections must be satisfied as they are requested in an online fashion. In this work we will take only the former version into consideration.

The goal of the min-RWA is to minimize the number of different wavelengths required to establish all the routes desired. Note that there are multiple criteria that can be used to evaluate the quality of a set of routes, such as the number of lightpaths used for each route, or generating particular traffic patterns. In this work we will be focusing only in optimizing the number of wavelengths, which will eventually lead to the \PCP{}.

\subsubsection{Two-phase resolution}

There are multiple methods, both heuristic and exact, for solving the min-RWA problem. Some of them handle both the routing and the wavelength assignment as the same problem, whereas other methods, such as the one proposed in \cite{Li00thepartition}, use a two-phase approach: a routing phase and an assignment phase.

In the routing phase, a set of potential routes is generated for every pair of nodes to be connected, mostly using shortest-path criteria.

The assignment phase then requires to pick a route from the set of candidates for every connection. The route chosen is assigned a wavelength, in such a way that no pair of routes that share a physical link have the same wavelength. This phase can be transformed into an instance of \PCP{}.

A partitioned graph $G$ can be constructed in the following way:
\begin{itemize}
\item{Every potential route generated in the routing phase is represented by a node $v \in V$.}
\item{Nodes belong to the same partition iff the routes they represent satisfy the same connection request.}
\item{An edge between two nodes $u,v$ is created if the routes share any physical link.}
\end{itemize}

\begin{figure}[h]
	
	\centering
	\begin{tikzpicture} 
		[bend angle=40,
		n/.style= {minimum size=3mm,thick,circle,draw=black!75,fill=black!25}] 
		
		\node[n] (n0) at ( 0,0) {$s_1$}; 
		
		\node[n] (n1) at ( 0,2) {}
			edge [-]	(n0)
		;

		\node[n] (n2) at ( 2,4) {}
			edge [-]	(n1)
		;
		
		\node[n] (n4) at ( 2,0) {}
			edge [-]	(n0)
		;
		
		\node[n] (n5) at ( 4,2) {}
			edge [-]	(n4)
		;
		
		\node[n] (n3) at ( 4,4) {$t_1$}
			edge [-]	(n2)
			edge [-]	(n5)
			edge [<-, bend right, draw=black!50] node[auto] {\textcolor{black!50}{$R_1$}} 	(n0)
			edge [<-, bend left, draw=black!50] node[auto,swap] {\textcolor{black!50}{$R_2$}} (n0)
		;
		
		\node[n] (n6) at ( 2,-2) {$s_2$}
			edge [-]	(n4)
		;
		
		\node[n] (n7) at ( 6,2) {$t_2$}
			edge [-]	(n5)
			edge [<-, bend right, draw=black!50] node[auto] {\textcolor{black!50}{$R_3$}}  (n6)
			edge [<-, bend left, draw=black!50]	node[auto,swap] {\textcolor{black!50}{$R_4$}} (n6) 
		;
		
		\node[n] (n8) at ( 4,-2) {}
			edge [-]	(n6)
		;

		\node[n] (n9) at ( 6,0) {}
			edge [-]	(n8)
			edge [-]	(n7)
		;
		
	\end{tikzpicture} 

		\caption{Sample network in which connections $s_1 \rightarrow t_1$ and $s_2 \rightarrow t_2$ are to be implemented. Potential routes $R_1, R_2$ are proposed for the first, while routes $R_3, R_4$ are proposed for the second one. The corresponding partitioned graph is presented in figure \ref{fig:solvedsamplerouting}.}
		
		\label{fig:samplerouting}
	\end{figure}
	
\begin{figure}[h]

		\centering	
		\begin{tikzpicture} 
			[n/.style= {minimum size=3mm,thick,circle,draw=black!75}] 

			\node[n,fill=blue!25] (n0) at ( 0,0) {$R_1$}; 

			\node[n,fill=black!25] (n1) at ( 0,2) {$R_2$};

			\node[n,fill=black!25] (n2) at ( 5,0) {$R_3$}
				edge [-]	(n1)
			;

			\node[n,fill=blue!25] (n3) at ( 5,2) {$R_4$};

				\begin{pgfonlayer}{background} 
					\node [fill=black!10,circle,fit=(n0) (n1),label=170:$s_1 \rightarrow t_1$] {}; 
					\node [fill=black!10,circle,fit=(n2) (n3),label=350:$s_2 \rightarrow t_2$] {}; 
				\end{pgfonlayer}

		\end{tikzpicture} 
	
		\caption{Conflicts partitioned graph for network from figure \ref{fig:samplerouting}. Routes $R_1$ and $R_2$ satisfy the same connection request, as such, they are contained in the same partition; same happens for $R_3$ and $R_4$. Since routes $R_2$ and $R_3$ share a physical link, the corresponding nodes are adjacent to prevent that they are assigned the same frequency. A 1-coloring, which assigns the same label to $R_1$ and $R_4$ is shown, and the corresponding lightpaths generated are shown in figure \ref{fig:solvedsampleroutingnetwork}.}

		\label{fig:solvedsamplerouting}
\end{figure}

\begin{figure}[h]
	\centering
	\begin{tikzpicture} 
		[bend angle=40,
		n/.style= {minimum size=3mm,thick,circle,draw=black!75,fill=black!25}] 
		
		\node[n] (n0) at ( 0,0) {$s_1$}; 
		
		\node[n] (n1) at ( 0,2) {}
			edge [-]	(n0)
		;

		\node[n] (n2) at ( 2,4) {}
			edge [-]	(n1)
		;
		
		\node[n] (n4) at ( 2,0) {}
			edge [-]	(n0)
		;
		
		\node[n] (n5) at ( 4,2) {}
			edge [-]	(n4)
		;
		
		\node[n] (n3) at ( 4,4) {$t_1$}
			edge [-]	(n2)
			edge [-]	(n5)
			edge [<-, bend right, draw=blue!100] node[auto] {\textcolor{blue!100}{$R_1$}} 	(n0)
			edge [<-, bend left, draw=black!10] node[auto,swap] {\textcolor{black!10}{$R_2$}} (n0)
		;
		
		\node[n] (n6) at ( 2,-2) {$s_2$}
			edge [-]	(n4)
		;
		
		\node[n] (n7) at ( 6,2) {$t_2$}
			edge [-]	(n5)
			edge [<-, bend right, draw=black!10] node[auto] {\textcolor{black!10}{$R_3$}}  (n6)
			edge [<-, bend left, draw=blue!100]	node[auto,swap] {\textcolor{blue!100}{$R_4$}} (n6) 
		;
		
		\node[n] (n8) at ( 4,-2) {}
			edge [-]	(n6)
		;

		\node[n] (n9) at ( 6,0) {}
			edge [-]	(n8)
			edge [-]	(n7)
		;
		
	\end{tikzpicture} 
		
		\caption{Solution for the network presented in figure \ref{fig:samplerouting} using the coloring obtained in \ref{fig:solvedsamplerouting}. Since $R_1$ and $R_4$ were the colored nodes, using the same label, then those are the routes established and lightpaths using that label are created to satisfy the connection requests.}
		\label{fig:solvedsampleroutingnetwork}
	\end{figure}

Each wavelength is represented as a color. The problem then consists in coloring a single node within each partition, this is, assigning a wavelength to a single route from the set of candidates for each connection request. The fact that two nodes may not be colored if they are adjacent guarantees  that no wavelength conflicts may occur between two different lightpaths in the same link.

An example is shown in figures \ref{fig:samplerouting}, \ref{fig:solvedsamplerouting} and \ref{fig:solvedsampleroutingnetwork}.

In this work we will focus on finding an exact solution for the partitioned coloring problem.

\subsection{Previous work}
\label{subsec:previouswork}

\subsubsection{Coloring}

The graph coloring problem has been studied extensively in literature, attempting to find both exact and approximate solutions.

Simplest heuristic approaches consist in greedy algorithms, using different criteria such as \textit{largest-first} \cite{welsh1967upper}, \textit{smallest-last} \cite{matula1972graph} or \textit{degree of saturation} \cite{brelaz1979new}. While the first two rely on a static ordering based on the degree of each node, the last one uses a dynamic ordering based on the number of different colors being used in the neighbourhood of each vertex.

These criteria may also be used in implicit enumeration techniques, which enumerate the possible colorings in the order determined by the chosen criteria. Using a good strategy is vital for finding good solutions as soon as possible, therefore pruning a great number of colorings. The implicit solution tree may be traversed in a BFS, DFS or best bound fashion. All these algorithms eventually find the optimal solution for the problem.

The \textsc{dsatur} enumeration algorithm proposed in \cite{brelaz1979new} has proven to be one of the most efficient implicit enumeration methods for the coloring problem, having several improvements such as \cite{sewell1996improved}.

More complex heuristic algorithms, using different metaheuristics, have also been used for the coloring problem.

There is also extensive work using integer linear programming formulations for the coloring problem by using different models:
\begin{itemize}
\item{In \cite{mehrotra1996column} a column generation approach is used based on an independent set formulation of the problem, in which a binary variable $x_S$ defines whether the independent set $S$ is given a color label or not; this formulation requires a variable for each possible color class in the graph.}
\item{An ILP model for acyclic orientations with path constraints is presented in \cite{figueiredo2005acyclic} and then applied to solve the vertex coloring problem.}
\item{The representatives model presented in \cite{campelo2004cliques} and \cite{campelo2008asymmetric} uses $x_{uv}$ variables which determine whether vertex $v$ \textit{represents} color $u$; having exactly one node represent each color class allows easy symmetry breaking.}
\item{M\'endez-D\'iaz and Zabala \cite{mendez2006branch,mendez2008cutting} developed both branch-and-cut and cutting planes algorithms for a standard formulation of the problem, using $x_{ij}$ variables to specify whether node $i$ used color $j$, and $w_j$ variables as witnesses to whether color $j$ was in use. Several symmetry breaking constraints were added to the model to ensure a fast resolution.}
\end{itemize}

Our goal in this work will be to generalize this last ILP formulation to solve the partitioned coloring problem.

\subsubsection{min-RWA}

Initial techniques to solve the min-RWA problem as a two-stage problem, such as \cite{hyytia14wavelength}, pick a single route for every connection using shortest-path algorithms and then use different heuristics to solve a standard coloring problem in the assignment stage. In \cite{manohar2002routing} the shortest-path routing solution is replaced by a maximum edge disjoint path solution in order to reduce conflicts between routes.

Other approaches to the problem tackle the routing and wavelength assignment as a single problem, without decomposing it in two separate phases. In \cite{skorin2007routing}, for example, bin packing heuristic algorithms are used to handle the problem, whereas \cite{noronha2007random} embeds this heuristic into a genetic evolutionary framework.

An exact approach using an integer programming formulation with column generation is used in \cite{lee2002optimization}, which solves both the routing and the wavelength assignment problems in the same formulation.

\subsubsection{PCP}

Two-stage approaches for min-RWA were improved in \cite{Li00thepartition} by introducing the partitioned coloring problem, which allowed multiple routes to be picked for each connection in the initial phase. In that work, two groups of heuristics were developed: one-step and two-step. The former consists in picking the easiest node to color in every partition, and then picking the hardest one from that set using different criteria (largest-first, smallest-last, color-degree); the latter makes an initial pass picking the easiest nodes in every partition and inducing a non-partitioned graph, onto which a standard heuristic is applied in a second stage.

In \cite{noronha2006routing} the one-step color-degree constructive heuristic is used in a tabu search approach, TS-PCP. Routes are generated in an initial stage using a $k$-EDR constructive procedure, based on the maximum edge disjoint path heuristic by \cite{kleinberg1996approximation}, and the resulting partitioned coloring problem is solved with TS-PCP.

Due to the complexity of the problem, most of the work on PCP is composed by heuristic approaches. However, in \cite{frota2010branch}, a branch and cut algorithm is devised, using an integer linear programming model based on the asymmetric representatives formulation for the standard coloring problem, presented in \cite{campelo2004cliques} and \cite{campelo2008asymmetric}.

\subsection{Definitions}
\label{subsec:intro:definitions}

In this section we will define all concepts and conventions to be used throughout this work: 

\begin{itemize}
	\defitem{Colors}{The set of valid color labels $C = \{1, \ldots, c\}$, where $c$ may be any upper bound to the chromatic number of the graph, such as $n$.}
	\defitem{Graph}{Defined as tuple $<V,E>$ where $V$ is the set containing the $n$ nodes and $E$ contains the $m$ undirected edges.}
	\defitem{Partitioned Graph}{Defined as tuple $<V,E,P>$, being $V$ and $E$ the same sets as above, and $P$ the set of $P_1, \ldots, P_q$ partitions of $V$.}
	\defitem{Partition function}{For every node $v$ in a partitioned graph, $p(v)$ returns the partition that contains that node.}
	\defitem{Neighbourhood}{$N(v)$ is the set of nodes in $V$ adjacent to node $v$.}
	\defitem{Partition Neighbourhood}{$N_P(v)$ is the set of partitions that contain at least one node adjacent to $v$.}
	\defitem{Degree}{$\delta(v)$ is the cardinal of the neighbourhood of $v$.}
	\defitem{Partition Degree}{$\delta_P(v)$ is the cardinal of the partition neighbourhood of $v$.}
	\defitem{Color Degree}{Number of different colors used in $N(v)$ for a node $v \in V$; also degree of saturation.}
	\defitem{Extended Clique}{Subset of $V$ such that for every pair of nodes $u,v$, either $u$ is adjacent to $v$ or $u$ and $v$ are contained in the same partition.}
	\defitem{Component Independent Set}{Subset of $V$ such that for every pair of nodes $u,v$, $u$ is not adjacent to $v$ and they belong to different partitions.}
	\defitem{Partition Graph}{The \textit{partition graph} $G'$ of a partitioned graph $G$ is a standard graph $G'=<V',E'>$ in which every node $v'_k \in V'$ corresponds to a partition $P_k \in P$, and two nodes $v'_i,v'_j \in V'$ are adjacent if and only if every node in $P_i$ in $G$ is adjacent to every node in $P_j$.}
\end{itemize}
\section{Model}

In this section we will present the BIP formulation for the PCP, as a generalization of the CP model used by M\'endez-D\'iaz and Zabala in \cite{mendez2008cutting}.

\subsection{Formulation}

Let $G = <V,E,P>$ a partitioned graph, being $V$ the set of nodes numbered from $1$ to $n$, $E$ the set of $m$ edges, and $P$ the set of partitions numbered from $1$ to $q$; and let $C$ be the set of color labels numbered from $1$ to $n$.

The standard coloring problem formulation, SCP, uses the following $(n + 1) \times n$ binary variables, where $i \in V$ and $j \in C$:
\begin{itemize}
\item $x_ij$ equals $1$ if and only if the node $i$ is colored with label $j$
\item $w_j$ equals $1$ if there is a node in the graph which uses color $j$
\end{itemize}

The goal is therefore to minimize the total number of colors used, this is, the number of $w_j$ variables set to $1$.

\begin{align}
\text{\textsc{minimize}} \quad & \sum_{j \in C} w_j \label{eqn:obj} \\
\text{\textsc{subject to}} \quad & \sum_{j \in C} x_{ij} = 1 \quad \forall i \in V \label{eqn:colorvertex} \\
& x_{ij} + x_{kj} \leq w_j \quad \forall (i,k) \in E, \; \forall j \in C \\
& x_{ij}, w_{j} \in \{0,1\} \quad \forall i \in V, \; \forall j \in C
\end{align}



%!TEX root = pcp.tex

\section{Valid inequalities}

In this section we will analyze the PCP polyhedron and derive valid inequalities for it, that will be later used in the branch-and-cut algorithm in cutting planes. We will focus our analysis on the most basic formulation presented in section \ref{subsec:model:formulation}, except for some inequalities which will exploit symmetry breaking restrictions from \ref{subsec:model:symmetry}. 

\begin{align}
\text{\textsc{minimize}} \quad & \sum_{j \in C} w_j \nonumber \\
\text{\textsc{subject to}} \quad & \sum_{i \in P_k} \sum_{j \in C} x_{ij} = 1 \quad \forall P_k \in P \nonumber \\
& x_{ij} + x_{kj} \leq w_j \quad \forall (i,k) \in E, \; \forall j \in C \sumheight \nonumber \\
& x_{ij}, w_{j} \in \{0,1\} \quad \forall i \in V, \; \forall j \in C \sumheight \nonumber
\end{align}

Note that any valid inequalities found for the basic formulation are still valid for any other strengthened models, so the derived inequalities can still be used as cuts regardless of the model implemented.

\subsection{Extended clique inequalities}

A classical inequality for the standard coloring problem is the clique inequality, which establishes that within a clique $K$, at most one node can be colored with a label $j$.

\begin{equation}
\nonumber
\sum_{i \in K} x_{ij} \leq w_{j} \quad \forall j \in C
\end{equation}

Combining this inequality with the fact that in PCP at most one node per partition can be colored with a label $j$, we define the \textit{extended clique inequalities} for PCP. Recall from \ref{subsec:intro:definitions} that an extended clique is a maximal subset $K_P$ of $V$ such that every pair of nodes is either adjacent or belong to the same partition.

\begin{equation}
\label{ineq:extendedclique}
\sum_{i \in K_P} x_{ij} \leq w_{j} \quad \forall j \in C
\end{equation}

Similar inequalities were developed by \cite{frota2010branch}, based on the asymmetric representatives formulation, but are generated only on component cliques\footnote{Clique in which every node belongs to a different partition.}. Extended cliques have the added benefit of covering a larger set of nodes, and maintain their effectiveness regardless of the partition size used.

\subsection{Component independent set inequalities}

As was defined in \ref{subsec:intro:definitions}, a component independent set $I_P$ is a standard independent set with the added restriction that every node must belong to a different partition. This allows to import the independent set inequality directly from the standard coloring problem.

\begin{equation}
\label{ineq:ciset}
\sum _{i \in W} x_{ij} \leq \alpha_P(W) w_{j} \quad \forall j \in C
\end{equation}

The restriction is applied to a subgraph of $G$ induced by the nodes $W \subseteq V$. Since the cardinal of the maximum component independent set of the subgraph, $\alpha_P(W)$, is not easy to calculate, being as difficult as the coloring problem trying to solve, this inequality is applied to particular subsets of the graph with an $\alpha_P$ easy to obtain.

\subsubsection*{Component hole inequalities}

A simple instantiation of the previous inequality is picking a subset $W$ that induces a component hole $H$\footnote{A component hole is a chordless cycle in which every node belongs to a different partition.} in the partitioned graph. As in a standard hole, it holds that $\alpha(H) = \floor{n / 2}$, where $n$ is the length of the hole, therefore the only effort required lies in finding a hole and not in calculating its $\alpha$.

Therefore, given a component hole $H$ in the partitioned graph, the component hole inequality is:

\begin{equation}
\label{ineq:chole}
\sum _{i \in H} x_{ij} \leq \floor{n / 2} w_{j} \quad \forall j \in C
\end{equation}


\subsubsection*{Component path inequalities}

Similar to the previous case, the component independent set can be instantiated with a component path $P$, which is a standard path where every node belongs to a different partition. In this case, it holds that for every component path of length $n$, $\alpha(P) = \ceil{n / 2}$, and the inequality results:

\begin{equation}
\label{ineq:cpath}
\sum _{i \in P} x_{ij} \leq \ceil{n / 2} w_{j} \quad \forall j \in C
\end{equation}

\subsubsection*{Strengthening by breaking symmetry}

Component independent set inequalities can be strengthened by taking into consideration symmetry breaking constraints \ref{eqn:lowerlabel} $w_j \geq w_{j+1}$. In case a component independent of size $\alpha_P$ set is colored with label $j^* \leq q - \alpha_P$, then it is possible to ensure that the colors with the highest label will be left unused, since there are $\alpha_P$ nodes using the same color $j^*$.

Therefore, in the worst case, in which all nodes in $V \setminus W$ use different colors, then an assignment as the one shown in table \ref{table:cisetcoloring} will occur. This coloring...

\begin{table}
\centering	
\begin{tabular}{cc}
\hline
\textbf{color label} & \textbf{partitions count} \\
\hline
$j_0$ & $1$\\
$j_1$ & $1$\\
\vdots & \vdots \\
$j^*$ & $\alpha_P$ \\
\vdots & \vdots \\
$j_{q - \alpha_P}$ & $1$\\
$j_{q - \alpha_P + 1}$ & $1$\\
$j_{q - \alpha_P + 2}$ & $0$\\
\vdots & \vdots \\
$j_{q}$ & $0$\\
\hline
\end{tabular}
\caption{Worst-case color assignment when a component independent set of size $\alpha_P$ is found in the partitioned graph.}
	\label{table:cisetcoloring}
\end{table}

Should these restrictions be included in the model, then \ref{ineq:ciset} can be rewritten for an induced component independent set $W$ as the following:

\begin{equation}
\label{ineq:cisetbs}
\sum_{i \in W} x_{ij_0} + \sum ^c _{j = j_t + 1} \sum _{i \in V} x_{ij} \leq \alpha_P(W) w_{j_0} + w_{j_t + 1} \quad \forall j_0 \leq j_t, \; j_t = q - \alpha_P(W)
\end{equation}

Any $(j_t + r)$-coloring, for $r \geq 1$, there are $\alpha_P(W) - r$ partitions repeating color, therefore, the maximum number of partitions that can be colored using labels $j_0, j_t + 1, \ldots, j_t + r$ is $\alpha_P(W) + 1$. 

\TODO{Complete this explanation} 

Both component hole (\ref{ineq:chole}) and component path inequalities (\ref{ineq:cpath}) can be strengthened using this argument.

\subsection{Partitions graph inequalities}

Let $G' = <V',E'>$ be the partitions graph of $G$\footnote{The partitions graph $G'$ of a partitioned graph $G$ is a standard graph $G'=<V',E'>$ in which every node $v' \in V'$ corresponds to a partition $P_v \in P$, and two nodes $u',v' \in V'$ are adjacent if and only if every node in $P_u$ in $G$ is adjacent to every node in $P_v$.}. Most bounds found for coloring $G'$ can be ported to the original $G$ by extending the constraint over every node represented by each $p \in V'$.

A clear example are independent set inequalities. Let $W' \subseteq V'$ a subset of nodes inducing a subgraph in $G'$, then the independent set inequality holds:

\begin{equation}
\label{ineq:gpiset}
\sum _{i \in W'} x_{ij} \leq \alpha(W') w_{j} \quad \forall j \in C
\end{equation}

As before, in inequalities \ref{ineq:cisetbs}, this restriction can be strengthened considering symmetry breaking constraints:

\begin{equation}
\label{ineq:gpisetbs}
\sum_{i \in W'} x_{ij_0} + \sum ^c _{j = j_t + 1} \sum _{i \in V'} x_{ij} \leq \alpha(W') w_{j_0} + w_{j_t + 1} \quad \forall j_0 \leq j_t, \; j_t = q - \alpha(W')
\end{equation}

These constraints over $G'$ can be converted to constraints $G$ by replacing every node $p \in V'$ with the sum over the nodes $v \in P_p$. Let $W \subseteq P$ be the set of partitions represented by the nodes in $W' \in V'$ in $G'$, then:

\begin{equation}
\label{ineq:gpisetbsg}
\sum_{P_k \in W} \sum_{i \in P_k} x_{ij_0} + \sum ^c _{j = j_t + 1} \sum _{i \in V} x_{ij} \leq \alpha(W') w_{j_0} + w_{j_t + 1} \quad \forall j_0 \leq j_t, \; j_t = q - \alpha(W')
\end{equation}

Once again, since the size $\alpha$ of the maximum independent set of a graph is NP-hard to calculate, the subgraph induced by $W'$ is chosen in such a way that this number is trivial to obtain. These inequality is therefore specialized when $W'$ induces both a path and a hole in $G'$, having $\alpha(W)$ equal to $\ceil{|W| / 2}$ and $\floor{|W| / 2}$, yielding \textit{partitions graph path inequalities} and \textit{partitions graph hole inequalities}, respectively.

Comparing partitions graph independent set inequalities with component independent set ones, the former are less frequent than the latter, since $G'$ tends to be less dense as partition sizes increase; however, the former are much stronger as they impose restrictions over all the nodes in the partitions covered, instead of over a single node per partition.

\subsection{Block color inequalities}

Block color inequalities arise from the symmetry breaking constraints \ref{eqn:lowerlabel} $w_j \geq w_{j+1}$. Given a partition $P_k$ and a color $j_0$, every coloring of partition $P_k$ using any color $j > j_0$, requires that color $j_0$ is already used, since \ref{eqn:lowerlabel} implies that color $j$ cannot be used unless all previous colors had ben used. 

As only a single $x_{ij}$ is set in every partition, this is, only one node is painted with a single color, the following inequality holds:

\begin{equation}
\label{ineq:blockcp}
\sum_{j \geq j_0} \sum_{i \in P_k} x_{ij} \leq w_{j_0} \quad \forall P_k \in P, j_0 \in C
\end{equation}

These $c \times q$ inequalities are extremely easy to generate and, as will be analyzed further in this work, have proven to greatly improve the cutting planes scheme.

%!TEX root = pcp.tex

\section{Enumeration algorithms}
\label{sec:heur}

Implicit enumeration algorithms enumerate all possible colorings for the graph, restricting the solution set as much as possible and pruning non-optimal solutions using known bounds. 

\subsection{Classical scheme}

A classical scheme for enumeration algorithms is presented in \ref{alg:enumeration}.

\begin{algorithm}
\caption{Classical coloring implicit enumeration scheme for simple graphs $G = <V,E>$}
\label{alg:enumeration}

\begin{algorithmic}
\CALL color(0,1)

\PROC{color(painted, label)}
	\IF{current coloring is greater than or equal best coloring}
		\STATE prune current solutions subtree
	\ELSIF{\textit{painted} equals to $|V|$}
		\STATE update best coloring with current coloring
	\ELSE
		\STATE pick next uncolored \textit{node} to color		
		\FOR{j = 1 to \textit{label}}
			\IF{can paint \textit{node} with color $j$}
				\STATE assign color $j$ to \textit{node}
				\CALL color(painted+1, label)
				\STATE uncolor \textit{node}
			\ENDIF
		\ENDFOR
		\COMMENT{try coloring \textit{node} with a new label}
		\STATE assign color \textit{label} + 1 to \textit{node}
		\CALL color(painted+1, label+1)
		\STATE uncolor \textit{node}
	\ENDIF
\ENDPROC

\end{algorithmic}
\end{algorithm}

The algorithm picks a node to be colored in each recursive call, attempting to color it with one of the already used labels if possible, and also  assigning a fresh color to it, in order to explore all possible colorings for the graph, though without considering several symmetric solutions. At every iteration, the current branch is checked if it can be pruned; this can be done if the current coloring is using as many labels as the best coloring found by the algorithm, since it implies that the best solution cannot be improved using the current one.

The strategy used for picking the node to be colored in each recursive call gives place to different algorithms. A very simple strategy is to use the degree of the node, coloring nodes with highest degree first, based on the assumption that difficult nodes should be handled first.

Another algorithm, one of the most widely used for this problem, is \textsc{dsatur}\cite{brelaz1979new}. This algorithm always picks the node with the highest degree of saturation\footnote{Number of different colors used in $N(v)$ for a node $v \in V$.}, using different strategies for tie-breaking, such as picking the node with the largest number of uncolored neighbours\cite{sewell1996improved}, and has proved to be one of the best enumeration algorithms available for graph coloring.

\subsection{Enumerating partitioned colorings}

The previous scheme must be modified in order to be used for partitioned coloring, since it does not require to color every node in the graph but a single node per partition.

A simple modification would be to simply re-use the previous scheme, picking a new partition instead of a new node on each recursive call, and iterate over all nodes in the partition as well as over every possible label; this modification is presented in algorithm \ref{alg:pcpenumerationwrong}. 

\begin{algorithm}
\caption{Modification of enumeration scheme for partitioned graphs $G = <V,E,P>$, picking partitions on every call}
\label{alg:pcpenumerationwrong}

\begin{algorithmic}
		\STATE pick next uncolored \textit{partition} to color		
		\FORALL{\textit{node} in \textit{partition}}
			\FOR{j = 1 to \textit{label}}
				\IF{can paint \textit{node} with color $j$}
					\STATE assign color $j$ to \textit{node}
					\CALL color(painted+1, label)
					\STATE uncolor \textit{node}
				\ENDIF
			\ENDFOR
			\COMMENT{try coloring \textit{node} with a new label}
			\STATE assign color \textit{label} + 1 to \textit{node}
			\CALL color(painted+1, label+1)
			\STATE uncolor \textit{node}
		\ENDFOR
\end{algorithmic}
\end{algorithm}

However, this modification forces to pick all the nodes within the same partition together, regardless of the criteria being used to pick nodes. For instance, if a largest-degree criteria is used, and remaining partitions (with their nodes' degrees) are $P_1 \{v_1(10),v_2(1)\}$ and $P_2 \{v_3(5)\}$, the proposed modification would pick nodes $v_1, v_2, v_3$ instead of $v_1, v_3, v_2$. This severely damages the validity of the strategy being used.

Therefore, we propose another modification, presented in algorithm \ref{alg:pcpenumeration}. In this case we use the original enumeration scheme, picking a node from an unpainted partition on every call, but before returning from the recursive call we create another branch in which we do not color the chosen node, so that the partition can be later colored using another node.

\begin{algorithm}
\caption{Partitioned coloring implicit enumeration scheme for partitioned graphs $G = <V,E,P>$}
\label{alg:pcpenumeration}

\begin{algorithmic}
\CALL color(0,1)

\PROC{color(painted, label)}
	\IF{current coloring is greater than or equal best coloring}
		\STATE prune current solutions subtree
	\ELSIF{\textit{painted} equals to $|P|$}
		\STATE update best coloring with current coloring
	\ELSE
		\STATE pick next uncolored \textit{node} to color	from all uncolored partitions	
		\FOR{j = 1 to \textit{label}}
			\IF{can paint \textit{node} with color $j$}
				\STATE assign color $j$ to \textit{node}
				\CALL color(painted+1, label)
				\STATE uncolor \textit{node}
			\ENDIF
		\ENDFOR
		
		\COMMENT{try coloring \textit{node} with a new label}
		\STATE assign color \textit{label} + 1 to \textit{node}
		\CALL color(painted+1, label+1)
		\STATE uncolor \textit{node}
		
		\COMMENT{leave node unpainted}
		\IF{there are other nodes left in the partition}
			\STATE mark \textit{node} as unavailable
			\CALL color(painted, label)
			\STATE mark \textit{node} as available again
		\ENDIF
		
	\ENDIF
\ENDPROC

\end{algorithmic}
\end{algorithm}

It is within this scheme that we embedded our two different strategies based on degree of saturation for partition coloring.

\subsection{Partitioned \textsc{dsatur}}

Classical \textsc{dsatur} picks the node with the highest color degree on each iteration, based on the assumption that nodes difficult to color should be handled first, which usually works well for most heuristics. In the case of partition coloring, as suggested in \cite{Li00thepartition}, nodes with lower degree are easier to color and should be preferred within a partition; also, it is better to color larger partitions first in order to reduce the problem size as early as possible.

Based on these assumptions, we generalized two different versions for partitioned \textsc{dsatur}: \textit{easiest node} and hardest partition.

\subsubsection*{Easiest node}

\TODO{onestepCD heuristic from li}

\subsubsection*{Hardest partition}

\TODO{different criteria used}

%!TEX root = pcp.tex

\chapter{Branch and Cut}
\label{sec:bnc}

In this chapter we will present the implemented branch-and-cut algorithm, including all the components that compose the branch and cut structure, such as initial heuristic, branching strategies, separation algorithms and primal heuristics.

\section{Algorithm structure}

The structure of a branch and cut algorithm can be considered a combination of cutting planes and branch and bound schemes, which we describe below.

\subsection{Cutting planes}

Cutting planes algorithms rely solely on valid inequalities for solving the problem. Given a solution of the model's linear relaxation \footnote{The linear relaxation of an integer linear programming consists in removing all integrality constraints on the variables.}, cutting planes are added to the formulation in order to remove the fractional solution obtained. 

Recall that valid inequalities have the property of being satisfied by all integer solutions of the model, but not necessarily by all fractional solutions in the relaxation; therefore, given a fractional solution $x^*$, there is a cutting plane generated by a linear inequality that holds for all integral solution but is not satisfied by $x^*$. 

The algorithm consists in repeating this process, re-solving the relaxation with all added cuts in each iteration, until an integer-feasible solution is obtained, no more violated inequalities can be found or any defined stopping criteria is achieved.

\begin{algorithm}

\begin{algorithmic}

\LOOP
	\STATE calculate solution of the linear relaxation
	\IF{solution is integer}
		\RETURN obtained solution
	\ELSIF{can generate inequalities to cut off the fractional solution}
		\STATE add the cutting planes to the model
	\ELSE
		\RETURN failure
	\ENDIF	
\ENDLOOP

\caption{General scheme for a cutting planes algorithm}
\label{alg:cuttingplanes}

\end{algorithmic}
\end{algorithm}

The algorithm depends on having an adequate set of cutting planes families, along with good separation heuristics, in order to find valid inequalities that cut off the relaxation's fractional solution on every iteration. Excessive generation of cutting planes may have the drawback that the relaxation becomes larger and larger on every iteration, requiring a greater computational effort to solve. Even worse, the cutting planes families chosen may be such that the algorithm requires an infinite number of iterations to converge to the integer solution, which forces the addition of an algorithm stopping condition based on elapsed time, number of iterations or relative improvement in the solution, among others.

Note that, besides the valid inequalities specific to each problem, such as the ones we have presented for \PCP{} in chapter \ref{sec:ineqs}, there are generic families of cutting planes that can be applied to any problem, like Gomory cuts \cite{gomory1958outline}, disjunctive cuts \cite{balas1993lift}, clique cuts, cover cuts, etc. However, problem-specific cuts usually have better performance than generic ones.

\subsection{Branch and bound}

The branch and bound scheme explores different solutions by setting bounds on fractional variables on every partial solution. The algorithm starts by solving the relaxation, like cutting planes, but instead of removing the fractional solution by adding a valid inequality, it creates two subproblems, usually by fixing a particular variable with a fractional value to either zero or one, in the case of binary integer programming problems in which variables are restricted to these two possible values. 

Each subproblem is then solved using the same strategy until the relaxation's solution is integer feasible, in which case that branch is closed; note that eventually all variables are fixed to an integer value, so an integer solution is always found. The algorithm runs until all nodes are explored.

Besides the \textit{branching} behaviour of the algorithm, both upper and lower \textit{bounds} for each node are considered. A global upper bound (in case of a minimization problem) is updated whenever a new integer feasible solution is found, keeping the one with the lowest objective value. This upper bound is compared with each node's lower bound provided by the relaxation's solution. Since the integer solution will always be greater than the relaxation's, when the lower bound of the node is larger than the global upper bound, the node can be fathomed.

\begin{algorithm}
\begin{algorithmic}

\STATE initialize \textit{tree} with the problem's formulation
\STATE 
\WHILE{there are open nodes in the \textit{tree}}
	\STATE pick an open \textit{node} from the tree
	\STATE calculate solution $x^*$ of the linear relaxation of the \textit{node}
	\IF{the linear relaxation is infeasible}
		\STATE close the node as the branch is infeasible
	\ELSIF{solution $x^*$ is integer feasible}
		\STATE update best integer solution $x_M$ with $x^*$ if $f(x^*) < f(x_M)$
		\STATE close the node as an integer solution was found
	\ELSIF{$f(x^*) > f(x_M)$} 
		\STATE close the node as best solution cannot be improved
	\ELSE
		\STATE generate new subproblem nodes by \textit{branching}
	\ENDIF
\ENDWHILE

\RETURN best integer solution $x_M$

\end{algorithmic}

\caption{General scheme for a branch and bound algorithm}
\label{alg:branchnbound}

\end{algorithm}

The branch and bound scheme does not specify which node to select on each iteration, or how to generate the subproblems on each node. The node selection strategy and the branching strategy are chosen depending on the problem to solve.

\subsection{Cut and branch}

The cut and branch scheme is simply an execution of the cutting plane algorithm until a certain threshold is reached (expressed in running time, number of cuts iterations or mip gap). Then the generation of cutting planes is stopped and a branch and bound algorithm is executed, using the initial formulation augmented with all the generated cutting planes.

This algorithm usually yields better results than the previous ones, as a cutting plane one may not find inequalities that lead to an optimal solution in a finite number of steps, and a pure branch and bound one usually takes too long to solve as the enumeration may be too large. However, since the cutting planes may cause that the relaxations in the branch and bound become tougher to solve, a good balance between the two phases is extremely important for a good overall performance.

It is worth noting that, since the cut and branch algorithm is ultimately reduced to a branch and bound scheme, this scheme always arrives to an optimal solution, unlike a pure cutting planes algorithm which may fail to obtain an integer solution.

\subsection{Branch and cut}

While the cut and branch algorithm applies cutting planes only at the root of the branch and bound tree, the branch and cut version uses cutting planes throughout the whole tree. Although cuts are applied more aggressively at the root, on certain internal nodes more iterations of cuts are executed. 

Note that these generated cuts can use local information, exploiting the variables fixed in the node due to the branching process. In this case, cuts generated on the node are only valid on the node and its subtree; otherwise, cuts generated on an internal node may be reused globally.

An improvement to this scheme consists in deriving integral solutions from node relaxations, obtaining global upper bounds earlier during the traversal, which allow to prune branches earlier. The generation of these solutions is done via \textit{primal heuristics}.

As with all previously mentioned algorithms, branch and cut schemes have a number of parameters and strategies to determine. To begin with, all the separation heuristics for cutting planes iterations must be chosen adequately to quickly find a reasonable number of cuts; on the branch and bound side, node selection and branching strategies must also be determined. Also, the number of cut iterations to perform on the root and on internal nodes must be set, as well as choosing on which nodes cutting planes and primal heuristics should be executed.

Throughout this section we will go over all these missing pieces in the branch and cut structure, and point out how we filled them in the context of the partitioned coloring problem.

\section{Preprocessing}

The first step in solving a \PCP{} instance consists in preprocessing the graph, applying all the following rules until no more modifications are made to the graph:

\begin{enumerate}
	\item{As an initial step, every edge with both ends within the same partition is removed. Since only one node is colored per partition, there can be no color conflicts between nodes of the same partition, and all edges connecting them can be removed, in order to greatly reduce the size of the graph and the number of adjacency restrictions generated.}
	\item{Partitions containing isolated nodes can be completely removed from the graph, as any isolated node can be trivially colored using the lowest possible label, and coloring a single node within a partition marks the whole partition as colored, therefore allowing us to completely remove it.}
	\item{Neighbourhood inclusion criteria is applied within a single partition in order to remove higher-order nodes. Let $u,v$ be two different nodes in a partition $P_k$, if $N(u) \subseteq N(v)$, then we can remove node $v$ from the graph. Since only one node per partition needs to be colored, any valid coloring that assigns color $j_0$ to node $v$ can be modified to assign color $j_0$ to node $u$ instead, still satisfying all color constraints. Intuitively, we are removing \textit{difficult} nodes from a partition when we find an easier one to substitute it. See figure \ref{fig:neighbourinclusion} for an example.}
	
\begin{figure}
	\centering
	\begin{tikzpicture} 
		[includer/.style= {minimum size=3mm,thick,circle,draw=red!75,fill=red!20}, 
		included/.style= {minimum size=3mm,thick,circle,draw=green!75,fill=green!20}, 
		other/.style= {minimum size=3mm,thick,circle,draw=black,fill=black}, 
		transition/.style={thick,draw=black,fill=black!20}] 
		
		\node[other] (n3) at ( -1,3) {}; 
		\node[other] (n4) at ( 0,3) {}; 
		\node[other] (n5) at ( 1,3) {}; 
		\node[other] (n6) at ( 2,3) {}; 
		
		\node[includer] (n1) at ( 0,0) {$v_1$}
			edge [-]	(n3)
			edge [-]	(n4)
			edge [-]	(n5)
			edge [-]	(n6)
		;
		
		\node[included] (n2) at ( 1,0) {$v_2$}
			edge [-]	(n4)
			edge [-]	(n5)
			edge [-]	(n6)
		;
		
	
		\begin{pgfonlayer}{background} 
			\node [fill=black!10,circle,fit=(n1) (n2),label=350:$P_1$] {}; 
		\end{pgfonlayer} 
		
	\end{tikzpicture} 
\caption{Neighbourhood inclusion example: node $v_1$ will be removed from the graph as its neighbourhood completely contains $N(v_2)$.}
	\label{fig:neighbourinclusion}
\end{figure}
	
	\item{A lower bound for the chromatic number of the graph is obtained by finding a maximal clique in the partition graph $G'$. Finding a clique of size $\omega$ in $G'$ implies that at least $\omega$ different colors are needed for coloring the partition graph, and the same lower bound clearly holds for $G$. All partitions in the clique will have their colors fixed to $1,\ldots,\omega$ in order to reduce the number of possible colorings, since each of them must be painted using a different color.}
	\item{As in step 2, partitions that contain at least one node with degree less than the lower bound found in the previous step are removed. A node with strictly less than $\omega$ neighbours can be assigned a color among $1,\ldots,\omega$, knowing that no color conflicts will occur; and since there are already $\omega$ colors required, the chromatic number is not increased by that assignment, and therefore the node can be discarded.}
\end{enumerate}

Last 3 steps are performed until no more changes are made to the graph. The resulting largest clique found is used to fix the colors of the partitions included in it. 

Every step is processed by brute force, since their running time is polynomial in the size of the graph, except for step 4 for which we use the algorithm presented below.

\subsection{Partition graph clique detection}

To find the maximum clique in the partition graph we use a simple backtracking algorithm. Since the running time of this algorithm can be excessive for a preprocessing step, we bound the running time of this algorithm to five seconds; however, the algorithm usually ends much sooner, as the partition graph is not only smaller but also much less dense than the original graph. 

In case the time bound is reached, the best solution found so far is returned. As the backtracking uses DFS to explore all possible solutions, a reasonably good solution is generated early in the algorithm, therefore a valid result is obtained regardless its interruption.

Starting with an initial node, the algorithm keeps a list of valid candidates for the clique, which is updated on each iteration by removing all nodes that are not adjacent to the clique. Keeping both the candidates list and all adjacency lists sorted by degree makes computing the intersection between these lists faster, and produces better initial solutions that can be used to prune other solutions later.

\begin{algorithm}

\begin{algorithmic}

\STATE sort all nodes and adjacency lists descendingly by degree

\FORALL{initial node $v$ in $V$}
	\STATE initialize \textit{clique} with node $v$
	\STATE initialize \textit{candidates} with $N(v)$
	\CALL clique 
\ENDFOR

\PROC{clique}
	\IF{\textit{candidates} is empty}
		\STATE update \textit{best} solution if current \textit{clique} is better
	\ELSIF{\textit{clique}.size + \textit{candidates}.size $\leq$ \textit{best}.size}
		\STATE prune current tree
	\ELSE
		\STATE pop node $u$ from \textit{candidates} and add it to \textit{clique}
		\STATE intersect \textit{candidates} with $N(u)$ and store \textit{removed} nodes
		\CALL clique
		\STATE remove node $u$ from \textit{clique}
		\STATE add \textit{removed} nodes back to \textit{candidates} 
		\CALL clique
	\ENDIF
\ENDPROC

\end{algorithmic}

\caption{Finding a maximum clique in a simple graph $G=<V,E>$}
\label{alg:gpclique}

\end{algorithm} 

\section{Initial heuristic}

A good initial heuristic solution gives an upper bound on the solution, eliminates a large number of variables and restrictions in the model, and can be used as an initial incumbent solution for the branch and cut algorithm. 

In order to generate this solution, we use the modification of the \textsc{dsatur} algorithm for partitioned graphs presented in \ref{sec:heur}. Since the algorithm generates an implicit enumeration of all possible colorings, which might take too long to compute, its running time is bounded to five seconds. The coloring of the partitions in the clique $K$ is fixed to labels $1, \ldots, \omega$ in order to reduce the solutions set.

\section{Initialization}

Using the initial solution as an upper bound $\hat{\chi}$ for the chromatic number, it is possible to eliminate all $x_{ij}$ and $w_j$ variables with $j > \hat{\chi}$, thus greatly reducing the number of involved variables and restrictions in the model.

Another optimization is to fix the colors for all partitions involved in the clique $K$ found during the preprocessing stage. Since it is not possible to determine which node within the partition is to be colored, we simply set to zero all $x_{ij}$ variables for nodes within the partitions that use a different color than the one assigned. Formally, let $K = \{ P_{K_1}, \ldots, P_{K_\omega} \}$ be the initial clique, then:
\begin{align*}
x_{ij} = 0 \quad &\forall i \in K[l],\ \forall 1 \leq l \leq \omega,\ \forall j \neq l \\
w_j = 1 \quad &\forall 1 \leq j \leq \omega
\end{align*}

Also, in case the partition being fixed to a color $j_k$ has a single node in it, then variable $x_{ij_0}$, where $i$ is the single node in the partition, is fixed to $1$.

Another bound based on nodes degree is imposed. A node $v$ of partition degree $\delta_P(v)$ can always be colored with a label $j_k$ such that $1 \leq j_k \leq \delta_P(v) + 1$, since it will be neighbour to at most $\delta_P(v)$ different colors, therefore any set of $\delta_P(v) + 1$ colors contains at least one valid label that does not generate color conflicts.  
\begin{equation*}
x_{ij} = 0 \quad \forall i \in V,\ \forall j > \delta_P(v) + 1 \\
\end{equation*}

Finally, in case minimum partition index breaking symmetry restrictions (\ref{eqn:minlabel}) are being used, this is, the color with the lowest label is assigned to the color class containing the partition with the lowest index, another bound can be imposed on $x_{ij}$ variables. Since in the worst case all partitions will have a different color, then a partition with index $k$ will never be colored with label greater than $k$. Therefore, the following bound is imposed:
\begin{equation*}
x_{ij} = 0 \quad \forall P_k \in P,\ \forall i \in P_k,\ \forall j > k \\
\end{equation*}

\section{Cuts separation}

For each family of valid inequalities listed in \ref{sec:ineqs}, an heuristic is implemented to find a set of valid inequalities being violated by a solution of the linear relaxation of the model. Note that finding a violated inequality in a solution is NP-Complete\footnote{Gr\"{o}tschel, Lov\'{a}sz and Schrijver \cite{grotschel1981ellipsoid} related the complexity of the optimization problem being solved to the complexity of the separation problem: the former is polynomial if and only if the latter is also polynomial.}, so heuristic procedures must be applied. 

Since these algorithms are applied frequently during the branch and cut tree, it is imperative that their running time is as fast as possible, in order to minimize the added overhead to the whole algorithm.

\subsection{Extended clique cuts}

Separation of extended clique cuts (\ref{ineq:extendedclique}) is done using a very similar algorithm to \ref{alg:gpclique}, adapted to partitioned graphs and without backtracking, so running time is acceptable for a separation algorithm. This algorithm is executed once for each color, and nodes are sorted based not on their degree but on their $x_{ij}$ value in the current solution.

For each initial node, a clique is constructed until the corresponding inequality is broken, and extended to up to $\kappa$ maximal cliques using backtracking, making use of the \textit{candidates} collection (note that in this case, \textit{candidates} is initialized with not only the initial node's neighbours, but also with all the nodes in its partition). In case no clique breaking the inequality is found, the next initial node is picked.

In order to avoid exploring the same solution space multiple times for different initial nodes, restrictions on how many times a node or an edge can be visited are applied. The resulting algorithm is presented in \ref{alg:sep:extclique}.

\begin{algorithm}
\begin{algorithmic}

\FORALL{color $j$ such that $w_j \geq \mu$}
\STATE sort all nodes and adjacency lists descendingly by $x_{ij}$ value
 
\FORALL{initial node $v$ in $V$}
	\STATE initialize \textit{clique} with node $v$
	\STATE initialize \textit{candidates} with $N(v) \cup P(v)$

	\WHILE{\textit{candidates} is not empty}
		\IF{current clique breaks inequality}
			\FORALL{maximal cliques $K$ containing \textit{clique} up to $\kappa$}
				\STATE add extended clique cut using $K$ and color $j$ 
			\ENDFOR
			\CONTINUE with next initial node
		\ELSIF{next candidate $u$ can be used}
			\STATE add $u$ to \textit{clique} and remove it from \textit{candidates} 
			\STATE remove nodes not adjacent to $u$ from \textit{candidates} 
		\ELSE
			\STATE remove $u$ from \textit{candidates}
		\ENDIF
	\ENDWHILE
		
\ENDFOR
\ENDFOR

\caption{Separation algorithm for extended clique cuts}
\label{alg:sep:extclique}

\end{algorithmic}
\end{algorithm} 

\subsection{Component independent set inequalities}

Component hole \ref{ineq:chole} and component path \ref{ineq:cpath} inequalities are separated within the same procedure using a greedy heuristic. In a similar fashion to algorithm \ref{alg:sep:extclique}, for every color the graph is sorted according to $x_{ij}$ values, and for each initial node a component path or hole is greedily constructed. Once again, bounds for a maximum number of visits on each node are imposed, thus rejecting nodes with a certain number of visits or belonging to a partition already in the path, since this would violate the \textit{component} property.

On every iteration the most promising node is added to the path being built. In case this node is adjacent to a node already in the path (thus generating a hole), it is added only if it violates the corresponding inequality, otherwise, the next candidate is picked, and so forth. Algorithm \ref{alg:sep:ciset} resumes this process.

\begin{algorithm}

\begin{algorithmic}

\FORALL{color $j$ such that $w_j \geq \mu$}
\STATE sort all nodes and adjacency lists descendingly by $x_{ij}$ value
 
\FORALL{initial node $v$ in $V$}
	\STATE initialize \textit{path} with node $v$

	\LOOP
		\FORALL{valid node $u$ adjacent to last node in the path}
			\IF{$u$ is adjacent to a previous node $w$ in the path}
				\IF{hole $H=[u,\ldots,w]$ violates inequality \ref{ineq:chole}}
					\STATE add component hole inequality with hole $H$ and color $j$
					\CONTINUE with next initial node
				\ENDIF
			\ELSE
				\STATE add node $u$ to \textit{path}	
				\IF{current \textit{path} breaks inequality \ref{ineq:cpath}}
					\STATE add component path inequality with \textit{path} and color $j$
					\CONTINUE with next initial node
				\ENDIF
			\ENDIF
		\ENDFOR
	\ENDLOOP
\ENDFOR

\ENDFOR

\caption{Separation algorithm for component independent set cuts}
\label{alg:sep:ciset}

\end{algorithmic}
\end{algorithm}

As an alternative to the previous algorithm, we also implemented the hole detection algorithm presented in \cite{nikolopoulos2004hole}. We adapted the algorithm presented in the paper to reject a node if it belongs to a partition already in the path, therefore exploring only component holes; testing this implementation against the previous one showed that our custom algorithm performed slightly better, and generated path as well as hole inequalities.

\subsection{Partition graph independent set inequalities}

For both path and hole inequalities (\ref{ineq:gpiset}) over the partition graph $G'$, algorithms equivalent to the ones used for component independent sets are applied as separation heuristics.

Graph $G'$ is constructed once at the beginning of the branch and cut and is then used as input for these heuristics. Since there are no $x_{ij}$ variables to use for sorting the nodes of the graph, the value $\sum_{i \in P_k} x_{ij}$ is used for each partition $P_k$, this is, the sum of the values of the partition's nodes.

\subsection{Block color inequalities}

Block color inequalities (\ref{ineq:blockcp}) are explored using brute force, since there are no more than $c.q$ of them, and checking whether they are violated or not can be performed fast enough.

Alternatively, they can be added initially to the cut pool provided by the branch and cut framework, instead of manually checking them at each cuts iteration.

\section{Node selection strategies}

On each iteration of the branch and cut algorithm, an unprocessed node must be chosen to be solved. Determining which node will be handled next is known as the \textit{node selection} strategy.

There are different standard node selection strategies, all of them implemented by the branch-and-cut framework we used. In chapter \ref{sec:results} we experiment with the behaviour of the algorithm under different strategies:
\begin{itemize}
	\defitem{DFS}{Depth-first search, picks the last node opened, attempting to generate an integral solution by diving to the bottom of the tree as fast as possible. This strategy has low memory consumption as relatively few nodes are left unprocessed on every iteration.}
	\defitem{BFS}{Breadth-first search, picks the first node opened; this strategy solves every open node at the same depth before proceeding to the following depth level. It usually has memory issues, as a large number of nodes tend to be left open, thus making this a poor choice in most cases.}
	\defitem{BestBound}{Best-bound, picks the open node with best objective function available, which tends to be near the root of the branch and bound tree. This strategy usually reports the best results.}
	\defitem{BestEstimate}{Best-estimate uses \textit{an estimate of a given node's progress toward integer feasibility relative to its degradation of the objective function}\cite{cplex121}; it improves the chance of finding feasible solutions when they are difficult to generate.}
\end{itemize}


\section{Branching strategies}
\label{subsec:alg:branching}

After each node in the branch and cut tree is processed, new child nodes are created by subdividing the problem into two easier subproblems; this is usually done by \textit{branching} on a certain variable. Usually, in the case of binary variables, a variable $x$ with a fractional value in the relaxation is chosen, and the two subproblems are created by fixing $x = 0$ and $x = 1$ and re-processing. Alternatively, bounds on expressions, instead of on variables, can be set.

\begin{figure}[h]
	\label{fig:branching}
	\centering
	\begin{tikzpicture} 
		[black/.style= {minimum size=3mm,thick,circle,draw=black,fill=black},
		 branch/.style= {->}] 
		
		\node[black] (n1) at ( 0,0) {};
		\node[black] (n2) at ( 2,0) {};
		
		\node[black] (n3) at ( 1,2) {}
			edge[branch] node[swap,auto] {$x_{ij} = 0$} (n1)
			edge[branch] node[auto] {$x_{ij} = 1$} (n2)	
		;
		
	\end{tikzpicture} 
\end{figure}

Whichever method of branching is chosen, it requires that the solutions represented by the union of all subproblems cover all integral solutions represented by the parent node; otherwise, feasible solutions may be left unexplored.

Choosing which variable to branch on, and what bounds are implied within each branch, is part of what is called the branching strategy. 

\subsection{Static priorities}

The simplest way to choose which variable to branch on is to assign a priority to each $x_{ij}$, which will be used to pick the branching variable when necessary: the integer infeasible variable with the highest priority is picked on every branching. 

We experimented in section \ref{subsec:resultsbranching} with different criteria for selecting variable $x_{i^*j^*}$. We first pick the node $v_{i^*}$ based on either the number of partitions adjacent to it or the size of the partition it belongs to, or a combination of both. Once the node is chosen, we choose from the set of variables $x_{i^*j_0} \ldots x_{i^*j_C}$ the one with the lowest $j$ such that its value in the relaxation is greater than zero (as we choose only integer infeasible variables for branching).

Using this criteria has the huge drawback that no information regarding the actual value of the $x_{ij}$ variable is used, so these priorities work best as a tie-breaker for another strategy.

\subsection{Fractional values}
\label{subsubsec:alg:branch:frac}

A common practice is to pick the most fractional variable to branch on. We determine such variable as:
\[
\min_{x_{ij}} \{ |x_{ij} - 0.5| \}
\]

In case of a tie, we use the static priority set for the variables to determine which one use to branch on. We also experimented in section \ref{subsec:resultsbranching} with the opposite criteria, this is, branching on the less fractional value (excluding those variables with already integral values).

This is a common branching technique, but does not exploit any particular feature of the problem being studied, unlike the one described below.

\subsection{Degree of saturation}
\label{subsubsec:alg:branch:dsatur}

A branching strategy specifically related to the partitioned coloring problem is to branch on a node with the highest degree of saturation. Since these nodes are usually the most difficult ones to handle, it is reasonable to fix their values as early as possible in the branch and cut tree.

This criteria for picking the branching variable requires first to compute an approximate degree of saturation for every node and choosing the one with the largest value, $i^*$, in order to obtain a set of candidate variables $x_{i^*j_0}, x_{i^*j_1}, \ldots, x_{i^*j_c}$ (once again, ties between nodes are broken using the already defined priorities).

Since the only available values are those of the fractional solution, we color one node in each partition using the largest value within the partition and neighbours: this is, for every node $i$ and color $j$ combination, if the value $x_{ij}$ is larger than all of its neighbours and nodes in the same partition, as well as larger than an arbitrary lower bound ($0.7$)\footnote{Note that if the classic constraints are being used, \ref{eqn:partsum} and \ref{eqn:adjscolorp}, by specifying a lower bound higher than $0.5$ it is not necessary to verify that the node has the highest value among its neighbours or within the partition, as it will be assured by the restrictions; the check is required in case alternative restrictions are used, such as allowing more than one node to be colored or grouping multiple color conflicts into single constraints.}, we assign color $j$ to node $i$. Note that some partitions might be left uncolored, in this case they will not contribute to the degree of saturation of their neighbours.

\begin{equation}
\label{eqn:fixcriteria}
v_i \leftarrow j \text{ if } x_{ij} > 0.7 \wedge x_{ij} > x_{kj}\ \forall k \in N(i) \cup P(i)
\end{equation}

Having chosen a node $v_i$ for branching, we must choose which variable $x_{i^*j^*}$ from the set of candidate $x_{i^*j_0}, x_{i^*j_1}, \ldots, x_{i^*j_c}$ will be branched on. We have implemented two different strategies for this:
\begin{itemize}
	\defitem{\textsc{DSatur-2}}{Choose the variable $x_{i^*j^*}$ with the highest value from the set, and branch on $x_{i^*j^*} = 0$ and $x_{i^*j^*} = 1$; this results in a classic $0-1$ branching on the variable corresponding to the most saturated node with its most \textit{likely} color.}
	\defitem{\textsc{DSatur-(C+1)}}{Create up to $C+1$ subproblems, one for each possible coloring of the node, branching on $x_{i^*j_0} = 1, \ldots, x_{i^*j_c} = 1$, plus another child which sets all $x_{i^*j_0}$ variables to zero, in case the node is not colored within its partition; this idea was first defined in \cite{brelaz1979new}, revisited in \cite{sewell1996improved}, and used in \cite{mendez2006branch}.}
\end{itemize}

\subsection{Implied bounds}
\label{subsubsec:alg:branch:bounds}

When manually specifying the branching variable and creating the subproblems, it is also possible to fix more variables that would be affected by the value assigned to the first one.

Regardless of the branching variable, it is possible to fix all color variables $w_j$ to $1$ for $j = 1,\ldots,\ceil{z}$, where $z$ is the value of the objective function of the current node's relaxation. For example, if the sum of all $w_j$ variables is $5.3$, which is a lower bound on the chromatic number, we can be sure that at least $6$ different colors are needed to color the graph, and therefore all $w_1,\ldots,w_6$ can be set to $1$.

When branching down on the selected variable\footnote{Fixing the branch variable's value to 0.}, there are no more logical implications than the previous one that can be used to bound more variables. This is easy to see since setting an $x_{ij}$ variable to zero implies that a certain color will not be used for a certain node, but does not grant any information on which \textit{node} on the partition will be colored and with which \textit{color}.

Branching up, on the other hand, provides much more information. Whenever a variable $x_{i^*j^*}$ is set to $1$, this is, node $i^*$ in partition $P(i^*)$ is assigned color $j^*$, we may specify the following conditions for that branch:

\begin{itemize}
\item Every other color-node combination in partition $P(i^*)$ can be set to zero, as only one node must be assigned a color in the partition.
\[
x_{ij} = 0 \quad \forall i \in P(i^*),\ \forall j \in C,\ i \neq i^* \vee j \neq j^*
\]

\item Every node adjacent to $i^*$ cannot use color $j^*$ in order to avoid color conflicts.
\[
x_{ij^*} = 0 \quad \forall i \in N(i^*)
\]
\end{itemize}

\section{Primal heuristic}
\label{subsec:alg:primal}

The algorithm used to create an integer feasible solution from the relaxation's solution is called the \textit{primal heuristic}. A typical primal heuristic consists in rounding the values of every fractional variable to the nearest integer value, as long as this process satisfies all the restrictions imposed by the model.

For \PCP{} we implemented a primal heuristic based on the \textsc{dsatur} algorithm. Given a fractional solution $x^*$, for every variable $x_{ij}$ with a large enough value, we fix that node-color combination. The criteria used for determining when a variable is fixed is the same as the one depicted in \ref{eqn:fixcriteria}.

Also, for every variable $x_{ij}$ with an upper bound set to $0$ as a product of the branching in the branch and cut tree, we forbid that node-color combination.

Having all these values fixed, an extremely short run of \textsc{dsatur} is executed, bounded to 200 milliseconds. The algorithm works reasonably fast as more and more variables are fixed, and bounds for the optimal coloring can be inferred from the branch and cut tree, further shortening the exploration of possible solutions. 
\begin{itemize}
\item{Value $\ceil{\sum_{j \in C} w_j}$ of the node's relaxation is a lower bound to the integer solution, so in case \textsc{dsatur} finds a solution using that number of colors, it can be assured that it is the local optimum.}
\item{The solution of the primal heuristic will be used as the global upper bound, replacing the current incumbent solution, only if it uses less colors. Therefore, \textsc{dsatur} is bounded to exploring solutions that use strictly less colors than the incumbent.}
\end{itemize}

The best coloring obtained by the algorithm is then used as an incumbent solution for the node. In case certain symmetry breaking restrictions are in place, a reordering of the labels assigned to each color class might have to be performed.

\section{Implicit enumeration}
\label{subsec:alg:implicit}

Early experimentation with the branch and cut algorithm and with the \textsc{dsatur} algorithm has shown that, for relatively small instances, the latter explores all possible solutions much faster than the former, since it does not have all the overhead imposed by the different artifacts present in a full branch and cut.

Therefore, when we have reduced the problem size to a relatively small one by fixing node-color assignments during the branching process, instead of proceeding with the traditional branch and cut algorithm, we execute a full run of \textsc{dsatur}. Since most partitions are already colored, the number of possible solutions is reasonably small to be fully explored.

In chapter \ref{sec:results} we experiment with different values for the number of uncolored partitions in the branch and cut tree to be used as the threshold for stopping the branch and cut and starting an unbounded execution of \textsc{dsatur}.

\section{Implementation details}

The branch and cut algorithm was implemented in Java 1.6 using CPLEX version 12.1 both as a branch-and-cut framework and a linear programming solver for relaxations. 

We made use of branch, heuristic and cut callbacks provided by the CPLEX API to manage the branching strategy, inject primal solutions and apply custom cuts on both the root and internal nodes.
\begin{itemize}
\item{The branch callback is invoked once the processing of a certain node has been completed in order to determine how to create the child subproblems; inside this callback we implement the different dynamic branching strategies described in \ref{subsec:alg:branching}. Static priorities are fixed during the initialization of the problem. This callback is also used to prune the branch and cut tree once a certain number of partitions have been fixed in order to proceed with the implicit enumeration from \ref{subsec:alg:implicit}.}
\item{The heuristic callback is invoked after the linear relaxation of a node has been solved, and provides the fractional values from the relaxation's solution to derive an integral feasible solution, using the color degree strategy explained in \ref{subsec:alg:primal}. We make use of this callback to inject the integral solution derived from implicit enumeration (\ref{subsec:alg:implicit}).}
\item{The cut callback is invoked after the linear relaxation is solved; every certain number of nodes the separation heuristics are invoked in an attempt to add planes to cut off the current linear solution. After the cuts are added, the relaxation is solved again, and more iterations of cutting planes may be optionally executed; while few iterations are performed in the internal nodes, a larger number is executed in the root.}
\end{itemize} 

The framework was configured to use standard branch and cut search, instead of dynamic search, to correctly determine the performance of the developed strategies. Multi-core processing was also disabled.
%!TEX root = pcp.tex

\section{Computational results}

\subsection{Partitioned \textsc{dsatur}}
\label{subsec:resultspdsatur}
	%!TEX root = pcp.tex

\subsection{Models comparison}

We executed a test suite for determining which inequalities to use in the formulation of the problem. Recall from section \ref{sec:model} that there are several restrictions that can be applied to define the model, as well as additional ones that may strengthen the model or reduce the number of symmetrical solutions.

In order to test the effectiveness of the different formulations, we applied a fixed number of cutting planes iterations, using all implemented cuts with a slightly aggressive configuration, and reported the resulting MIP gap and running time (in seconds), as well as how many rounds of cutting planes were executed. It is worth noting that in some cases fewer iterations than total were applied as the separation heuristics were not able to find any more violated inequalities.

For these tests we used binomial graphs with a fixed size of 100 nodes with exactly 2 nodes per partition, and powerlaw cluster graphs with the same size, changing only the density of the graphs. All graphs were preprocessed beforehand.

\subsubsection{Adjacency constraints}

We first tested the four different adjacency (or color conflict) constraints we had proposed, using arbitrarily chosen constraints \ref{eqn:partsum}, \ref{eqn:lowerlabel} and \ref{eqn:wjleqsumcolor} to complete the model:

\begin{align*}
&\sum_{i \in P_k} \sum_{j \in C} x_{ij} = 1 &\quad \forall P_k \in P \tag{\ref{eqn:partsum}} \\
&w_j \geq w_{j+1} &\quad \forall 1 \leq j < c \tag{\ref{eqn:lowerlabel}} \\
&w_j \leq \sum_{i \in V} x_{ij} &\quad \forall j \in C \tag{\ref{eqn:wjleqsumcolor}} \\
\end{align*}

The different adjacency constraints being tested in this experiment are the following: 

\begin{align*}
&x_{ij} + x_{kj} \leq w_j \sumheight &\quad \forall (i,k) \in E, \; \forall j \in C \tag{\ref{eqn:adjscolorp}} \\
&x_{ij} + x_{kj} \leq 1 \sumheight &\quad \forall (i,k) \in E, \; \forall j \in C \tag{\ref{eqn:adjscolorpone}} \\
&\sum_{i \in N(i_0)} x_{i_0j} + r * x_{i_0j} \leq r * w_j &\quad \forall j \in C, \; \forall i_0 \in V \tag{\ref{eqn:adjsneighb}} \\
&\sum_{i \in P_k \cap N(i_0)} x_{ij} + x_{i_0j} \leq w_j &\quad \forall j \in C, \; \forall P_k \in P, \; \forall i_0 \in V \tag{\ref{eqn:adjsperpart}} 
\end{align*}

Results are displayed on table \ref{table:modelsadj}. Differences between gaps are almost non existent, whereas time required changes greatly between graphs with different densities. On higher density graphs, constraints \ref{eqn:adjsneighb} using a clique coverage of the neighbourhood report a better running time than the others; while on lower density \ref{eqn:adjscolorpone} works better than \ref{eqn:adjscolorp}, even though the former uses $n \times c$ additional constraints \ref{eqn:nodelessthanwj}.

Graphics \ref{fig:models:adj} \TODO{Explain these graphics}

\begin{sidewaystable}
\centering

	\begin{tabular}{|c|ccc|ccc|ccc|ccc|}
	\hline
	\multicolumn{1}{|c|}{Id} & \multicolumn{3}{|c|}{Constraint \ref{eqn:adjscolorp}} & \multicolumn{3}{|c|}{Constraint \ref{eqn:adjscolorpone}} & \multicolumn{3}{|c|}{Constraint \ref{eqn:adjsneighb}} & \multicolumn{3}{|c|}{Constraint \ref{eqn:adjsperpart}} 
	\\
	& gap & rounds & time & gap & rounds & time & gap & rounds & time & gap & rounds & time 
	\\
	\hline
	EW 20\% & 0.458 & 14.6 & 5.632 & 0.458 & 10.6 & \b{4.568} & 0.458 & 20.4 & 7.915 & 0.458 & 16.2 & 5.728
	\\
	EW 40\% & 0.466 & 22.4 & 16.998 & 0.466 & 25.6 & \b{14.09} & 0.466 & 32.6 & 17.884 & 0.466 & 24.8 & 16.976
	\\
	EW 60\% & 0.42 & 62.8 & 98.642 & 0.42 & 57.4 & \b{77.694} & 0.42 & 72.6 & 87.575 & 0.42 & 78.6 & 120.138
	\\
	EW 80\% & 0.292 & 193.0 & 449.054 & 0.296 & 181.2 & 469.844 & 0.292 & 192.4 & \b{349.557} & 0.294 & 160.0 & 451.126
	\\
	\hline
	HK 10\% &  0.2 &  0.8 & 0.156 &  0.2 &  0.8 & 0.128 &  0.2 &  0.8 & 0.106 &  0.2 &  0.8 & 0.168
	\\
	HK 20\% & 0.12 &  0.6 & 0.278 & 0.12 &  0.6 & 0.324 &  0.2 &  1.0 & 0.181 & 0.12 &  0.6 & 0.306
	\\
	HK 30\% &  0.0 &  0.8 & 0.308 &  0.0 &  0.8 & 0.276 &  0.0 &  3.2 & 0.489 &  0.0 &  0.8 & 0.318
	\\
	HK 40\% & 0.048 &  2.2 & 0.28 & 0.048 &  3.0 & 0.312 & 0.048 &  5.2 & 0.416 & 0.048 &  2.6 & 0.292
	\\
	\hline 
	 \end{tabular}
	
	\caption{Comparison of different color conflict constraints on the model formulation: adjacent nodes sum bounded by $w_j$ (\ref{eqn:adjscolorp}), adjacent nodes sum bounded by $1$ (\ref{eqn:adjscolorpone}), adjacencies grouped by partition (\ref{eqn:adjsperpart}) and using clique coverage of the neighbourhood (\ref{eqn:adjsneighb}).}
	\label{table:modelsadj}
\end{sidewaystable}


\begin{figure}
\centering
\subfloat[EW 100 Nodes 20\% Density]{\includegraphics[width=0.6\textwidth]{plots/modelsgap-ew20-n100-adj.png}}
\subfloat[EW 100 Nodes 40\% Density]{\includegraphics[width=0.6\textwidth]{plots/modelsgap-ew40-n100-adj.png}}
\\
\subfloat[EW 100 Nodes 60\% Density]{\includegraphics[width=0.6\textwidth]{plots/modelsgap-ew60-n100-adj.png}}
\subfloat[EW 100 Nodes 80\% Density]{\includegraphics[width=0.6\textwidth]{plots/modelsgap-ew80-n100-adj.png}}
\caption{Comparison of the inclusion of different color conflict constraints in the model, visualizing evolution of the gap during time in a cutting planes algorithm. Compared constraints are: adjacent nodes sum bounded by $w_j$ (\ref{eqn:adjscolorp}), adjacent nodes sum bounded by $1$ (\ref{eqn:adjscolorpone}), adjacencies grouped by partition (\ref{eqn:adjsperpart}) and using clique coverage of the neighbourhood (\ref{eqn:adjsneighb}).}
\label{fig:models:adj}
\end{figure}

\subsubsection{Colored nodes per partition}

A quick test we also ran in parallel was to determine whether to paint exactly one node per partition (\ref{eqn:partsum}), or to relax this constraint and allow for painting more than a single node (\ref{eqn:partsumgeq}). 

Results on table \ref{table:models:partsum} confirm our expectations: while the former has a slightly larger running time, it also reports a slightly lower gap than the latter in some cases. The simplicity provided by \ref{eqn:partsum} when extracting solutions from the model (when constructing the the partial solutions to be processed during the primal heuristic, or during the branching process) makes us choose this option in our formulation.

\begin{table}
\label{table:models:partsum}
\centering

\begin{tabular}{|c|cc|cc|}
\hline
\multicolumn{1}{|c|}{Id} & \multicolumn{2}{|c|}{At least 1} & \multicolumn{2}{|c|}{Exactly 1}
\\
 & gap & time & gap & time
\\
\hline
EW 20\% & 0.46 & 5.472 & 0.458 & 7.915
\\
EW 40\% & 0.466 & 17.324 & 0.466 & 17.884
\\
EW 60\% & 0.42 & 93.578 & 0.42 & 87.575
\\
EW 80\% & 0.294 & 354.612 & 0.292 & 349.557
\\
\hline
HK 10\% &  0.2 & 0.112 &  0.2 & 0.106
\\
HK 20\% &  0.2 & 0.136 &  0.2 & 0.181
\\
HK 30\% &  0.0 & 0.434 &  0.0 & 0.489
\\
HK 40\% & 0.076 & 0.398 & 0.048 & 0.416
\\
\hline 
 \end{tabular}

\caption{Comparison of constraints specifying whether exactly one node must be assigned one color in the partition, or at least one node should be painted with at least one color.}

\end{table}

\subsubsection{Model strengthening}

We also compared applying only constraint \ref{eqn:wjleqsumcolor}, which ensures that variable $w_j$ is set only if a node uses color $j$ (regardless of the objective function), to adding restrictions \ref{eqn:wjgeqsumnode} and \ref{eqn:wjgeqsumpart}:

\begin{align*}
\sum_{j \in C} w_j \geq \sum_{j \in C} j x_{ij} \quad \forall i \in V \tag{\ref{eqn:wjgeqsumnode}} \\
\sum_{j \in C} w_j \geq \sum_{j \in C} \sum_{i \in P_k} j x_{ij} \quad \forall P_k \in P \tag{\ref{eqn:wjgeqsumpart}}
\end{align*}

Results on table \ref{table:models:colorbound} show that there is very little difference between the three variants. Overall, the simplest one, \ref{eqn:wjleqsumcolor}, seems to be the fastest one to execute, although taking slightly more cuts iterations in non-medium density graphs. The graphics \ref{fig:models:colorbound} \TODO{Explain them}.

\begin{table}
\label{table:models:colorbound}
\centering

\begin{tabular}{|c|ccc|ccc|ccc|}
\hline
\multicolumn{1}{|c|}{Id} & \multicolumn{3}{|c|}{\ref{eqn:wjleqsumcolor}} & \multicolumn{3}{|c|}{\ref{eqn:wjgeqsumnode}} & \multicolumn{3}{|c|}{\ref{eqn:wjgeqsumpart}}
\\
 & gap & niters & time & gap & niters & time & gap & niters & time
\\
\hline
EW 20\% & 0.458 & 20.4 & 7.915 & 0.458 & 15.8 & 7.318 & 0.458 & 15.4 & \b{7.286}
\\
EW 40\% & 0.466 & 32.6 & 17.884 & 0.466 & 37.6 & 19.464 & 0.466 & 32.0 & \b{17.748}
\\
EW 60\% & 0.42 & 72.6 & \b{87.575} & 0.42 & 84.2 & 91.272 & 0.42 & 79.4 & 89.03
\\
EW 80\% & 0.292 & 192.4 & \b{349.557} & 0.294 & 170.0 & 355.518 & 0.294 & 180.6 & 378.936
\\
\hline
HK 10\% &  0.2 &  0.8 & \b{0.106} &  0.2 &  0.8 & 0.108 &  0.2 &  0.8 & 0.112
\\
HK 20\% &  0.2 &  1.0 & \b{0.181} &  0.2 &  1.0 & 0.198 &  0.2 &  1.0 & 0.184
\\
HK 30\% &  0.0 &  3.2 & \b{0.489} &  0.0 &  3.2 & 0.552 &  0.0 &  3.2 & 0.51
\\
HK 40\% & 0.048 &  5.2 & \b{0.416} & 0.048 &  4.0 & 0.398 & 0.048 &  5.2 & 0.438
\\
\hline 
 \end{tabular}

\caption{Comparison of different model strengthening constraints: (\ref{eqn:wjleqsumcolor}) which ensures that variable $w_j$ is set only if a node uses color $j$, and (\ref{eqn:wjgeqsumnode}) and (\ref{eqn:wjgeqsumpart}) which eliminate certain fractional constraints, adding over all colors of node and of a partition, respectively.}

\end{table}

\begin{figure}
\centering
\subfloat[EW 100 Nodes 20\% Density]{\includegraphics[width=0.6\textwidth]{plots/modelsgap-ew20-n100-cb.png}}
\subfloat[EW 100 Nodes 40\% Density]{\includegraphics[width=0.6\textwidth]{plots/modelsgap-ew40-n100-cb.png}}
\\
\subfloat[EW 100 Nodes 60\% Density]{\includegraphics[width=0.6\textwidth]{plots/modelsgap-ew60-n100-cb.png}}
\subfloat[EW 100 Nodes 80\% Density]{\includegraphics[width=0.6\textwidth]{plots/modelsgap-ew80-n100-cb.png}}
\caption{Comparison of the inclusion of model strengthening constraints in the model, visualizing evolution of the gap during time in a cutting planes algorithm. Compared constraints are: (\ref{eqn:wjleqsumcolor}) which ensures that variable $w_j$ is set only if a node uses color $j$, and (\ref{eqn:wjgeqsumnode}) and (\ref{eqn:wjgeqsumpart}) which eliminate certain fractional constraints, adding over all colors of node and of a partition, respectively.}
\label{fig:models:colorbound}
\end{figure}

\subsubsection{Symmetry breaking}

Results obtained from comparing no symmetry breaking constraints whatsoever with color label (\ref{eqn:lowerlabel}), node count (\ref{eqn:symnodecount}) and minimum node label (\ref{eqn:nodeszero},\ref{eqn:minlabel}) ordering restrictions are shown on table \ref{table:models:sym}. The evolution of the obtained gap in time for different densities is shown in figure \ref{fig:models:sym}. \TODO{Analyze those graphs}

\begin{align*}
& w_j \geq w_{j+1} \sumheight \quad &\forall 1 \leq j < c \tag{\ref{eqn:lowerlabel}} \\
& \sum_{i \in V} x_{ij} \geq \sum_{i \in V} x_{ij+1} \quad &\forall 1 \leq j < c \tag{\ref{eqn:symnodecount}} \\
& x_{ij} = 0 \sumheight \quad &\forall j > p(i) + 1 \tag{\ref{eqn:nodeszero}} \\
& x_{ij} \leq \sum_{l = j-1}^{k-1} \sum_{u \in P_l} x_{uj-1} \quad &\forall 1 < k \leq q, \; \forall i \in P_k, \; \forall 1 < j \leq k \tag{\ref{eqn:minlabel}}
\end{align*}

It is with these constraints that significative changes in solution gaps are found. While there is little difference between applying or not the simplest restrictions \ref{eqn:lowerlabel} (although they are required for the validity of other inequalities and bounds), stricter restrictions that further eliminate symmetrical solutions report much lower gaps, in some cases even reaching optimality at this stage. 

Minimum partition index constraints (\ref{eqn:nodeszero},\ref{eqn:minlabel}) have the best gaps, require fewer cutting planes iterations, and run within acceptable times (in some cases even faster than its counterparts).

\begin{sidewaystable}
\label{table:models:sym}
\centering

\begin{tabular}{|c|ccc|ccc|ccc|ccc|}
\hline
\multicolumn{1}{|c|}{Id} & \multicolumn{3}{|c|}{Constraint \ref{eqn:minlabel}} & \multicolumn{3}{|c|}{None} & \multicolumn{3}{|c|}{Constraint \ref{eqn:lowerlabel}} & \multicolumn{3}{|c|}{Constraint \ref{eqn:symnodecount}}
\\
 & gap & niters & time & gap & niters & time & gap & niters & time & gap & niters & time
\\
\hline
EW 20\% & 0.46 & 20.8 & 3.086 & \b{0.458} & 18.2 & 8.45 & \b{0.458} & 20.4 & 7.915 & 0.466 & 16.0 & 6.032
\\
EW 40\% & \b{0.298} & 34.4 & 20.342 & 0.466 & 31.6 & 21.244 & 0.466 & 32.6 & 17.884 & 0.314 & 39.8 & 31.484
\\
EW 60\% & \b{0.04} & 34.8 & 153.702 & 0.42 & 73.2 & 96.82 & 0.42 & 72.6 & 87.575 & 0.16 & 86.4 & 433.19
\\
EW 80\% & \b{0.052} & 43.2 & 299.66 & 0.294 & 192.2 & 401.3 & 0.292 & 192.4 & 349.557 & 0.16 & 100.6 & 202.282
\\
\hline
HK 10\% &  \b{0.2} &  0.6 & 0.082 & 0.206 &  4.4 & 0.26 &  \b{0.2} &  0.8 & 0.106 & 0.15 &  0.6 & 0.084
\\
HK 20\% & \b{0.12} &  0.6 & 0.144 & 0.206 &  9.8 & 0.752 &  0.2 &  1.0 & 0.181 & \b{0.12} &  0.6 & 0.12
\\
HK 30\% &  0.0 &  2.0 & 0.314 &  0.0 &  3.4 & 0.516 &  0.0 &  3.2 & 0.489 &  0.0 &  2.4 & 0.454
\\
HK 40\% & 0.048 &  3.8 & 0.316 & 0.05 &  6.2 & 0.494 & 0.048 &  5.2 & 0.416 & \b{0.024} &  4.4 & 0.376
\\
\hline 
 \end{tabular}

\caption{Comparison of the inclusion of different symmetry breaking constraints in the model: assigning the lowest color label to the color class with the lowest node index \eqref{eqn:minlabel}, applying no constraint whatsoever, forcing lower labels to be used first \eqref{eqn:lowerlabel} and assigning the lowest color label to the color class with the greatest number of nodes \eqref{eqn:symnodecount}.}

\end{sidewaystable}

\begin{figure}
\centering
\subfloat[EW 100 Nodes 20\% Density]{\includegraphics[width=0.6\textwidth]{plots/modelsgap-ew20-n100-sym.png}}
\subfloat[EW 100 Nodes 40\% Density]{\includegraphics[width=0.6\textwidth]{plots/modelsgap-ew40-n100-sym.png}}
\\
\subfloat[EW 100 Nodes 60\% Density]{\includegraphics[width=0.6\textwidth]{plots/modelsgap-ew60-n100-sym.png}}
\subfloat[EW 100 Nodes 80\% Density]{\includegraphics[width=0.6\textwidth]{plots/modelsgap-ew80-n100-sym.png}}
\caption{Comparison of the inclusion of different symmetry breaking constraints in the model, visualizing evolution of the gap during time in a cutting planes algorithm. Compared constraints are: assigning the lowest color label to the color class with the lowest node index \eqref{eqn:minlabel}, applying no constraint whatsoever, forcing lower labels to be used first \eqref{eqn:lowerlabel} and assigning the lowest color label to the color class with the greatest number of nodes \eqref{eqn:symnodecount}.}
\label{fig:models:sym}
\end{figure}


\subsubsection{Chosen formulation from cutting planes}
\label{subsubsec:results:model:chosen}

Taking into account all previous results in a cutting planes algorithm, the set of constraints that we will use in the \PCP{} formulation for subsequent computational experiments will be the following:

\begin{align}
\sum_{i \in P_k} \sum_{j \in C} x_{ij} = 1 \quad &\forall P_k \in P \tag{\ref{eqn:partsum}} \\
 \sum_{i \in N(i_0)} x_{i_0j} + r * x_{i_0j} \leq r * w_j \quad &\forall j \in C, \; \forall i_0 \in V \tag{\ref{eqn:adjsneighb}} \\
 w_j \leq \sum_{i \in V} x_{ij} \quad &\forall j \in C \tag{\ref{eqn:wjleqsumcolor}} \\
 x_{ij} \leq \sum_{l = j-1}^{k-1} \sum_{u \in P_l} x_{uj-1} \quad &\forall 1 < k \leq q, \; \forall i \in P_k, \; \forall 1 < j \leq k \tag{\ref{eqn:minlabel}} \\
 %w_j \geq w_{j+1} \sumheight \quad &\forall 1 \leq j < c \tag{\ref{eqn:lowerlabel}} \\
  x_{ij}, w_{j} \in \{0,1\} \quad &\forall i \in V, \; \forall j \in C \sumheight \nonumber
\end{align}

First two constraints define the problem itself, by specifying that a node must be colored in each partition and no color conflicts must occur; constraints \ref{eqn:wjleqsumcolor} simply strengthen the linear relaxation; and \ref{eqn:minlabel} eliminate symmetrical solutions. Last set of constraints are the binary restrictions.

Note that while adjacency restrictions \ref{eqn:adjscolorpone} reported better results than the chosen ones (\ref{eqn:adjsneighb}) in most cases, the latter worked better in dense graphs, which are the ones that, due to a larger problem size, take longer to solve their linear relaxation. Therefore, we opt for improving the resolution of the hardest graphs instead of getting slightly better results in the rest. 

\subsubsection{Branch and bound testing}

While the previous formulation was chosen for working on a cutting planes algorithm, we are also interested in the behaviour of different models in standard branch and bound algorithms. 

We tested many variations to the chosen model in a branch and bound algorithm, bound to $1800$ seconds, with graphs with $90$ nodes, partition size $2$ and different densities. The branch and bound uses default \textsc{cplex} settings, no custom callbacks were yet applied.

We present in table \ref{table:models:bnb} the following configurations, chosen based on their results:

\begin{itemize}
\defitem{M4}{Chosen model from cutting planes experimentation phase.}
\defitem{M1}{Relaxes that exactly one node must be painted per partition (\ref{eqn:partsum}) by replacing it with at least one painted per partition (\ref{eqn:partsumgeq}).}
\defitem{M8}{Strengthens the model using not only \ref{eqn:wjleqsumcolor} restrictions but also applying \ref{eqn:wjgeqsumpart}.}
\defitem{M3}{Uses simple color conflict constraints, requiring that two adjacent nodes cannot use the same color (\ref{eqn:adjscolorpone}).}
\defitem{M10}{Bases symmetry breaking on the number of nodes of each color class (\ref{eqn:symnodecount}).}
\end{itemize}

\begin{table}
\label{table:models:bnb}
\centering

\begin{tabular}{|c|c|c|c|c|}
\hline
\multicolumn{1}{|c|}{Id} & \multicolumn{1}{|c|}{M4} & \multicolumn{1}{|c|}{M1} & \multicolumn{1}{|c|}{M3}  & \multicolumn{1}{|c|}{M8} 
\\
\hline
EW 20 N=90 & \textbf{0.00} & \textbf{0.00} & 0.25  & \textbf{0.00}  
\\
EW 40 N=90 & 0.33 & \textbf{0.22} & 0.33 & 0.33 
\\
EW 60 N=90 & 0.39 & \textbf{0.37} & 0.41 & \textbf{0.37}
\\
EW 80 N=90 & 0.38 & 0.43 & \textbf{0.31} & 0.39
\\
\hline 
 \end{tabular}

\caption{Gap obtained in a standard branch and bound algorithm for different models.}

\end{table}

Results were most interesting. The formulation chosen for the cutting planes algorithm yielded good results only for lowest density graphs. In other cases, using different models returned better results:
\begin{itemize}
\item In graphs with $40\%$ density, relaxing the \ref{eqn:partsum} constraint on painting one node per partition greatly reduces the obtained gap, as can be seen in the results for $M1$.
\item In the most dense graphs, using tighter model strengthening constraints yields better results, as $M8$ has a much lower gap than the chosen model.
\item In the highest density graphs, using simple color conflict constraints returned a better result.
\end{itemize}

Although we will be using the previous formulation for fixing different settings throughout the following experimentations, we will reuse the results exposed here as alternative models for the final branch and cut algorithm, in order to pick the best performing model for the final version of the algorithm.

\clearpage
	%!TEX root = pcp.tex

\subsection{Partitioned \textsc{dsatur}}
\label{subsec:resultspdsatur}

Considering we had three different partitioned \textsc{dsatur} implementations (see section \ref{subsec:heur:pdsatur}), we ran quick tests on multiple graphs to determine how they performed when executed for short periods of time. For each instance, we executed the different algorithms for one minute, and report the lowest bound for the chromatic number obtained, as well as how fast was this bound obtained. Since we will be using \textsc{dsatur} mostly as an heuristic, it is of our interest that good solutions are found as quickly as possible.

\subsubsection{Hardest partition parametrization}

Before comparing the three different algorithms, we had to fix the criteria used to pick the \textit{hardest partition} on each call in this variant of the algorithm\footnote{Recall that this strategy picked the hardest partition on each call, and then the easiest node from it.}. As we had already mentioned, the criteria used is a combination of:

\begin{itemize}
	\item{Color degree of the partition, defined as the number of different colors adjacent to all of the nodes in the partition; considering that a larger color degree implies a harder partition to color}
	\item{Size of the partition, as a larger partition being colored earlier helps reducing the problem size, therefore the larger the partition the earlier it should be handled}
	\item{Number of uncolored partitions adjacent to the partition, equivalent to the tie breaking criteria used for classic \textsc{dsatur}} 
\end{itemize}

We generated six different configurations, based on all different possible orderings of these items. For example, for the first configuration, we first compared by the number of adjacent uncolored partitions, then by the degree of saturation, and finally by the size of the partition. The following are the configurations we established:

\begin{itemize}
\defitem{C1}{Uncolored, saturation, size}
\defitem{C2}{Saturation, uncolored, size}
\defitem{C3}{Uncolored, size, saturation}
\defitem{C4}{Saturation, size, uncolored}
\defitem{C5}{Size, uncolored, saturation}
\defitem{C6}{Size, saturation, uncolored}
\end{itemize}

Results in table \ref{table:pdsatur:hardp} show little difference for most instances in which partition size is constant (fixed to two nodes). Whereas in some cases, mostly those with lower density, configurations C1, C3 and C5 (those who choose the hardest partition by number of uncolored neighbour partitions before by saturation degree) find the solution earlier, in other cases configurations C2, C4 and C6 report better times. All of them find always nearly the same values for the chromatic number.

Differences arise, however, when we have different partition sizes. Configurations C1 and C3, those who pick the partition on uncolored neighbours, obtain better bounds in less time than the others. Surprisingly, configurations based on partition sizes offer the worst results for these cases.

Since we will be working mostly with partitioned graphs with equal partition sizes, we will not take into consideration configurations that use this as a criteria; and as graphs with higher density have been taking the longest time to process, we will prefer a configuration that best deals with this cases, such as C2.

\begin{sidewaystable}
\label{table:pdsatur:hardp}
\centering

\begin{tabular}{|c|cc|cc|cc|cc|cc|cc|}
\hline
\multicolumn{1}{|c|}{Graphs} & \multicolumn{2}{|c|}{C1} & \multicolumn{2}{|c|}{C2} & \multicolumn{2}{|c|}{C3} & \multicolumn{2}{|c|}{C4} & \multicolumn{2}{|c|}{C5} & \multicolumn{2}{|c|}{C6}
\\
 & chi & found & chi & found & chi & found & chi & found & chi & found & chi & found
\\
\hline
EW 20\% N=140 &  6.0 & 0.192 &  5.8 & 1.514 &  6.0 & 0.196 &  5.8 & 1.504 &  6.0 & 0.184 &  5.8 &  1.5
\\
EW 40\% N=140 & 10.0 & 6.372 &  9.6 & 16.878 & 10.0 & 6.384 &  9.6 & 16.874 & 10.0 & 6.382 &  9.6 & 16.886
\\
EW 60\% N=140 & 14.2 & 27.692 & 14.4 & 16.208 & 14.2 & 27.758 & 14.4 & 16.154 & 14.2 & 27.79 & 14.4 & 16.184
\\
EW 80\% N=140 & 21.8 & 8.29 & 21.8 & 8.138 & 21.8 & 8.314 & 21.8 & 8.14 & 21.8 & 8.294 & 21.8 & 8.148
\\
\hline
EW 50\% N=140 P=(1..2) & 16.2 & 1.414 & 16.0 & 5.232 & 16.2 & 1.414 & 15.6 & 0.428 & 18.0 & 19.652 & 15.6 & 0.612
\\
EW 50\% N=140 P=(1..3) & 10.4 & 6.61 & 11.4 & 11.576 & 10.4 & 6.648 & 12.4 &  9.7 & 13.0 & 1.17 & 12.6 & 0.068
\\
EW 50\% N=140 P=(1..4) &  9.2 & 6.788 &  9.4 & 10.542 &  9.2 & 6.794 & 10.8 & 5.37 & 11.0 & 30.262 & 10.8 & 14.542
\\
EW 50\% N=140 P=(2..3) &  8.8 & 3.366 &  8.8 & 0.486 &  8.8 & 3.35 &  9.8 & 0.37 &  9.6 & 6.964 &  9.8 & 0.382
\\
EW 50\% N=140 P=(2..4) &  7.2 & 0.366 &  7.2 & 16.604 &  7.2 & 0.36 &  8.2 & 20.728 &  8.8 & 0.15 &  8.2 & 25.316
\\
EW 50\% N=140 P=(3..4) &  6.6 & 1.504 &  6.6 & 4.122 &  6.6 & 1.506 &  6.6 & 0.23 &  6.6 & 0.246 &  6.6 & 0.23
\\
\hline
EW 50\% N=080 &  7.0 & 2.248 &  7.0 & 0.658 &  7.0 & 2.252 &  7.0 & 0.648 &  7.0 & 2.25 &  7.0 & 0.656
\\
EW 50\% N=100 &  8.2 & 19.472 &  8.0 & 10.01 &  8.2 & 19.504 &  8.0 & 10.008 &  8.2 & 19.526 &  8.0 & 10.014
\\
EW 50\% N=120 & 10.0 & 12.04 & 10.0 & 1.578 & 10.0 & 12.066 & 10.0 & 1.588 & 10.0 & 12.058 & 10.0 & 1.586
\\
EW 50\% N=140 & 11.6 & 8.88 & 11.8 & 3.832 & 11.6 & 8.892 & 11.8 & 3.81 & 11.6 & 8.904 & 11.8 & 3.826
\\
EW 50\% N=160 & 13.4 & 6.066 & 13.4 & 6.352 & 13.4 & 6.07 & 13.4 & 6.344 & 13.4 & 6.068 & 13.4 & 6.35
\\
EW 50\% N=180 & 14.0 & 1.376 & 14.0 & 6.962 & 14.0 & 1.382 & 14.0 & 6.97 & 14.0 & 1.382 & 14.0 & 6.92
\\
EW 50\% N=200 & 16.2 & 0.57 & 15.4 & 0.766 & 16.2 & 0.574 & 15.4 & 0.776 & 16.2 & 0.572 & 15.4 & 0.774
\\
\hline
HK 10\% N=140 &  6.4 & 0.106 &  6.4 & 0.02 &  6.4 & 0.106 &  6.4 & 0.02 &  6.4 & 0.114 &  6.4 & 0.02
\\
HK 20\% N=140 &  9.4 & 0.114 &  9.0 & 9.804 &  9.4 & 0.112 &  9.0 & 9.824 &  9.4 & 0.11 &  9.0 & 9.846
\\
HK 30\% N=140 & 12.4 & 3.066 & 11.8 & 20.746 & 12.4 & 3.066 & 11.8 & 20.758 & 12.4 & 3.066 & 11.8 & 20.818
\\
HK 40\% N=140 & 14.8 & 21.552 & 14.2 &  6.8 & 14.8 & 21.576 & 14.2 & 6.798 & 14.8 & 21.588 & 14.2 & 6.792
\\
\hline 
 \end{tabular}
\caption{Best value obtained for the chromatic number and time at which this value was obtained in one-minute runs of the \textit{hardest partition} version of \textsc{dsatur}, using different combinations of strategies to pick the hardest partition at each call.}

\end{sidewaystable}

\subsubsection{Strategies comparison}

After fixing the \textit{hardest partition} strategy to C2, we will compare its performance with both the \textit{easiest node} and the \textit{randomized easiest node} variants. We ran the same tests as before, and present the results on table \ref{table:pdsatur:comp}.

Regardless of the configuration chosen for the \textit{hardest partition} variant, both \textit{easiest node} strategies find much better bounds within the one-minute running time. From those, the deterministic one offers slightly better results, and is therefore the algorithm that we will be using for the following experimentations. 

The randomized version might perform better in larger graphs on lengthier running periods, since it has a chance to find a different solution than the deterministic and obtain a sudden improvement on the bound, whereas the deterministic may spend several iterations trying similar assignments. However, in very short runs as these ones, the deterministic version is clearly preferred.  

\begin{table}
\label{table:pdsatur:comp}
\centering

\begin{tabular}{|c|cc|cc|cc|}
\hline
\multicolumn{1}{|c|}{Graphs} & \multicolumn{2}{|c|}{\textit{easiest node}} & \multicolumn{2}{|c|}{\textit{randomized node}} & \multicolumn{2}{|c|}{\textit{hardest partition}}
\\
 & chi & found & chi & found & chi & found
\\
\hline
EW 20\% N=140 &  5.0 & 0.004 &  5.0 & 0.016 &  5.8 & 1.514
\\
EW 40\% N=140 &  8.0 & 4.172 &  8.2 & 27.334 &  9.6 & 16.878
\\
EW 60\% N=140 & 12.4 & 10.062 & 12.4 & 4.656 & 14.4 & 16.208
\\
EW 80\% N=140 & 18.6 & 25.522 & 18.4 & 10.77 & 21.8 & 8.138
\\
\hline
EW 50\% N=140 P=(1..2) & 13.6 & 4.356 & 13.8 & 2.346 & 16.0 & 5.232
\\
EW 50\% N=140 P=(1..3) &  9.6 & 31.634 &  9.8 & 8.31 & 11.4 & 11.576
\\
EW 50\% N=140 P=(1..4) &  8.0 & 6.464 &  8.8 & 9.796 &  9.4 & 10.542
\\
EW 50\% N=140 P=(2..3) &  7.4 & 11.172 &  8.0 & 0.046 &  8.8 & 0.486
\\
EW 50\% N=140 P=(2..4) &  6.8 & 0.702 &  7.0 & 1.224 &  7.2 & 16.604
\\
EW 50\% N=140 P=(3..4) &  5.8 & 1.494 &  6.0 & 0.652 &  6.6 & 4.122
\\
\hline
EW 50\% N=080 &  6.0 & 2.584 &  6.4 & 0.778 &  7.0 & 0.658
\\
EW 50\% N=100 &  7.2 & 1.646 &  7.2 & 15.728 &  8.0 & 10.01
\\
EW 50\% N=120 &  9.0 & 0.042 &  9.0 & 1.618 & 10.0 & 1.578
\\
EW 50\% N=140 & 10.2 & 1.59 & 10.0 & 33.264 & 11.8 & 3.832
\\
EW 50\% N=160 & 11.2 & 5.134 & 11.8 & 2.576 & 13.4 & 6.352
\\
EW 50\% N=180 & 12.6 & 10.242 & 13.0 & 4.818 & 14.0 & 6.962
\\
EW 50\% N=200 & 13.6 & 14.636 & 14.0 & 0.966 & 15.4 & 0.766
\\
\hline
HK 10\% N=140 &  4.0 & 0.016 &  4.0 & 0.044 &  6.4 & 0.02
\\
HK 20\% N=140 &  6.0 & 0.016 &  6.0 & 0.168 &  9.0 & 9.804
\\
HK 30\% N=140 &  8.0 & 0.008 &  8.0 & 0.118 & 11.8 & 20.746
\\
HK 40\% N=140 &  9.8 & 0.002 &  9.8 & 0.008 & 14.2 &  6.8
\\
\hline 
 \end{tabular}

\caption{Best value obtained for the chromatic number and time at which this value was obtained in one-minute runs for the \textit{hardest partition}, \textit{easiest node} and \textit{randomized easiest node} versions of \textsc{dsatur}.}

\end{table}
	%!TEX root = pcp.tex

\section{Branching strategies}
\label{subsec:resultsbranching}

We evaluated the different branching strategies described in section \ref{subsec:alg:branching} on regular graphs with fixed size and different density, in order to determine which reports the best results. We used a simple branch and bound algorithm bounded to 15 minutes of running time.

\subsection{Priorities}

The first test we implemented applied only priorities on the variables during the problem's initialization. Priorities were assigned according to the following formula:
\begin{equation*}
	prio(x_{ij}) = \alpha \delta_P(v_i) + \beta j
\end{equation*}

We tested with different values for $\alpha$ and $\beta$, both positive and negative, to generate different priorities. Although we found hardly any differences in higher density graphs, the ones with the lowest densities ($20\%$) did have significative differences.

In table \ref{table:branch:static} we report those $(\alpha,\beta)$ values which gave results better or near the ones obtained when not using priorities, this is, allowing \textsc{cplex} to choose automatically which variable to branch on.

\begin{table}[h]
\centering

\begin{tabular}{|c|c|c|}
\hline
\textbf{Priorities} & \textbf{Time} & \textbf{Gap} \\
\hline
$\alpha = 10$, $\beta = -1$ &  232.57 & 0.00 \\
$\alpha = 10$, $\beta = 1$ & 523.10 & 0.00 \\
cplex & 570.72 & 0.00 \\
\hline
 \end{tabular}

\caption{Gap and running time for branch and bound executions on $20\%$ density graphs with different priorities on the branching variables.}
\label{table:branch:static}

\end{table}	

Clearly giving the highest priority to nodes with the highest $\delta_P(v_i)$ value, tie-breaking in favor of higher color labels, is the best static branching priority.

\subsection{Dynamic strategies}

Having fixed the priorities to set on the variables, we use them as tie breaking strategies for the two devised strategies which depend on the variable's value (\ref{subsubsec:alg:branch:frac} and \ref{subsubsec:alg:branch:dsatur}). 

We set up a suite of graphs of different size and density to test most and less fractional strategies, as well as both degree of saturation strategies: branching on a particular $x_{ij}$ variable or creating one subproblem for each possible color for a particular node $v_i$. Results are displayed in table \ref{table:branch:dyn}; we report the resulting MIP gap, on which node that gap was obtained, and how many nodes were explored during the 15 minutes of execution.

\begin{sidewaystable}
\centering

\begin{tabular}{|c|ccc|ccc|ccc|ccc|ccc|ccc|}
\hline
\multicolumn{1}{|c|}{Graph} & \multicolumn{3}{|c|}{\textsc{dsatur-2}} & \multicolumn{3}{|c|}{\textsc{dsatur-(C+1)}} & \multicolumn{3}{|c|}{Less fractional} & \multicolumn{3}{|c|}{Most fractional} 
\\
 & gap & found & nodes & gap & found & nodes & gap & found & nodes & gap & found & nodes
\\
\hline
EW 20\% N=90 & \b{0.17} & 77 & 94 & \b{0.17} & \b{76} & 95 & 0.25 & 177 & 250 & 0.25 & 134 & 178 
\\
EW 40\% N=100 & \b{0.33} & 20 & 24 & \b{0.33} & \b{14} & 24 & 0.39 & 27 & 39 & 0.33 & 30 & 44 
\\
EW 60\% N=80 & 0.37 & \b{7} & 18 & 0.37 & 18 & 19 & 0.37 & 30 & 32 & 0.37 & 23 & 27 
\\
EW 80\% N=100 & 0.42 & 2 & 2 & 0.42 & \b{1} & 3 & 0.44 & 3 & 4 & 0.42 & 4 & 4
\\
\hline 
 \end{tabular}

\caption{Results for fractional and degree of saturation (spanning either $2$ or $C+1$ subproblems) branching strategies on branch and bound schemes. Data reported is MIP gap after $15$ minutes of execution, on which node (in thousands) that gap was found, and how many nodes (in thousands) were explored in total.}
\label{table:branch:dyn}

\end{sidewaystable}

Within fractional strategies, branching on the \textit{most fractional} variable generates the best results, although there is little difference with the \textit{less fractional} criteria. There is a significative difference, mostly in low-density graphs, between fractional and degree of saturation strategies. Whereas the former requires less computational time to execute and allows the algorithm to explore a larger number of nodes, the latter obtains a much smaller gap much earlier in the exploration.

We will be using degree of saturation criteria, and test its both alternatives in conjunction with a custom primal heuristic to determine the best branching and primal configuration for the problem.

However, the obtained gaps were better using fixed priorities on the variables than using either dynamic strategy. We may infer that the overhead generated by overriding the engine's default behavior, and the computational effort required iterating all the variables, cause the custom strategies to behave poorer than the fixed priorities. We will compare them again once we add the custom primal heuristic in section \ref{subsec:resultsprimal}.

\subsection{Implied bounds}

As explained in section \ref{subsubsec:alg:branch:bounds}, whenever we fix a variable $x_{i^*j^*}$ to $1$ when branching, we have the possibility of fixing the value of other variables due to logical implications:
\begin{itemize}
\item Fix all other variables in the partition to zero, as only one node must be painted within the partition.
\[
x_{ij} = 0 \quad \forall i \in P(i^*),\ \forall j \in C,\ i \neq i^* \vee j \neq j^*
\]
\item Fix all variables corresponding to adjacent nodes with same color to zero, due to color conflict restrictions.
\[
x_{ij^*} = 0 \quad \forall i \in N(i^*)
\]
\item Fix all color variables $w_j$ to $1$ for all labels less or equal than the current lower bound $\chi_{LB}$ on the chromatic number.
\[
w_j = 1 \quad \forall j \leq \ceil{\chi_{LB}}
\] 
\end{itemize}

We re-ran the previous test with the \textsc{dsatur-2} configuration, with and without these bounds, to check if the overhead generated by forcing multiple bounds on each branching is compensated by the reduction in the branch-and-bound tree. Table \ref{table:branch:bounds} reflects the average gaps for different bounds set.

\begin{table}[h]
\centering

\begin{tabular}{|c|c|c|}
\hline
$x_{ij}$ bounds & $w_j$ bounds & gap \\
\hline
On & On & 0.3225 \\
On & Off & 0.4225 \\
Off & On & 0.4225 \\
Off & Off & 0.4225 \\
\hline
\end{tabular}

\caption{Gaps on branch-and-bound executions for different bounds set during branching, using \textsc{dsatur-2} branching strategy.}
\label{table:branch:bounds}

\end{table}	

Applying all bounds during the process clearly reports the best results. The total number of nodes generated during the process was similar between all the configurations, so the implied bounds do not generate a noticeable overhead. 

Therefore, we will be applying all bounds for both $x_{ij}$ and $w_j$ variables for all upcoming tests.

	%!TEX root = pcp.tex

\subsection{Primal Heuristic}
\label{subsec:resultsprimal}

In this section we evaluate the effectiveness of the devised primal heuristic, in comparison with the default heuristic provided by the \textsc{cplex} engine, in the context of a branch and bound algorithm. We used the same test suite as in section \ref{subsec:resultsbranching}.

\subsubsection*{Using priorities branching}

We first tested the primal heuristic using the simple priorities branching scheme, which offered a good performance according to the results presented in table \ref{table:branch:static}. We ran our branch and bound using \textsc{cplex}'s default heuristic, our custom degree of saturation primal heuristic, and a combination of both. Executions for the regular graphs used were bounded to 15 minutes each, and we executed three instances of each kind. Results are presented in table \ref{table:primal:prios}.

\begin{table}[h]
\label{table:primal:prios}
\centering

\begin{tabular}{|c|ccc|ccc|ccc|}
\hline
\multicolumn{1}{|c|}{Graph} & \multicolumn{3}{|c|}{\textsc{cplex}} & \multicolumn{3}{|c|}{Custom} & \multicolumn{3}{|c|}{Custom + \textsc{cplex}}
\\
 & gap & found & nodes & gap & found & gap & gap & nodes & nodes
\\
\hline
EW 20\% N=90 & 0.00 & 63.33 & 63.33 & 0.00 & 62.33 & 62.33 & 0.00 & 63.33 & 63.33
\\
EW 40\% N=100 & 0.44 & 55.33 & 55.33 & 0.46 & 26.00 & 26.00 & 0.46 & 25.33 & 25.33
\\
EW 60\% N=80 & 0.38 & 96.67 & 98.67 & 0.40 & 27.67 & 28.33 & 0.40 & 27.67 & 27.67
\\
EW 80\% N=100 & 0.45 & 13.00 & 13.00 & 0.41 & 2.67 & 3.33 & 0.41 & 3.33 & 3.33
\\
\hline 
 \end{tabular}

\caption{Obtained gap, node number (in thousands) in which the gap was obtained and total number of nodes explored (in thousands) for different primal heuristics in a branch and bound using priorities branching.}

\end{table}

Using the engine's heuristic in conjunction with the custom heuristic does not report any benefit from using just the custom one. As for using the custom primal heuristic or \textsc{cplex}'s default, the former reports a lower gap on denser graphs, although this trend is reversed in medium-density graphs.

The most noticeable difference is in the total number of nodes generated in the tree within the running time. The processing required by the custom primal heuristic in each step (with running time bounded to 200 ms) is much greater than the built-in heuristic, and this causes that the number of nodes that can be explored in the same time window is much smaller. This shows that the custom heuristic is much more effective, as it obtains similar or even better gaps by exploring only a fraction of the nodes.

\subsubsection*{Using \textsc{dsatur-(C+1)} branching}

Next, we used the \textsc{dsatur-(C+1)} branching criteria instead of the priorities branching strategy. Again, the tests compare the obtained gap, the node number in which that gap was obtained and the total number of nodes explored, for both the \textsc{cplex} engine's primal heuristic and our degree of saturation heuristic presented in \ref{subsec:alg:primal}. Results are presented in table \ref{table:primal:dsatur}.

\begin{table}[h]
\label{table:primal:dsatur}
\centering

\begin{tabular}{|c|ccc|ccc|ccc|}
\hline
\multicolumn{1}{|c|}{Graph} & \multicolumn{3}{|c|}{\textsc{cplex}} & \multicolumn{3}{|c|}{\textsc{dsatur}} & \multicolumn{3}{|c|}{\textsc{dsatur} + \textsc{cplex}}
\\
 & gap & found & nodes & gap & found & nodes & gap & found & nodes 
\\
\hline
EW 20\% N=90 & 0.17 & 76.67 & 95.67 & 0.17 & 83.00 & 93.67 & 0.17 & 76.67 & 90.00
\\
EW 40\% N=100 & 0.33 & 14.33 & 25.00 & 0.33 & 12.00 & 21.33 & 0.33 & 14.33 & 21.00
\\
EW 60\% N=80 & 0.37 & 17.33 & 19.00 & 0.37 & 8.33 & 16.67 & 0.37 & 15.33 & 16.00
\\
EW 80\% N=100 & 0.42 & 1.33 & 2.67 & 0.38 & 2.33 & 2.33 & 0.41 & 1.67 & 2.33
\\
\hline 
\end{tabular}

\caption{Obtained gap, node number (in thousands) in which the gap was obtained and total number of nodes explored (in thousands) for different primal heuristics in a branch and bound using \textsc{dsatur-(C+1)} branching.}

\end{table}

Although there are differences in the gap only on higher density graphs, the custom primal heuristic finds a good solution earlier in the branch and bound tree, which reports more benefits in longer executions. The total number of nodes explored is only slightly larger when the engine's default primal heuristic is used in this case, unlike the previous one, as the cost of the branching strategy seems to amortize the primal heuristic. 

\subsubsection*{Comparison}

Even though the priorities branching scheme by itself generated better results than the degree of saturation branching strategy, when adding a primal heuristic the results changed. Except for low-density graphs, the best configuration is to use \textsc{dsatur-(C+1)} branching together with the custom primal heuristic, without relying on \textsc{cplex} for either branching strategy or primal heuristic. 

In the case of very low-density graphs, using the engine's default strategies works better, as the overhead generated by the custom algorithms is not compensated by the gain in the quality of incumbent solutions. 
	%!TEX root = pcp.tex

\subsection{Cuts}
\label{subsec:resultscuts}

Best configs for clique, block color and iset

clique.colorsAsc: false,

clique.backtrackBrokenIneqs: false,

clique.backtrackLastCandidate: false,

blockColor.usePool: false,

cuts.iset.usePathsAlgorithm: true,

cuts.iset.useBreakingSymmetry: false
	%!TEX root = pcp.tex

\subsection{Branch and cut}
\label{subsec:resultsbnc}

Last but not least, in determining the best configuration for the different components of a branch and cut algorithm, we evaluated the algorithm's performance with different settings relative to the whole branch and cut process. We evaluated different criteria for running the exhaustive implicit enumeration in subtrees, as described in \ref{subsec:alg:implicit}, and also different MIP relative parameters in the underlying \textsc{cplex} framework we used.

\subsubsection{Exhaustive implicit enumeration}

Our first test, once most parameters in the branch and cut algorithm were fixed, was to determine the threshold to run a full \textsc{dsatur} on a node once enough partitions' colors had been fixed during the branching process. Since the algorithm considered only non-fixed partitions for its execution, we experimented with values within acceptable ranges for an exhaustive enumeration: we chose 20, 40 and 60 as the number of remaining partitions to color which triggered the enumeration.

We used graphs of 100 nodes with 2 nodes per partition and different densities, in branch and cut executions of 30 minutes, to check the behaviours of these strategies.

Results were not encouraging, as shown in table \ref{table:bnc:prune}. Setting a low number of unfixed partitions as the threshold to start the exact algorithm caused the algorithm to be never invoked, as the branch and cut itself could prune the whole subtree after very few partitions were colored in the branch process.

On the other hand, making the exact \textsc{dsatur} start earlier caused the algorithm to consume much more time than the available, surpassing the $1800$ seconds bound for high-density graphs, or simply left less time to explore a larger number of nodes, in both cases greatly hurting the obtained gap.

As a result of these experiments, we will be enabling exhaustive implicit enumeration only for very low thresholds ($20$ partitions pending) in order to avoid any problems caused by running the exact algorithm for long periods. Although this setting may cause the algorithm to never be triggered, in other cases such as larger graphs with a greater time bound there is a possibility for this algorithm to actually be effective. 

\begin{sidewaystable}[h]
\label{table:bnc:prune}
\centering

\begin{tabular}{|c|ccc|ccc|ccc|}
\hline
\multicolumn{1}{|c|}{Id} & \multicolumn{3}{|c|}{20} & \multicolumn{3}{|c|}{40} & \multicolumn{3}{|c|}{60}
\\
 & \# times & nodes & gap & \# times & nodes & gap & \# times & nodes & gap
\\
\hline
EW 20 N=100 & 0.00 & 11833.00 & 0.25 & 0.00 & 11834.00 & 0.25 & 115.33 & 11877.00 & 0.25
\\
EW 40 N=100 & 0.00 & 2341.00 & 0.22 & 0.00 & 2340.67 & 0.22 & 192.33 & 2367.67 & 0.30
\\
EW 60 N=100 & 0.00 & 1225.67 & 0.22 & 0.00 & 1225.67 & 0.22 & 345.00 & 1050.00 & 0.29
\\
EW 80 N=100 & 0.00 & 307.67 & 0.19 & 2.00 & 313.67 & 0.19 & 28.00 & 119.00 & 0.21 (*)
\\
\hline 
 \end{tabular}

\caption{Average number of times the enumeration was triggered, number of nodes in the tree and resulting gap, for different number of uncolored partitions for triggering the exhaustive enumeration. The execution marked with a (*) indicate that the execution of the enumeration algorithm took an unacceptable amount of time for the imposed bounds.}

\end{sidewaystable}

\subsubsection{Probing}

An available setting in the \textsc{cplex} framework is the probing level. This controls how much processing is invested in a preprocessing stage to derive logical implications from setting binary variables to a fixed value. As \textsc{cplex}'s manual \cite{cplex121} explains:

\begin{quote}

Probing is a technique that looks at the logical implications of fxing each binary variable to 0 (zero) or 1 (one). It is performed after preprocessing and before the solution of the root relaxation. Probing can be expensive, so this parameter should be used selectively. On models that are in some sense easy, the extra time spent probing may not reduce the overall time enough to be worthwhile. On diffcult models, probing may incur very large runtime costs at the beginning and yet pay off with shorter overall runtime. 

\end{quote}

We experimented with binomial graphs of fixed size and different densities, as usual, with different probing levels set. Results are shown in table \ref{table:bnc:probing}, and differ greatly between different densities.

For low densities, a moderate level of probing seems to be the best option, as it managed to explore a greater amount of nodes in the tree during the imposed $1800$ seconds. 

On the other hand, greater densities seems to benefit more from disabling probing whatsoever, as the custom bounds implied during the branch process (see \ref{subsubsec:alg:branch:bounds}) benefit largely from higher-degree nodes, making the engine's probing unnecesary and yielding a better gap.

Therefore, we will be using moderate probing settings for low density graphs, and disabling probing altogether for higher densities.

\begin{sidewaystable}[h]
\label{table:bnc:probing}
\centering

\begin{tabular}{|c|cc|cc|cc|cc|}
\hline
\multicolumn{1}{|c|}{Id} & \multicolumn{2}{|c|}{disabled} & \multicolumn{2}{|c|}{moderate} & \multicolumn{2}{|c|}{aggressive} & \multicolumn{2}{|c|}{very aggressive}
\\
 & nnodes & gap & nnodes & gap & nnodes & gap & nnodes & gap
\\
\hline
EW 20 N=100 & 11319.00 & 0.25 & 23284.33 & 0.25 & 24523.67 & 0.25 & 24517.67 & 0.25
\\
EW 40 N=100 & 2366.67 & 0.22 & \textbf{5396.67} & \textbf{0.22} & 2348.00 & 0.22 & 2345.33 & 0.22
\\
EW 60 N=100 & 1227.67 & 0.22 & 1227.33 & 0.22 & 1171.67 & 0.22 & 1171.33 & 0.22
\\
EW 80 N=100 & \textbf{347.00} & \textbf{0.15} & 346.00 & 0.19 & 308.33 & 0.19 & 309.00 & 0.19
\\
\hline 
 \end{tabular}
 
\caption{Average number of nodes in the tree and resulting gap, for different MIP probing levels.}

\end{sidewaystable}

\subsubsection{Emphasizing feasibility and optimality}

Arriving to an optimal solution in a branch and cut algorithm requires both (1) obtaining integral feasible solutions of decreasing objective value, and (2) generate a proof that the best integral solution obtained is actually an optimum. The emphasis the framework puts on these two parts of the algorithm is controlled by a \textit{MIP emphasis} parameter, which can be given the following values:

\begin{itemize}
\defitem{Balanced}{Have a reasonable balance between feasibility and optimality, which is the default behaviour.}
\defitem{Emphasize feasibility}{Focus on feasibility instead of optimality, which produces better solutions earlier and works better under tight time constraints when an optimality proof is not necessary.}
\defitem{Emphasize optimality}{Focus on the proof of optimality by attempting to raise the best bound\footnote{The best bound is the lowest possible value that an integer feasible solution could have.} faster.}
\defitem{Emphasize best bound}{Focus even more in the proof of optimality by attempting solely to move the best bound; this causes intermediate optimal solutions to be rarely found as it cares exclusively to arrive to a final optimal solution.}
\defitem{Hidden feasibility}{Attempts to find high quality feasible solutions that are considered hidden, this is, difficult to obtain through the branch and cut process; this causes the proof of optimality to take longer than with other settings.}
\end{itemize}

We evaluated these different configurations in the usual set of binomial graphs, reporting both gaps and number of nodes explored in the tree; results are shown in table \ref{table:bnc:emph}. All of them arrived to the same gap values, but there were observable differences between the number of nodes explored in the tree.

For low density graphs, emphasizing the best bound yielded the highest number of nodes explored within the same time frame, while in higher density graphs a balanced approach managed to explore more nodes. These configurations will be used for further experimentation, depending on the processed graph's density.

\begin{sidewaystable}[h]
\label{table:bnc:emph}
\centering

\begin{tabular}{|c|cc|cc|cc|cc|cc|}
\hline
\multicolumn{1}{|c|}{Id} & \multicolumn{2}{|c|}{balanced} & \multicolumn{2}{|c|}{feasibility} & \multicolumn{2}{|c|}{optimality} & \multicolumn{2}{|c|}{best bound} & \multicolumn{2}{|c|}{hidden}
\\
 & nodes & gap & nodes & gap & nodes & gap & nodes & gap & nodes & gap
\\
\hline
EW 20 N=100 & 11840.00 & 0.25 & 11705.00 & 0.25 & 18266.33 & 0.25 & \textbf{20810.00} & 0.25 & 11841.33 & 0.25
\\
EW 40 N=100 & 2343.33 & 0.22 & 2186.00 & 0.22 & 3055.33 & 0.17 & \textbf{3707.00} & \textbf{0.17} & 2342.67 & 0.22
\\
EW 60 N=100 & \textbf{1226.00} & 0.22 & 1219.33 & 0.22 & 820.33 & 0.22 & 675.67 & 0.22 & 1225.33 & 0.22
\\
EW 80 N=100 & \textbf{308.67} & 0.19 & 305.33 & 0.19 & 188.67 & 0.19 & 104.67 & 0.19 & 308.00 & 0.19
\\
\hline 
\end{tabular}

 
\caption{Average number of nodes in the tree and resulting gap, for different MIP emphasis settings.}

\end{sidewaystable}

\clearpage

\subsubsection{Alternate models in branch and cut}
%
%\begin{itemize}
%\item solution.chi
%\item solution.nnodes
%\item solution.time
%\item solution.gap
%\end{itemize}
%Series:
%\begin{itemize}
%\item S1: strategy.partition: PaintAtLeastOne, strategy.adjacency: AdjacentsNeighbourhood, strategy.symmetry: MinimumNodeLabel, strategy.colorBound: UpperNodesSumLowerSumPartition, strategy.objective: Equal, solver.probing: 1, solver.mipEmphasis: 3
%\item S2: strategy.partition: PaintExactlyOne, strategy.adjacency: AdjacentsNeighbourhood, strategy.symmetry: MinimumNodeLabel, strategy.colorBound: UpperNodesSumLowerSumPartition, strategy.objective: Equal, solver.probing: 1, solver.mipEmphasis: 3
%\item S3: strategy.partition: PaintExactlyOne, strategy.adjacency: AdjacentsNeighbourhood, strategy.symmetry: UseLowerLabelFirst, strategy.colorBound: UpperNodesSum, strategy.objective: Equal, solver.probing: -1, solver.mipEmphasis: 3
%\end{itemize}
%\begin{tabular}{|c|ccc|ccc|ccc|}
%\hline
%\multicolumn{1}{|c|}{Id} & \multicolumn{3}{|c|}{S1} & \multicolumn{3}{|c|}{S2} & \multicolumn{3}{|c|}{S3}
%\\
% & solution.nnodes & solution.time & solution.gap & solution.nnodes & solution.time & solution.gap & solution.nnodes & solution.time & solution.gap
%\\
%\hline
%EW 20 N=90 & 23078.25 & 579.85 & 0.00 & 19959.75 & 1161.73 & 0.00 & 27248.25 & 1800.29 & 0.19
%\\
%EW 40 N=90 & 9387.75 & 2400.06 & 0.17 & 13978.50 & 2400.05 & 0.17 & 8405.50 & 2400.03 & 0.17
%\\
%\hline 
% \end{tabular}
%
%
%\begin{itemize}
%\item solution.chi
%\item solution.nnodes
%\item solution.time
%\item solution.gap
%\end{itemize}
%Series:
%\begin{itemize}
%\item S1: strategy.partition: PaintExactlyOne, strategy.adjacency: AdjacentsLeqOne, strategy.symmetry: UseLowerLabelFirst, strategy.colorBound: UpperNodesSum, strategy.objective: Equal, solver.probing: -1, solver.mipEmphasis: 0
%\item S2: strategy.partition: PaintExactlyOne, strategy.adjacency: AdjacentsNeighbourhood, strategy.symmetry: MinimumNodeLabel, strategy.colorBound: UpperNodesSumLowerSumPartition, strategy.objective: Equal, solver.probing: -1, solver.mipEmphasis: 0
%\item S3: strategy.partition: PaintExactlyOne, strategy.adjacency: AdjacentsNeighbourhood, strategy.symmetry: VerticesNumber, strategy.colorBound: UpperNodesSumLowerSumPartition, strategy.objective: Equal, solver.probing: -1, solver.mipEmphasis: 0
%\end{itemize}
%\begin{tabular}{|c|cc|cc|cc|}
%\hline
%\multicolumn{1}{|c|}{Id} & \multicolumn{2}{|c|}{S1} & \multicolumn{2}{|c|}{S2} & \multicolumn{2}{|c|}{S3}
%\\
% & nodes & gap & nodes & gap & nodes & gap
%\\
%\hline
%EW 60 N=90 & 1750.75& 0.23 & 2381.00 & 0.23 & 2200.25 & 0.23
%\\
%EW 80 N=90 & 477.50 & 0.16 & 1133.00& 0.15 & 644.75 &0.16
%\\
%\hline 
% \end{tabular}
	\section{Final Results}

Having fixed the best configurations of the algorithm for binomial random graphs in the previous sections, such as models, initial heuristic, branching strategies, primal heuristic and cuts, we now compare our branch and cut algorithm for \PCP{} against other solutions.

\subsection{Comparison with CPLEX}

The first evaluation to perform is to analyze the improvement introduced by the custom modifications we made on top of the \textsc{cplex} engine, by comparing our results to those returned by an unmodified execution of \textsc{cplex}\footnote{All tests were performed against version 12.1 of \textsc{cplex}.}. 

We used binomial random graphs with 90 nodes, 2 nodes per partition, and picked 2 instances for each node-density pair; with running time of 2 hours.

First, we compared our algorithm to \textsc{cplex}'s default branch and cut, both with and without fixing an initial clique and performing other simplifications to the model (described in section \ref{sec:bnc}). In all cases we provided the same initial solution $\chi_0$, which considerably reduced the number of variables in the model by eliminating those $x_{ij}$ and $w_j$ with $j > \chi_0$.

Then, we performed the same tests, but this time using \textsc{cplex} dynamic search algorithm, instead of the traditional branch and cut we were using. This algorithm, introduced in version 11 of \textsc{cplex} and improved in version 12, uses the same building blocks as traditional branch and cut, but does not allow for user customization via callbacks, therefore working as a black box solver, often yielding better results than its counterpart.

\begin{table}[h]
\centering
\begin{tabular}{|c|cc|cc|cc|}
\hline
\multicolumn{1}{|c|}{Graph} & \multicolumn{2}{|c|}{Cplex branch and cut} & \multicolumn{2}{|c|}{Cplex branch and cut} & \multicolumn{2}{|c|}{Custom \PCP{}}
\\
\multicolumn{1}{|c|}{} & \multicolumn{2}{|c|}{w/o initial clique} & \multicolumn{2}{|c|}{with initial clique} & \multicolumn{2}{|c|}{branch and cut}
\\
\hline
& gap & time & gap & time & gap & time
\\
\hline
EW 20\% N=90 & 0.0\% & 1.49 & 0.0\% & 1.48 & 0.0\% & 0.141
\\
EW 40\% N=90 & 16.7\% & 7200 & 16.7\% & 7200 & 16.7\% & 7200
\\
EW 60\% N=90 & 33.3\% & 7200 & 36.7\% & 7200 & 22.2\% & 7200
\\
EW 80\% N=90 & 26.7\% & 7200 & 23.3\% & 7200 & 11.3\% & 7200
\\
\hline 
\end{tabular} 
\caption{Average gap and running time in seconds for graphs with different densities, comparing our custom \PCP{} branch and cut algorithm with \textsc{cplex}'s default branch and cut, with and without fixing an initial clique for the resolution.}
\label{table:final:cplexbnc}
\end{table}

\begin{table}[h]
\centering
\begin{tabular}{|c|cc|cc|cc|}
\hline
\multicolumn{1}{|c|}{Graph} & \multicolumn{2}{|c|}{Cplex Dynamic Search} & \multicolumn{2}{|c|}{Cplex Dynamic Search} & \multicolumn{2}{|c|}{Custom \PCP{}}
\\
\multicolumn{1}{|c|}{} & \multicolumn{2}{|c|}{w/o initial clique} & \multicolumn{2}{|c|}{with initial clique} & \multicolumn{2}{|c|}{branch and cut}
\\
\hline
& gap & time & gap & time & gap & time
\\
\hline
EW 20\% N=90 & 0.0\% & 0.758 & 0.0\% & 0.758 & 0.0\% & 0.141
\\
EW 40\% N=90 & 16.7\% & 7200 & 16.7\% & 7200 & 16.7\% & 7200
\\
EW 60\% N=90 & 22.2\% & 7200 & 22.2\% & 7200 & 22.2\% & 7200
\\
EW 80\% N=90 & 11.8\% & 7200 & 12.0\% & 7200 & 11.3\% & 7200
\\
\hline 
\end{tabular} 
\caption{Average gap and running time in seconds for graphs with different densities, comparing our custom \PCP{} branch and cut algorithm with \textsc{cplex}'s dynamic search, with and without fixing an initial clique for the resolution.}
\label{table:final:cplexdynamicsearch}
\end{table}

The obtained results showed that the customizations oriented towards solving the \PCP{} did produce an improvement in the solution. The difference with \textsc{CPLEX}'s traditional branch and cut algorithm is remarkable, requiring $10\%$ of the time to solve to optimality in sparse instances, and achieving more than a $10\%$ improvement in graphs with a high density.

As for \textsc{CPLEX} dynamic search, it is clear that it performs much better than its branch and cut, but there are still improvements attained by out algorithm in graphs with very low and high density, in terms of time and gap respectively.

Something interesting to notice is that fixing the initial clique does not always report a benefit when running \textsc{CPLEX} algorithms, even though it did report a significative improvement on our customized algorithm.

\subsection{Comparison with Asymmetric Representatives Branch and Cut}

We also compared our algorithm to the other branch and cut algorithm designed specifically for the \PCP{} we found in the literature: the one devised by Frota, Maculan, Noronha and Ribeiro in \cite{frota2010branch}, based on the asymmetric representatives formulation for traditional coloring, (\cite{campelo2004cliques},\cite{campelo2008asymmetric}).

It is worth noting that both implementations and execution environments differ considerably. While the aforementioned algorithm was run under Linux, implemented in C++ and  using \textsc{XPRESS} to solve linear relaxations, our algorithm was executed in Windows, implemented in Java and built on top of the \textsc{CPLEX} engine using its Java API. This makes both algorithms difficult to compare using the reported results of their respective implementations; nevertheless, we will be presenting this comparison as an informative result.

Even though multiple results are presented in \cite{frota2010branch}, we focused in the number of instances reported to have been solved to optimality in random graphs with $90$ nodes and $2$ nodes per partition, with different edge densities (which is also the kind of graphs in which we focused our testing in these last sections). 

For each density, we executed our algorithm in five instances of binomial (Erdos-R\'enyi) graphs and five instances of powerlaw-cluster\footnote{As described in \ref{subsec:results:instances}, these graphs are generated by three parameters: node count $n$, number of nodes $m$ to which each new node is attached, and probability $p$ to add an extra edge generating a triangle when a new node is added. These graphs are constructed with an initial empty graph of size $m$. In order to attain densities higher than $50\%$ with this kind of graphs, we modified the algorithm to start with a binomial graph of size $m$, and iteratively expand it to the desired node count $n$ using the original procedure.} (Holme-Kim) graphs. Every graph instance was ran until the optimal solution was found, or for up to two hours.

\begin{table}[h]
\centering
\begin{tabular}{|c|rl|rl|rl|}
\hline
Graph & \multicolumn{2}{|c|}{B\&C} & \multicolumn{2}{|c|}{B\&C} & \multicolumn{2}{|c|}{Frota et al.} \\
Density & \multicolumn{2}{|c|}{Binomial} & \multicolumn{2}{|c|}{Holme-Kim} & \multicolumn{2}{|c|}{} \\
\hline
20\% & \b{100\%} & (5/5) & \b{100\%} & (5/5) & 20\% & 3/15 \\
40\% & 0\% & (0/5) & \b{100\%} & (5/5) & 7\% & 1/15 \\
60\% & 0\% & (0/5) & 60\% & (3/5) & \b{80\%} & 12/15 \\
80\% & 0\% & (0/5) & 0\% & (0/5) & \b{100\%} & 15/15 \\
\hline 
\end{tabular} 
\caption{Fraction of the tested instances that were solved to optimality using the implemented branch and cut algorithm on both binomial and powerlaw cluster graphs of different densities, and fraction solved to optimality as reported by Frota et al. in \cite{frota2010branch}.}
\label{table:final:frotaetal}
\end{table}

The obtained results (presented in table \ref{table:final:frotaetal}) are most interesting. Whereas our algorithm outperforms \cite{frota2010branch} in low density graphs, the latter wins in high density graphs. It is also worth noting that our algorithm handles clustered graphs much better than binomial ones, most probably because of the high level of symmetry usually found in binomial graphs.

Also, both algorithms have a tight bound on the difference between the lower bound and the solution found in most cases: for algorithm by Frota et al. this difference is never greater than one color, and in our algorithm it reaches its maximum difference of two colors only in a few high-density binomial graphs.

\vskip 25pt

These results show that different models and different algorithms can tackle the same problem efficiently in different cases. While our algorithm easily solved low density instances, the branch and cut based on the asymmetric representatives model for the \PCP{} performs clearly better in high density graphs. Instances with medium density are still the most difficult to solve, as also happens in standard coloring problems.

%!TEX root = pcp.tex

\section{Conclusions}
\label{sec:conclusions}

Conclusions...
\include{references}

\end{document}

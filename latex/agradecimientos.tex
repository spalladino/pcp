\begin{otherlanguage}{spanish}
\renewcommand{\abstractname}{Agradecimientos}
\begin{abstract}

A Isabel y Paula, mis directoras, no solamente por la gu�a indispensable que fueron para el desarrollo de este trabajo (con notable paciencia), sino por haberme ense\~nado todo lo que s� acerca de M�todos Num�ricos e Investigaci�n Operativa, y haberme transmitido su gran pasi�n por estas �reas. Su trabajo y dedicaci�n como directoras es solamente comparable a su desempe\~no como las excelentes docentes que son.

A Irene Loiseau y Esteban Feuerstein, por haber aceptado ser jurados de este trabajo, y por las correcciones que permitieron concluirlo como corresponde.

A todo el Departamento de Computaci�n de la Facultad de Ciencias Exactas y Naturales de la UBA, especialmente a sus docentes, por haberme brindado una educaci�n excelente y gratuita; as� como a los grupos de Algoritmos y M�todos Num�ricos, por darme un espacio como ayudante en el cual devolver una peque\~na parte de lo aprendido.

A mis compa\~neros de la cursada, la gente-de-la-facu, junto a quienes recorr� la carrera y a quienes les debo las alegr�as dentro de la facultad en estos �ltimos seis a\~nos: Mat�as Blanco, Fernando Bugni, Luis Brassara, Facundo Carreiro, Bruno Cuervo Parrino, Diego Freijo, Maxi Giusto, Pablo Laciana, Sergio Medina, Ariel Neisen, Leo Rodr�guez, Nati Rodriguez, Viviana Siless, Javier Silveira, Leo Spett, Pablo Zaidenvoren, Eddy Zoppi; con una menci�n m�s que especial para Leandro Radusky, Andr�s Taraciuk y Mart�n Verzilli, mis compa\~neros infaltables de TPs, con quienes aprend� a trabajar codo a codo y pudimos llevar adelante todos los trabajos con los que nos enfrentamos.

A la gente de investigaci�n de la facultad, por haberme hecho un lugar dentro de la gran familia que representan: Nicol�s Botbol, Juan Manuel Chaneton, Alexandra Diehl, Marina Groshaus, Fernando Hern�ndez, Leandro Montero, Mercedes P�rez Mill�n, Juan Pablo Puppo, Francisco Soulignac, Juanjo Miranda y Pablo Factorovich; especialmente a estos dos �ltimos por haberme ayudado en incontables ocasiones y servido de ejemplo a seguir acerca de lo que significa el desarrollo en Investigaci�n Operativa.

A Manas, mi lugar de trabajo en la industria desde hace m�s de tres a\~nos: como empresa, por haberme brindado un espacio dentro de la misma para el desarrollo de mi tesis; y m�s especialmente, como grupo humano, por ser un lugar donde trabajar y compartir el d�a a d�a con excelentes profesionales y a�n mejores personas. Una menci�n especial para Caro Hadad, por la paciencia para hacer de audiencia ante los ensayos de la defensa de este trabajo.

A mis amigos que a�n est�n a mi lado desde el colegio: Mariana Lavia, Virginia Raschia, Gaby Revale, Pablo Cagnoni, Mauro Lampo, Ale Maggi, Nicol�s Muschirintello, Mart�n Kalos, Leandro Paizal y Hern�n S�nchez; con quienes hemos compartido ya la mitad de lo que va de la vida, si no m�s en algunos casos, y a quienes agradezco por su amistad durante todo este tiempo.

A mis padres y a Mariano, mi hermano; por el apoyo incondicional siempre, por haberme hecho la persona que soy, y por haberme permitido llegar hasta esta etapa de mi vida.

A todos ellos, y a todos los que fueron una parte de esta larga carrera o de los pasos para llegar hasta ella, muchas gracias!!

\end{abstract}
\end{otherlanguage}
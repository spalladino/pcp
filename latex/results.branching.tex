%!TEX root = pcp.tex

\subsection{Branching strategies}
\label{subsec:resultsbranching}

We evaluated the different branching strategies described in section \ref{subsec:alg:branching} on regular graphs with fixed size and different density, in order to determine which reports the best results. We used a simple branch and bound algorithm bounded to 15 minutes of running time.

\subsubsection*{Priorities}

The first test we implemented applied only priorities on the variables during the problem's initialization. Priorities were assigned according to the following formula:
\begin{equation*}
	prio(x_{ij}) = \alpha * \delta_P(v_i) + \beta * j
\end{equation*}

We tested with different values for $\alpha$ and $\beta$, both positive and negative, to generate different priorities. Although we found hardly any differences in higher density graphs, the ones with the lowest densities ($20\%$) did have significative differences.

In table \ref{table:branch:static} we report those $(\alpha,\beta)$ values which gave results better or near the ones obtained when not using priorities, this is, allowing \textsc{cplex} to choose automatically which variable to branch on.

\begin{table}
\label{table:branch:static}
\centering

\begin{tabular}{|c|c|c|}
\hline
\textbf{Priorities} & \textbf{Time} & \textbf{Gap} \\
\hline
$\alpha = 10$, $\beta = -1$ &  232.57 & 0.00 \\
$\alpha = 10$, $\beta = 1$ & 523.10 & 0.00 \\
cplex & 570.72 & 0.00 \\
\hline
 \end{tabular}

Clearly giving the highest priority to nodes with the highest $\delta_P(v_i)$ value, tie-breaking in favor of higher color labels, is the best static branching priority.

\caption{Gap and running time for branch and bound executions on $20\%$ density graphs with different priorities on the branching variables.}

\end{table}	

\subsubsection*{Dynamic strategies}

Having fixed the priorities to set on the variables, we use them as tie breaking strategies for the two devised strategies which depend on the variable's value (\ref{subsubsec:alg:branch:frac} and \ref{subsubsec:alg:branch:dsatur}). 

We set up a suite of graphs of different size and density to test most and less fractional strategies, as well as both degree of saturation strategies: branching on a particular $x_{ij}$ variable or creating one subproblem for each possible color for a particular node $v_i$. Results are displayed in table \ref{table:branch:dyn}; we report the resulting MIP gap, on which node that gap was obtained, and how many nodes were explored during the 15 minutes of execution.

\begin{sidewaystable}
\label{table:branch:dyn}
\centering

\begin{tabular}{|c|ccc|ccc|ccc|ccc|ccc|ccc|}
\hline
\multicolumn{1}{|c|}{Graph} & \multicolumn{3}{|c|}{dsatur $2$} & \multicolumn{3}{|c|}{dsatur $C+1$} & \multicolumn{3}{|c|}{less fractional} & \multicolumn{3}{|c|}{most fractional} 
\\
 & gap & found & nodes & gap & found & nodes & gap & found & nodes & gap & found & nodes
\\
\hline
EW 20\% N=90 & 0.17 & 77 & 94 & 0.17 & 76 & 95 & 0.25 & 177 & 250 & 0.25 & 134 & 178 
\\
EW 40\% N=100 & 0.33 & 20 & 24 & 0.33 & 14 & 24 & 0.39 & 27 & 39 & 0.33 & 30 & 44 
\\
EW 60\% N=80 & 0.37 & 7 & 18 & 0.37 & 18 & 19 & 0.37 & 30 & 32 & 0.37 & 23 & 27 
\\
EW 80\% N=100 & 0.42 & 2 & 2 & 0.42 & 1 & 3 & 0.44 & 3 & 4 & 0.42 & 4 & 4
\\
\hline 
 \end{tabular}

\caption{Results for fractional and degree of saturation (spanning either $2$ or $C+1$ subproblems) branching strategies on branch and bound schemes. Data reported is MIP gap after $15$ minutes of execution, on which node (in thousands) that gap was found, and how many nodes (in thousands) were explored in total.}

\end{sidewaystable}

Within fractional strategies, branching on the \textit{most fractional} variable generates the best results, although there is little difference with the \textit{less fractional} criteria. There is a significative difference, mostly in low-density graphs, between fractional and degree of saturation strategies. Whereas the former requires less computational time to execute and allows the algorithm to explore a larger number of nodes, the latter obtains a much smaller gap and much earlier in the exploration.

We will be using degree of saturation criteria, and test its both alternatives in conjunction with a custom primal heuristic to determine the best branching and primal configuration for the problem.

\begin{itemize}
\item \sout{Static priorities: See branchstatic1, S4 wins, dont include tables}
\item \sout{Dynamics: Compare fractional, dsatur, with and without color consec; both dsatur win, regardless consec color, gap found better, dont compare with gap}
\item Bounds: Once dsatur is fixed, try with and without fixing bounds, see fixing works much better
\item Remark statics are faster than dynamics due to processing time
\end{itemize}
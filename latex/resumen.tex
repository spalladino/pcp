\begin{otherlanguage}{spanish}
\begin{abstract}

El problema de coloreo particionado, \PCP{}, es una generalizaci�n del cl�sico problema de coloreo de grafos. En esta variante el conjunto de nodos del grafo de entrada se encuentra particionado, y el problema consiste en colorear un solo nodo por partici�n utilizando la menor cantidad de colores posible, manteniendo la restricci�n de que dos nodos adyacentes no pueden compartir color.

Este problema fue propuesto por Li y Simha en el contexto del problema de \textit{ruteo y asignaci�n de longitudes de onda} (RWA) en redes \textit{multiplexadas por divisi�n de longitud de onda} (WDM). Dichos autores proponen una resoluci�n en dos etapas: una primera en la que se generan posibles caminos como soluciones factibles para el problema de ruteo, y una segunda en la que se determinan los caminos a usar y se les asignan longitudes de onda, buscando minimizar la cantidad de longitudes de onda usadas. Esta �ltima etapa se corresponde con una instancia del \PCP{}.

El \PCP{}, al igual que coloreo tradicional de grafos, es un problema NP completo, con lo que no se conoce un algoritmo que pueda resolverlo en tiempo polinomial. Por este motivo, la mayor�a de los enfoques para resolver este problema se basan en t�cnicas heur�sticas, dejando poco lugar a algoritmos exactos para la resoluci�n del mismo.

En este trabajo modelamos el \PCP{} como un problema de programaci�n lineal entera, generalizando el modelo propuesto por M�ndez-D�az y Zabala para coloreo de grafos, lo que nos permite resolverlo mediante la t�cnica de \textit{branch and cut}. Para ello, desarrollamos una heur�stica inicial, una heur�stica primal, estrategias de branching, y algoritmos de separaci�n para distintas familias de desigualdades v�lidas que caracterizamos para el poliedro. A partir de estos componentes implementamos el algoritmo de \textit{branch and cut} para la resoluci�n del \PCP{}.

\end{abstract}
\end{otherlanguage}
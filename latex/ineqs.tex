%!TEX root = pcp.tex

\chapter{Valid inequalities}
\label{sec:ineqs}

In the previous chapter, we defined the basic \PCP{} polyhedron as the set of points that satisfy the inequalities presented in \ref{subsec:model:formulation} (\ref{eqn:obj}, \ref{eqn:partsum} and \ref{eqn:adjscolorp}), along with a number of variants of that model, which span alternative polyhedra.

In this chapter we will derive valid inequalities for the basic \PCP{} polyhedron, which will be used as cutting planes in the branch-and-cut algorithm. It is important to note that the inequalities derived are valid for both the basic polyhedron and for all its presented alternatives; therefore, they can be used as cuts in the branch and cut algorithm regardless of which model is implemented.

\section{Extended clique inequalities}

A classical inequality for the standard coloring problem is the clique inequality, which establishes that within a clique $K$, at most one node can be colored with a label $j$.

\begin{equation}
\nonumber
\sum_{i \in K} x_{ij} \leq w_{j} \quad \forall j \in C
\end{equation}

Combining this inequality with the fact that in \PCP{} at most one node per partition can be colored with a label $j$, we define the \textit{extended clique inequality} for PCP. Recall from \ref{subsec:intro:definitions} that an extended clique is a maximal subset $K_P$ of $V$ such that every pair of nodes is either adjacent or belong to the same partition. These inequalities specify that among all nodes in $K_P$ at most one of them may use color $j$.

\begin{equation}
\label{ineq:extendedclique}
\sum_{i \in K_P} x_{ij} \leq w_{j} \quad \forall j \in C
\end{equation}

Similar inequalities were developed by \cite{frota2010branch}, based on the asymmetric representatives formulation, but applied only on component cliques\footnote{Clique in which every node belongs to a different partition.}. Extended cliques have the added benefit of covering a larger set of nodes, being stronger than their component-based counterparts, and maintain their effectiveness regardless of the partition size used.

\section{Component independent set inequalities}

As was defined in \ref{subsec:intro:definitions}, a component independent set $I_P$ is a standard independent set with the added restriction that every node must belong to a different partition, and $\alpha_P(G)$ is the size of the largest component independent set of a graph. These definitions allow us to adapt the independent set inequality directly from the standard coloring problem:

\begin{equation}
\label{ineq:ciset}
\sum _{i \in W} x_{ij} \leq \alpha_P(W) w_{j} \quad \forall j \in C
\end{equation}

The restriction is applied to a subgraph of $G$ induced by the nodes $W \subseteq V$. Since the cardinal of the maximum component independent set of the subgraph, $\alpha_P(W)$, is not easy to calculate, as it is as difficult as the coloring problem itself, this inequality is applied to particular subsets of the graph with an $\alpha_P$ easy to determine\footnote{Note that even though component clique inequalities, which state that every vertex in a component clique must have a different color, are a particular instantiation of these inequalities, they are superseded by extended clique inequalities, as the latter bound the sum of a larger set of nodes; this occurs because every component clique is an extended clique, but not the other way around. As such, component clique inequalities will not be considered.}.

\subsection{Component hole inequalities}

A simple instantiation of the previous inequality can be done by picking a subset $W$ that induces a component hole $H$\footnote{A component hole is a chordless cycle in which every node belongs to a different partition.} in the partitioned graph. As in a standard hole, it holds that $\alpha_P(H) = \floor{\frac{n}{2}}$, where $n$ is the length of the hole, therefore the only effort required lies in finding a component hole in the graph.

Therefore, given a component hole $H$ in the partitioned graph, the component hole inequality is:

\begin{equation}
\label{ineq:chole}
\sum _{i \in H} x_{ij} \leq \floor{\frac{n}{2}} w_{j} \quad \forall j \in C
\end{equation}

\subsection{Component path inequalities}

Similar to the previous case, the component independent set can be instantiated with a component path $P$, which is a standard path where every node belongs to a different partition. In this case, it holds that for every component path of length $n$, $\alpha(P) = \ceil{\frac{n}{2}}$, and the inequality results:

\begin{equation}
\label{ineq:cpath}
\sum _{i \in P} x_{ij} \leq \ceil{\frac{n}{2}} w_{j} \quad \forall j \in C
\end{equation}

\subsection{Strengthening by breaking symmetry}

Component independent set inequalities can be strengthened by taking into consideration symmetry breaking constraints, which forbid using a color unless all of the previous colors are used (\ref{eqn:lowerlabel} being the weakest). This means that the following inequalities will only be valid for models which include symmetry breaking constraints.

In case a component independent of size $\alpha_P$ set is colored with label $j^* \leq q - \alpha_P$, then it is possible to ensure that the colors with the highest $\alpha_P +1$ labels, $j_{q-\alpha_P+2} \ldots$, will be left unused, since there are $\alpha_P$ nodes using the same color $j^*$.

Being $W$ the component independent set, in the worst case, in which all nodes in $V \setminus W$ use different colors, an assignment as the one shown in table \ref{table:cisetcoloring} will occur. This coloring uses only the first $q - \alpha_P + 1$ colors, and leaves all colors with a greater label unused. 

\begin{table}
\centering	
\begin{tabular}{cc}
\hline
\textbf{color label} & \textbf{partitions count} \\
\hline
$j_1$ & $1$\\
$j_2$ & $1$\\
\vdots & \vdots \\
$j^*$ & $\alpha_P$ \\
\vdots & \vdots \\
$j_{q - \alpha_P}$ & $1$\\
$j_{q - \alpha_P + 1}$ & $1$\\
$j_{q - \alpha_P + 2}$ & $0$\\
\vdots & \vdots \\
$j_{q}$ & $0$\\
\hline
\end{tabular}
\caption{Worst-case color assignment when a component independent set of size $\alpha_P$ is found in the partitioned graph.}
	\label{table:cisetcoloring}
\end{table}

This assignment may happen only if every node in $V \setminus W$ uses a different color; if there is any node repeating color then label $q - \alpha_P + 1$ is unused as well. Therefore, at most one node may be colored using $q - \alpha_P + 1$, while colors with a greater label will never be used, so the following inequality holds:

\begin{equation}
\label{ineq:cisetbshigh}
\sum ^c _{j = j_t + 1} \sum _{i \in V} x_{ij} \leq w_{j_t + 1} \quad j_t = q - \alpha_P(W)
\end{equation}

Combining this inequality with \ref{ineq:ciset}, results in the following component independent set inequality, which becomes strengthened via symmetry breaking:

\begin{equation}
\label{ineq:cisetbs}
\sum_{i \in W} x_{ij_0} + \sum ^c _{j = j_t + 1} \sum _{i \in V} x_{ij} \leq \alpha_P(W) w_{j_0} + w_{j_t + 1} \quad \forall j_0 \leq j_t, \; j_t = q - \alpha_P(W)
\end{equation}

Both component hole (\ref{ineq:chole}) and component path inequalities (\ref{ineq:cpath}) can be strengthened using this argument.

\section{Partition graph inequalities}

Let $G' = <V',E'>$ be the partition graph\footnote{The \textit{partition graph} $G'$ of a partitioned graph $G$ is a standard graph $G'=<V',E'>$ in which every node $v'_k \in V'$ corresponds to a partition $P_k \in P$, and two nodes $v'_i,v'_j \in V'$ are adjacent if and only if every node in $P_i$ in $G$ is adjacent to every node in $P_j$.} of $G$. Most bounds found for coloring $G'$ can be reused in the original $G$ by extending the constraint over every node represented by each $p \in V'$.

A clear example are independent set inequalities. Let $W' \subseteq V'$ a subset of nodes inducing a subgraph in $G'$, then the independent set inequality holds:

\begin{equation}
\label{ineq:gpiset}
\sum _{i \in W'} x_{ij} \leq \alpha(W') w_{j} \quad \forall j \in C
\end{equation}

As in inequalities \ref{ineq:cisetbs}, these inequalities can be strengthened considering symmetry breaking constraints:

\begin{equation}
\label{ineq:gpisetbs}
\sum_{i \in W'} x_{ij_0} + \sum ^c _{j = j_t + 1} \sum _{i \in V'} x_{ij} \leq \alpha(W') w_{j_0} + w_{j_t + 1} \quad \forall j_0 \leq j_t, \; j_t = q - \alpha(W')
\end{equation}

These constraints over $G'$ can be converted to constraints $G$ by replacing every node $p \in V'$ with the sum over the nodes $v \in P_p$. Let $W \subseteq P$ be the set of partitions represented by the nodes in $W' \in V'$ in $G'$, then:

\begin{equation}
\label{ineq:gpisetbsg}
\sum_{P_k \in W} \sum_{i \in P_k} x_{ij_0} + \sum ^c _{j = j_t + 1} \sum _{i \in V} x_{ij} \leq \alpha(W') w_{j_0} + w_{j_t + 1} \quad \forall j_0 \leq j_t, \; j_t = q - \alpha(W')
\end{equation}

Once again, since the size $\alpha$ of the maximum independent set of a graph is NP-hard to calculate, the subgraph induced by $W'$ is chosen in such a way that this number is trivial to obtain. This inequality is then specialized with $W'$ inducing either a path or a hole in $G'$, having $\alpha(W)$ equal to $\ceil{|W| / 2}$ and $\floor{|W| / 2}$, yielding \textit{partition graph path inequalities} and \textit{partition graph hole inequalities}, respectively. These kind of inequalities easy to work with, as they have been extensively studied for the standard coloring problem, and they have not been applied to this problem in any previous work (such as \cite{frota2010branch}).

\textit{Partition graph clique inequalities} predicate over a set of nodes which are all adjacent to each other or belong to the same partition, which is the very definition of an \textit{extended clique}. Since partition graph clique inequalities are more restrictive than the extended clique ones, as they require that every node in every partition involved is part of the clique, the former will not be considered and we will be applying only the latter.

Note that partition graph independent set inequalities are less frequent than component independent set ones, since $G'$ tends to be less dense as partition sizes increase; however, the former are much stronger as they impose restrictions over all the nodes in the partitions covered, instead of over a single node per partition.

\section{Block color inequalities}

Block color inequalities arise from the symmetry breaking constraints \ref{eqn:lowerlabel} $w_j \geq w_{j+1}$. Given a partition $P_k$ and a color $j_0$, every coloring of partition $P_k$ using label $j > j_0$ requires color $j_0$ to be already used in the graph, since \ref{eqn:lowerlabel} implies that a color cannot be used unless all previous ones had ben used. 

As only a single $x_{ij}$ is set in every partition, or equivalently, exactly one node is painted in each partition, the following inequality holds:

\begin{equation}
\label{ineq:blockcp}
\sum_{j \geq j_0} \sum_{i \in P_k} x_{ij} \leq w_{j_0} \quad \forall P_k \in P, j_0 \in C
\end{equation}

These $c.q$ inequalities are extremely easy to generate and, as will be analyzed further in this work, have proven to greatly improve the cutting planes scheme.

\section{Conclusiones}

\begin{frame}{Conclusiones}
\begin{itemize}
\item<2->{El algoritmo desarrollado espec�ficamente para este problema mejora los resultados provistos por un framework gen�rico como es Cplex.}
\item<3->{\textsc{DSatur} particionado arroja excelentes resultados iniciales, corresponde al Branch and Cut lograr y principalmente probar la optimalidad.}
\item<4->{Distintos modelos para un mismo problema pueden comportarse de manera diametralmente opuesta seg�n las instancias que resuelven.}
\end{itemize}
\end{frame}

\begin{frame}{Trabajo a Futuro}
\begin{itemize}
\item<2->{Realizar un an�lisis te�rico m�s detallado del poliedro, hallando su dimensi�n y caracterizando facetas, as� como buscando nuevas desigualdades v�lidas.}
\item<3->{Convertir dichas desigualdades en nuevos planos de corte que ayuden al Branch and Cut a demostrar optimalidad m�s r�pidamente.}
\item<4->{Hacer un an�lisis m�s detallado de las variantes de \textsc{DSatur} como algoritmo per se y no como un componente de otro.}
\item<5->{Analizar la performance del branch and cut sobre otras familias de grafos, especialmente aquellas resultantes de instancias reales del min-RWA.}
\end{itemize}
\end{frame}

\begin{frame}{Preguntas}
\begin{center}
\Huge{\textbf{�?}}
\end{center}
\end{frame}

\begin{frame}{That's all folks...}
\begin{center}
\huge{Gracias!}
\end{center}
\end{frame}
\section{Resoluci�n}
\begin{frame}{Resoluci�n}

Una manera de resolver un problema de programaci�n lineal entera consiste en aplicar un algoritmo de \textit{branch and cut}, el cual es una combinaci�n de las t�cnicas de \alert<2>{planos de corte} y de \alert<3>{branch and bound}.
\vskip 5pt
\uncover<2->{La primera se basa en resolver el problema de programaci�n lineal \textbf{sin} las restricciones de integralidad, eliminar la solucci�n fraccionaria con alg�n criterio, y repetir el proceso hasta llegar a una soluci�n �ptima entera.}
\vskip 5pt
\uncover<3->{La segunda subdivide el problema sucesivamente en otros m�s peque�os, eliminando ciertas soluciones fraccionarias, y manteniendo durante el recorrido del �rbol generado una cota superior y otra inferior para el �ptimo buscado.}

\end{frame} 

\begin{frame}{Componentes}

Un algoritmo de branch and cut consta entonces, de los siguientes componentes:

\begin{itemize}
\item<2->{\textbf{Algoritmos de separaci�n}, para remover soluciones fraccionales aplicando planos de corte construidos a partir de desigualdades v�lidas}
\item<3->{\textbf{Estrategias de branching}, para decidir con qu� criterio se subdivide el problema a cada nodo del �rbol}
\item<4->{\textbf{Heur�sticas inicial y primal}, para contar con soluciones enteras factibles durante el recorrido del �rbol, que act�an como cotas superiores para el �ptimo.}
\end{itemize}

\end{frame} 

\subsection{Planos de corte}

\begin{frame} 
\frametitle{Desigualdades v�lidas para PCP}

\begin{itemize}

\item{En una clique extendida, cada nodo debe tener un color distinto.
\lpineq{\sum_{i \in K} x_{ij_0} \leq w_{j_0}}{\forall j_0 \in C}
\extclique
\uncover<2>{Usamos un algoritmo goloso basado en los valores de las variables y los grados de los nodos para construir los planos de corte correspondientes.}
}

\end{itemize}

\end{frame} 


\begin{frame} 
\frametitle{Desigualdades v�lidas para PCP}

\begin{itemize}

\item{Una partici�n no puede colorearse con el color $j_0$ a menos que todos los anteriores ya hayan sido usados.
\lpineq{\sum_{i \in p_0}\sum_{j \geq j_0} x_{ij} \leq w_{j_0}}{\forall p_0 \in P, j_0 \in C}}

\uncover<2->{
Hay solamente $|P| \times |C|$, con lo que pueden resolverse mediante simple enumeraci�n.
}

\end{itemize}

\uncover<3->{
\begin{centerblock}{}
Estas desigualdades v�lidas, junto a las de clique extendida, fueron las que m�s contribuyeron al algoritmo de planos de corte implementado.
\end{centerblock}
}

\end{frame} 

\begin{frame} 
\frametitle{Desigualdades v�lidas para PCP}

\begin{itemize}

\item{Dado un conjunto independiente m�ximo $I$ de tama�o $\alpha$ tal que cada nodo est� en una partici�n distinta, a lo sumo $\alpha$ nodos pueden tener el mismo color.
\lpineq{\sum _{i \in I} x_{ij_0} \leq \alpha w_{j_0}}{\forall j_0 \in C}

\uncover<2->{
Especializamos esta desigualdad tomando subgrafos cuyos conjunto independientes m�ximos son f�ciles de calcular.

\begin{itemize}
\item{Component paths}
\item{Component holes}
\end{itemize}
}

\uncover<3->{Nuevamente usamos un algoritmo goloso para construir estos planos de corte, acotando la cantidad de veces que cada nodo y cada eje puede ser visitado.}}

\end{itemize}
\end{frame} 

\begin{frame} 
\frametitle{Desigualdades v�lidas para PCP}

\begin{itemize}

\item Dado un grafo, definimos su \textit{grafo de particiones} como un grafo que tiene un nodo por cada partici�n del original, y dos nodos son adyacentes sii todos los nodos de las dos particiones eran adyacentes entre s�:
\uncover<2->{
	\alt<2>
	{\togprime}
	{\gprime}
}

\uncover<4->{Las desigualdades de conjunto independiente se pueden aplicar sobre el grafo de particiones y llevarse al grafo original.}
\vfill
\end{itemize}

\end{frame} 

\subsection{Estrategia de branching}

\begin{frame}{Estrategia de Branching}

Lo siguiente es definir una estrategia de branching, que determina c�mo generar los subproblemas a partir de un nodo del �rbol.
\vskip 5pt
\uncover<2->{La estrategia t�pica es tomar una variable con valor fraccionario en la soluci�n de la relajaci�n, y forzar a que tome valor $0$ o $1$ en cada hijo.
\vskip 5pt

\begin{figure}[h]
	\centering
	\branchingtree
\end{figure}
}

\end{frame} 

\begin{frame}{Estrategia de Branching en PCP}

En PCP usamos como criterio de branching seleccionar un nodo de una partici�n sin colorear y asignarle un color distinto entre todos los posibles en los subproblemas:

\begin{figure}[h]
	\centering
	\pcpbranchingtree
\end{figure}

\uncover<2->{El nodo elegido a colorear es el que tiene mayor grado de saturaci�n, es decir, distintos colores usados para sus vecinos.}

\end{frame} 
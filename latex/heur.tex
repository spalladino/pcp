%!TEX root = pcp.tex

\chapter{Enumeration algorithms}
\label{sec:heur}

Implicit enumeration algorithms walk through all possible colorings for the graph, restricting the solution set as much as possible and pruning non-optimal solutions using known bounds. In this chapter we will review implicit enumeration techniques for the classical coloring problem and discuss different generalizations for \PCP{}.

\section{Classical scheme}

A classical scheme for enumeration algorithms is presented in \ref{alg:enumeration}.

\begin{algorithm}
\caption{Classical coloring implicit enumeration scheme for simple graphs $G = <V,E>$}
\label{alg:enumeration}

\begin{algorithmic}
\CALL color(0,1)

\PROC{color(painted, label)}
	\IF{current coloring is no better than best coloring}
		\STATE prune current solutions subtree
	\ELSIF{\textit{painted} equals to $|V|$}
		\STATE update best coloring with current coloring
	\ELSE
		\CALL paintnext(painted,label)
	\ENDIF
\ENDPROC

\PROC{paintnext(painted, label)}
		\STATE pick next uncolored \textit{node} to color		
		\FOR{j = 1 to \textit{label}}
			\IF{can paint \textit{node} with color $j$}
				\STATE assign color $j$ to \textit{node}
				\CALL color(painted+1, label)
				\STATE uncolor \textit{node}
			\ENDIF
		\ENDFOR
		\COMMENT{try coloring \textit{node} with a new label}
		\STATE assign color \textit{label} + 1 to \textit{node}
		\CALL color(painted+1, label+1)
		\STATE uncolor \textit{node}
\ENDPROC

\end{algorithmic}
\end{algorithm}

The algorithm picks a node to be colored in each recursive call, attempting to color it with one of the already used labels if possible; it also assigns a fresh color to the node, in order to explore all possible colorings for the graph. Note that several symmetric colorings are left out of the exploration. 

At every iteration, the partial solution is pruned if the coloring is using as many labels as the best coloring found by the algorithm, as this implies that the best solution cannot be improved using the current one. The algorithm runs until all possible colorings have been explored, therefore it effectively returns a minimum coloring of the graph.

The strategy used for picking the node to be colored in each recursive call gives place to different algorithms. A very simple strategy is to use the degree of the node, coloring nodes with highest degree first, based on the assumption that difficult nodes should be handled early.

Another algorithm, which is one of the most widely used for the coloring problem, is \textsc{dsatur}\cite{brelaz1979new}. This algorithm always picks the node with the highest degree of saturation\footnote{Number of different colors used in $N(v)$ for a node $v \in V$.}, using different strategies for tie-breaking, such as picking the node with the largest number of uncolored neighbours\cite{sewell1996improved}. It has proved to be one of the best enumeration algorithms available.

For an implementation of the \textsc{dsatur} algorithm, we used the code provided by Trick in \cite{trickdsatur}, which we ported to Java and adapted for partitioned coloring as described in \ref{subsec:heur:enumpcp}.

\section{Enumerating partitioned colorings}
\label{subsec:heur:enumpcp}

The previous scheme must be modified in order to generate valid partitioned colorings. A simple modification would be to simply pick a new partition instead of a new node on each recursive call, and iterate over all nodes in the partition using all candidate labels. This modification is presented in algorithm \ref{alg:pcpenumerationwrong}. 

\begin{algorithm}
\caption{Modification of enumeration scheme for partitioned graphs $G = <V,E,P>$, picking partitions on every call}
\label{alg:pcpenumerationwrong}

\begin{algorithmic}
	\STATE [...] \\
	\PROC{paintnext(painted, label)}
		\STATE pick next uncolored \textit{partition} to color		
		\FORALL{\textit{node} in \textit{partition}}
			\FOR{j = 1 to \textit{label}}
				\IF{can paint \textit{node} with color $j$}
					\STATE assign color $j$ to \textit{node}
					\CALL color(painted+1, label)
					\STATE uncolor \textit{node}
				\ENDIF
			\ENDFOR
			\COMMENT{try coloring \textit{node} with a new label}
			\STATE assign color \textit{label} + 1 to \textit{node}
			\CALL color(painted+1, label+1)
			\STATE uncolor \textit{node}
		\ENDFOR
	\ENDPROC	
\end{algorithmic}
\end{algorithm}

However, this modification imposes that all the nodes within the same partition are explored together in the enumeration, regardless of the criteria being used to choose each candidate. For example, if a largest-degree criteria is used, and the remaining partitions (with their nodes' degrees) are $P_1 \{v_1(10),v_2(1)\}$ and $P_2 \{v_3(5)\}$, algorithm \ref{alg:pcpenumerationwrong} would enumerate nodes $v_1, v_2, v_3$ instead of $v_1, v_3, v_2$. This severely hurts the efectiveness of the strategy being used.

Therefore, we propose another modification, presented in algorithm \ref{alg:pcpenumeration}. In this case we use the original enumeration scheme, picking a node from an unpainted partition on every call, but before returning from the procedure we create another branch in which we do not color the chosen node, so that the partition can be later colored using another node.

\begin{algorithm}
\caption{Partitioned coloring implicit enumeration scheme for partitioned graphs $G = <V,E,P>$}
\label{alg:pcpenumeration}

\begin{algorithmic}
\CALL color(0,1)

\PROC{color(painted, label)}
	\IF{current coloring is greater than or equal best coloring}
		\STATE prune current solutions subtree
	\ELSIF{\textit{painted} equals to $|P|$}
		\STATE update best coloring with current coloring
	\ELSE
		\CALL paintnext(painted, label)
	\ENDIF
\ENDPROC

\PROC{paintnext(painted, label)}
		\STATE pick next uncolored \textit{node} to color	from all uncolored partitions	
		\FOR{j = 1 to \textit{label}}
			\IF{can paint \textit{node} with color $j$}
				\STATE assign color $j$ to \textit{node}
				\CALL color(painted+1, label)
				\STATE uncolor \textit{node}
			\ENDIF
		\ENDFOR
		
		\COMMENT{try coloring \textit{node} with a new label}
		\STATE assign color \textit{label} + 1 to \textit{node}
		\CALL color(painted+1, label+1)
		\STATE uncolor \textit{node}
		
		\COMMENT{leave node unpainted}
		\IF{there are other nodes left in the partition}
			\STATE mark \textit{node} as unavailable
			\CALL color(painted, label)
			\STATE mark \textit{node} as available again
		\ENDIF
		
\ENDPROC

\end{algorithmic}
\end{algorithm}

It is within this scheme that we embedded our two different strategies based on degree of saturation for partition coloring.

\section{Partitioned DSatur}
\label{subsec:heur:pdsatur}

Classical \textsc{dsatur} picks the node with the highest color degree on each iteration, based on the assumption that nodes difficult to color should be handled first, which usually works well for most heuristics. In the case of partition coloring, as suggested in \cite{Li00thepartition}, nodes with lower degree are easier to color and should be preferred within a partition; also, it is better to color larger partitions first in order to reduce the problem size as early as possible.

Based on these assumptions, we generalized two different versions for partitioned \textsc{dsatur}: \textit{easiest node} and \textit{hardest partition}.

\subsection{Easiest node}

The easiest node variant is based on the \textit{onestepCD} heuristic proposed in \cite{Li00thepartition}. In order to pick the node to color, the easiest node is chosen from every uncolored partition, where \textit{easiest} is defined in terms of lowest color degree, with tie breaking on lowest number of uncolored neighbours. From the resulting set, the node with the highest color degree is chosen, as in classic \textsc{dsatur}. 

In other words, this algorithm picks the hardest node from the set of the easiest nodes on each uncolored partition. Note that if every partition contains a single node, this algorithm behaves exactly as classic \textsc{dsatur}.

Also, in an attempt to explore different solutions earlier to obtain better upper bounds, we also implemented a randomized version of the algorithm. From the set of the easiest nodes in every partition, only half of the time the hardest node is chosen, the other half another candidate is chosen with probability decreasing as its color degree decreases.

\subsection{Hardest partition}

Instead of choosing the easiest node from each partition and then picking the hardest one to color, this algorithm first chooses the hardest partition to handle, and then picks the easiest node from that partition. While the easiest node is picked using the usual color degree criteria, choosing the hardest partition requires a new strategy. 

Therefore, in order to determine the hardest partition to color, we experimented with combinations of the following metrics:
\begin{itemize}
	\item{Color degree of the partition, defined as the number of different colors adjacent to all of the nodes in the partition; considering that a larger color degree implies a harder partition to color}
	\item{Size of the partition, as a larger partition being colored earlier helps reducing the problem size, therefore, the larger the partition the earlier it should be handled}
	\item{Number of uncolored partitions adjacent to the partition, equivalent to the tie breaking criteria used for classic \textsc{dsatur}} 
\end{itemize}

Note that using the color degree of the easiest node in the partition as a criteria, under the premise that a partition can be considered to be as hard as its easiest node, yields the \textit{easiest node} algorithm.

Different combinations of these criteria, as well as the different variants of the algorithm, will be compared in chapter \ref{subsec:resultspdsatur}.

\subsection{Ad-hoc modifications}

As it will be presented in chapter \ref{sec:bnc}, bounded versions of this algorithm will be used during the branch and cut algorithm as an initial heuristic, primal heuristic and subtree pruning. As such, certain adaptations were made to the algorithm.

Since the preprocessing step identifies a large clique before any coloring is performed, the algorithm supports forcing a set of partitions to be assigned a specific set of colors; therefore all partitions $P_{K_1},\ldots,P_{K_\omega}$ contained in the initial clique are colored with labels $1,\ldots,\omega$. Since we still have to determine which node to color in each partition, we have to try every possible way of picking a single node from each partition to be colored. Therefore, the algorithm colors $P_{K_1},\ldots,P_{K_\omega}$ first, trying all possible $\prod_{i=1}^{\omega} |P_{K_i}|$ node combinations, before proceeding with the partitioned \textsc{dsatur} on the rest of the graph.

Also, since the algorithm is used as a primal heuristic within the branch and cut tree (see \ref{subsec:alg:primal}), it supports forcing the coloring of certain nodes, keeping those assignments fixed during the enumeration process. In case certain node-color combinations are forbidden due to the restrictions imposed during branching, solutions assigning those combinations are not explored in the algorithm.

In case symmetry breaking constraints \ref{eqn:symnodecount} or \ref{eqn:minlabel} are used, the color classes obtained by the algorithm are sorted based on node count or minimum partition index respectively, so the obtained solution can be successfully injected into the branch and cut scheme.


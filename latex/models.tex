%!TEX root = pcp.tex

\section{Model}
\label{sec:model}

In this section we will present various binary integer programming formulations for the \PCP{}, generalized from the CP model presented in \cite{mendez2008cutting}, which will be used in the branch and cut algorithm.

\subsection{Formulation}
\label{subsec:model:formulation}

Let $G = <V,E,P>$ a partitioned graph, being $V$ the set of nodes numbered from $1$ to $n$, $E$ the set of $m$ edges, and $P$ the set of partitions numbered from $1$ to $q$; and let $C$ be the set of color labels numbered from $1$ to $n$.

The standard coloring problem formulation, SCP, uses the following $(n + 1) \times c$ binary variables, where $i \in V$ and $j \in C$:
\begin{itemize}
\item $x_{ij}$ equals $1$ if and only if the node $i$ is colored with label $j$
\item $w_j$ equals $1$ if there is at least one node in the graph which uses color $j$
\end{itemize}

The goal is to minimize the total number of colors used, this is, the number of $w_j$ variables set to $1$.

\begin{align}
\text{\textsc{minimize}} \quad & \sum_{j \in C} w_j \nonumber \\
\text{\textsc{subject to}} \quad & \sum_{j \in C} x_{ij} = 1 \quad \forall i \in V \label{eqn:vertexsum} \\
& x_{ij} + x_{kj} \leq w_j \quad \forall (i,k) \in E, \; \forall j \in C \sumheight \label{eqn:adjscolor}\\
& x_{ij}, w_{j} \in \{0,1\} \quad \forall i \in V, \; \forall j \in C \sumheight \nonumber
\end{align}

Equation \ref{eqn:adjscolor} implies that two adjacent vertices may not use the same color, and also ensures that any variable $x_{ij}$ set to $1$ will cause $w_j$ to be set as well.

Restriction \ref{eqn:vertexsum} requires that every node is assigned exactly one color. Since the difference between standard coloring and partition coloring relies solely in the fact that, in the latter, only one node per partition must be colored, adjusting this last restriction provides a simple model for PCP.

\begin{align}
\text{\textsc{minimize}} \quad & \sum_{j \in C} w_j \label{eqn:obj} \\
\text{\textsc{subject to}} \quad & \sum_{i \in P_k} \sum_{j \in C} x_{ij} = 1 \quad \forall P_k \in P \label{eqn:partsum} \\
& x_{ij} + x_{kj} \leq w_j \quad \forall (i,k) \in E, \; \forall j \in C \sumheight \label{eqn:adjscolorp}\\
& x_{ij}, w_{j} \in \{0,1\} \quad \forall i \in V, \; \forall j \in C \sumheight \nonumber
\end{align}

This model has $(n + 1) \times c$ variables as well, $q$ restrictions \ref{eqn:partsum} and $m \times c$ \ref{eqn:adjscolorp} restrictions, plus all integrality constraints.

\subsection{Variants}

There are several variants for the previously presented model for \PCP{}, all of which provide valid partition colorings. We will explore different alternatives to the basic formulation (\ref{eqn:obj},\ref{eqn:partsum},\ref{eqn:adjscolorp}).

\subsubsection{Color a single node per partition}

Restriction \ref{eqn:partsum} can be relaxed by requiring that at least one node is colored per partition, instead of requiring that exactly one node is colored. Even more, we may also accept colorings in which a single node is assigned more than one color. 

\begin{equation}
\label{eqn:partsumgeq}
\sum_{i \in P_k} \sum_{j \in C} x_{ij} \geq 1
\end{equation}

The minimization of the total number of colors used will ensure that no additional colors will be used, and a valid coloring can be extracted from the resulting solution by picking any color from any node on every partition, as no color conflicts will occur since restriction \ref{eqn:adjscolorp} is still in place.

\subsubsection{Color conflicts}

An alternative to restriction \ref{eqn:adjscolorp}, for both preventing color conflicts and set $w_j$ variables upon usage of color $j$, is to decouple this two concepts into different restrictions. Therefore, instead of restricting $x_{ij} + x_{kj} \leq w_j$ for every edge, we may require:

\begin{align}
x_{ij} + x_{kj} \leq 1 \quad &\forall (i,k) \in E, \; \forall j \in C \label{eqn:adjscolorpone}\\
x_{ij} \leq w_{j} \quad &\forall i \in V, \; \forall j \in C \label{eqn:nodelessthanwj}
\end{align}

This alternative, however, yields an even larger number of restrictions than the original \ref{eqn:adjscolorp}. Looking forward to reducing the number of equations in the model, we propose the following alternative:

\begin{equation}
\label{eqn:adjsperpart}
\sum_{i \in P_k \cap N(i_0)} x_{ij} + x_{i_0j} \leq w_j \quad \forall j \in C, \; \forall P_k \in P, \; \forall i_0 \in V 
\end{equation}

This equation establishes that either node $i_0$ may use color $j$, or at most one neighbor in every adjacent partition may use it (as no more than a single node may be colored per partition). This formulation considerably reduces the amount of restrictions when partition sizes are large; otherwise, it does not report any benefits over the original version \ref{eqn:adjscolorp}.

Another variant that further reduces the number of restrictions makes heavy use of the maximum number of nodes that may use the same color within a neighbourhood:

\begin{equation}
\label{eqn:adjsneighb}
\sum_{i \in N(i_0)} x_{i_0j} + r * x_{i_0j} \leq r * w_j \quad \forall j \in C, \; \forall i_0 \in V 
\end{equation}

A simple value for $r$ could be the number of different partitions in the neighbourhood of node $i_0$. In that case, the restriction implies that either node $i_0$ uses color $j$, or at most $r$ nodes in its neighbourhood may use it simultaneously.

However, we may tighten the restriction by replacing $r$ by the number of components in an extended clique coverage of the node's neighbourhood, as this value provides an upper bound for the maximum number of colors that can be used for a set of nodes. We use a simple greedy heuristic to generate a standard clique cover of the partition graph induced by the neighbourhood of $i_0$ to obtain the value $r$.

This restriction generates a much lower number of equations in dense graphs, as it requires just $c$ restrictions per node instead of per edge.

\subsection{Breaking symmetry}
\label{subsec:model:symmetry}

One of the main issues with the model presented is that it allows for multiple symmetric solutions; the same happens with the standard coloring model SCP. 

Since it does not matter which color labels are used in the coloring, any $\chi$ colors can be used, resulting in $P(c,k)$\footnote{Number of different ordered subsets of size $k$ from a set of size $c$, equals to $c! / (c - k)!$} different solutions for every equivalent coloring; therefore, introducing additional constraints with the purpose of removing symmetric solutions is expected to produce an improvement in the algorithm, as it greatly limits the solution space.

The easiest restriction to generate is to prevent color $j+1$ from being used unless color $j$ is used in the coloring. This ensures that only colors with labels $1 \ldots k$ are used in a k-coloring, leaving always the last $k+1 \ldots c$ unused, thus reducing the number of symmetric solutions from $P(c,k)$ to $k!$.

\begin{equation}
\label{eqn:lowerlabel}
w_j \geq w_{j+1} \quad \forall 1 \leq j < c 
\end{equation}

A stricter requirement that can be imposed is to forbid having more vertices colored with label $j+1$ than with label $j$. This removes all symmetric solutions in the case that every color class has a different node count, which is a vast improvement from the previous restriction.

\begin{equation}
\label{eqn:symnodecount}
\sum_{i \in V} x_{ij} \geq \sum_{i \in V} x_{ij+1} \quad \forall 1 \leq j < c 
\end{equation}

However, this restriction allows symmetric solutions by exchanging labels between color classes with same cardinal, which is likely to occur in regular graphs. 

In order to further remove symmetric solutions, it is possible to enforce the following restriction, which implies that among all possible assignments to the set of partitions defined by color classes, only the one that assigns the lowest possible color label to the partition with the lowest index is used: 

\begin{align}
x_{ij} = 0 \quad &\forall j > p(i) + 1 \label{eqn:nodeszero} \\
x_{ij} \leq \sum_{l = j-1}^{k-1} \sum_{u \in P_l} x_{uj-1} \quad &\forall 1 < k \leq q, \; \forall i \in P_k, \; \forall 1 < j \leq k \label{eqn:minlabel}
\end{align}

Equation \ref{eqn:nodeszero} establishes that color with label $j$ may not be used for a partition with index greater than $j$; whereas equation \ref{eqn:minlabel} imposes that color $j$ cannot be used for a partition unless color $j-1$ was used in a previous partition.

As an example, suppose a graph such that $P = \{ P_1, \ldots, P_q \}$ and $P_k = \{ x_{2k-1}, x_{2k} \}$, this is, every partition has two nodes. The first instantiations of restriction \ref{eqn:minlabel} would be:

\begin{align*}
k = 2, i = 3, j = 2 \qquad & x_{3,2} \leq \sum_{u \in P_1} x_{u,1} = x_{1,1} + x_{2,1} \\
k = 2, i = 4, j = 2 \qquad & x_{4,2} \leq \sum_{u \in P_1} x_{u,1} = x_{1,1} + x_{2,1} \\
&\\
k = 3, i = 5, j = 2 \qquad & x_{5,2} \leq \sum_{u \in P_1} x_{u,1} + \sum_{u \in P_2} x_{u,1} = x_{1,1} + x_{2,1} + x_{3,1} + x_{4,1} \\
k = 3, i = 6, j = 2 \qquad & x_{6,2} \leq \sum_{u \in P_1} x_{u,1} + \sum_{u \in P_2} x_{u,1} = x_{1,1} + x_{2,1} + x_{3,1} + x_{4,1} \\
&\\
k = 3, i = 5, j = 3 \qquad & x_{5,3} \leq \sum_{u \in P_2} x_{u,2} = x_{3,2} + x_{4,2} \\
k = 3, i = 6, j = 3 \qquad & x_{6,3} \leq \sum_{u \in P_2} x_{u,2} = x_{3,2} + x_{4,2} \\
\vdots \qquad & \vdots
\end{align*}

\subsection{Objective function}
\label{subsec:model:obj}

Another way of reducing the number of symmetric solutions is to prefer lower-label colors in the objective function, therefore choosing the lowest labels possible in each coloring. To achieve this we simply add higher coefficients to lower $w_j$ variables in the objective:

\begin{equation}
\label{eqn:objlinear}
\text{\textsc{minimize}} \quad \sum_{j \in C} j \times w_j 
\end{equation}

However, early tests showed that this objective function offers poor computational results, so it was early discarded from the set of possible formulations.

\subsection{Strengthening the model}

There are other inequalities, besides the already mentioned symmetry breaking ones, that are not necessary for the formulation of a valid coloring, yet they strengthen the model relaxation, helping during the branch and cut process. These inequalities are entirely optional in the formulation, their inclusion depends strictly in the tradeoff between building a more complex model that takes more time to solve and strengthening its relaxation so the algorithm's overall performance is increased.

A simple restriction, which is already implied by the objective function, consists in preventing a $w_j$ variable from being set unless there is a node painted with color $j$:

\begin{equation}
\label{eqn:wjleqsumcolor}
w_j \leq \sum_{i \in V} x_{ij} \quad \forall j \in C
\end{equation}

The usage of this restriction will become clear when we present the branching strategies in \ref{subsec:alg:branching}, in which we directly enforce bounds on the $w_j$ variables based on additional information on the coloring problem.

Another equation, which improved the obtained results in \cite{mendez2006branch}, avoids the generation of fractional solutions such as $x_{ij} = 1/c$:
\begin{equation}
\label{eqn:wjgeqsumnode}
\sum_{j \in C} w_j \geq \sum_{j \in C} j x_{ij} \quad \forall i \in V
\end{equation}

The rationale behind that restriction is that only one $x_{ij}$ may be set per node, therefore the right hand side of the inequality is the label of the color assigned to node $i$, which cannot be greater than the total of colors used.

This restriction can be further strengthened by extending the sum of the $x_{ij}$ variables over partitions instead of nodes:
\begin{equation}
\label{eqn:wjgeqsumpart}
\sum_{j \in C} w_j \geq \sum_{j \in C} \sum_{i \in P_k} j x_{ij} \quad \forall P_k \in P
\end{equation}
